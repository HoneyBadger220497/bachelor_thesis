Das Ising-Modell war das erste, nichttriviale Modell eines Ferromagneten und wurde erstmals von seinem Namensgeber Ernst Ising im Rahmen seiner Doktorarbeit 1925 genauer untersucht. Wolfgang Nolting beschreibt die heutige Rolle des Insing-Modells in seinem Lehrbuch wie folgt:
\begin{quote}
``Es hat sich vielmehr zum allgemeinen Demonstatrionsmodell der Statistischen Physik entwickelt. Als das wohl einfachste mikroskopische Modell, das einen Phasenübergang zweiter Ordnung für d $\geq2$ vollzieht, steht es im Mittelpunkt vieler Überlegungen und Untersuchungen der Theorie der Phasenübergänge und kritischen Phänomene.'' \cite{StatPhys_Nolting_K4}
\end{quote}
Bei seinen damaligen Untersuchungen konnte Ising nur das eindimensionale Modell exakt lösen. Für das zweidimensionale Modell legte der norwegische Physiker Lars Onsager erstmals 1944 eine exakte Lösung vor und Aussagen über das dreidimensionale Modell sind bis heute nur numerisch zugänglich.
Im Jahre 1948 publizierte Onsager zudem die in \eqref{eq: expected Magnetisation} angegebene, exakte Formel für die Temperaturabhängigkeit der spontane Magnetisierung des zweidimensionale Ising-Modells, ohne aber seine Herleitung zu veröffentlichen. Erst vier Jahre später, in 1952, gelang es dem Physiker Chen Ning Yang die Formel ebenfalls abzuleiten. Seine Herleitung gilt allerdings als ausgesprochen kompliziert \cite{Montroll_Potts_Ward}.\\
\begin{equation} \label{eq: expected Magnetisation}
\mathcal{M}_S = \left\{ \begin{array}{cr} \mathcal{M}_0\left(1-\frac{1}{\sinh(\frac{2J}{k_B\,T})^4}\right)^{\frac{1}{8}} & \text{für } T < T_c \\ 0 &\text{für } T > T_c   \end{array} \right.
\end{equation}
\begin{quote}
\small
$\mathcal{M}_0$ beschreibt dabei eine Sättigungs-Magentisierung für $T = 0\,\mathrm{K}$ und $J$ ist eine Kopplungskonsante, genannt Austauschintegral, welche die Wechselwirkung zwischen den Spins der Gitteratome beschreibt. Eine ausführlichere Beschreibung des Modells findet sich in Abschnitt \ref{sec: fundamentals}. 
\end{quote}

\noindent 
Die exakte Lösung eines Statistischen Modells ist insofern wichtig, da sie tiefere Einsichten in das Modell liefert und es als Test-Modell für numerische Näherungsmethoden befähigt, mithilfe derer dann nicht exakt lösbare Modelle untersuchte werden können.\\
Das Ziel dieser Arbeit ist nun die Angabe einer möglichst einfachen Herleitung des exakten Ausdrucks für die spontanen Magnetisierung des zweidimensionalen Ising-Modells. Die Koppelung $J$ wird für die Arbeit als homogen Isotrop angenommen und es werden periodische Randbedingungen festgelegt. Die Idee für diesen einfacheren Zugang geht schließlich auf die Arbeiten von Stuart Samuels zurück, der 1980 das Ising-Modell mithilfe von Graßmann-Zahlen beschrieb und zeigte, dass sich so die auftauchenden Probleme statistischer oder graphischer Natur auf einfach handhabbare algebraische Ausdrücke abbilden lassen \cite{StuartSamuel1} \cite{StuartSamuel2}. Zudem bedient sich der hier dargestellte Beweis einer Idee aus dem Paper \cite{Montroll_Potts_Ward} von Elliot W.Montroll, Renfrey B. Potts und John C.Ward, welche das Ising-Modell mithilfe von Pfaffschen Determinanten 1962 untersuchten und denen es gelang die Spin-Spin-Korrelation für zwei beliebig weit entfernte Spins auf elegante Weise mit dem starken Szegö Grenzwertsatz zu berechnen. 
% Die hier angewandten Methoden sind zudem nicht allein auf das Ising-Modell beschränkt, siehe hierzu \cite{StuartSamuel1}.   

\noindent Der Hauptteil der Arbeit gliedert sich in fünf Abschnitte. Die ersten zwei machen dabei circa ein Drittel der Arbeit aus und stellen die nötigen Grundlagen bereit. In Abschnitt \ref{sec: fundamentals} finden sich die physikalischen Grundlagen. Hier wird das Ising-Modell genauer beschrieben, die notwendigen Begriffe aus der Statistischen Physik erklärt und der in der Arbeit zentrale Begriff der spontanen Magnetisierung eingeführt, sowie der Zusammenhang zur Spin-Spin-Korrelation hergestellt. In Abschnitt \ref{sec: grassmann} finden sich die mathematischen Grundlagen. Es wird die notwendige Theorie zu Graßmann-Zahlen, Graßmann-Algebra und Pfaffschen Determinanten aufbereitet. Für den mathematisch interessierten Leser finden sich im Anhang noch ein paar Beweise zu den allgemein bekannte Eigenschaften Pfaffscher Determinanten, welche ausschließlich im Rahmen der Graßmann-Zahlen geführt werden. Im Gegensatz zu Standartbeweisen, benötigen diese keine Vorwissen über die Spektraleigenschaften antisymmetrischer Matrizen. \\
In Abschnitt \ref{sec: reformulation} wird die Berechnung der Zustandssumme des Ising-Modells als Problem des Zählens von geeigneten Graphen auf dem rechteckigen Gitter formuliert und dann mithilfe der Graßmann-Zahlen auf ein algebraisches Problem abgebildet. Zudem wird, über die Hochtemperatur-Darstellung des Ising-Modells, ein Ausdruck für die Spin-Spin-Korrelation abgeleitet, welcher sich als modifizierte Zustandssumme interpretieren lässt und somit das Problem der Magnetisierung auf die zuvor erwähnte algebraische Darstellung der Zustandssumme zurückführt. Abschließend wird noch eine diskrete Fouriertransformation der Graßmann-Zahlen auf dem Ising-Gitter besprochen, welche die Lösung des algebraische Problems stark vereinfacht.\\
Das letzte Drittel der Arbeit machen die Abschnitte \ref{sec: calcC} und \ref{sec: calcM} aus. In Ersterem wird über den algebraischen Ausdruck für die Zustandssumme die Spin-Spin-Korrelation berechnet und der Übergang in den Thermodynamischen Limes vollzogen. Dieser Abschnitt ist der wohl technischste Teil der Arbeit. Hierbei wird insbesondere versucht die Definition der Pfaffschen Determinante in den Vordergrund zu rücken, anstatt diese mithilfe der Relation $\pf{\bm{A}}^2 = \det{\bm{A}}$ auszuwerten. Dieser Zugang erweist sich als elegant und verhindert zudem Vorzeichenprobleme. 
In Abschnitt \ref{sec: calcM} wird letztlich die Magnetisierung über die Korrelation zweier beliebig weit entfernter Spins berechnet. Um den hier benötigten Grenzübergang elegant zu vollziehen kommt der starke Grenzwertsatz von Szegö zum Einsatz. Anhand der Magnetisierung wird das kritische Verhalten und der Phasenübergang des zweidimensionale Ising-Gitters identifiziert, sowie die kritische Temperatur $T_C$ bestimmt.\\
In Abschnitt \ref{sec: conclusio} findet sich dann eine abschließende Analyse des Beweises sowie ein Diskussion weiterer Anwendungsmöglichkeiten für die hier verwendeten Methoden.

\subsection{Bemerkung zur Notation}
In der Arbeit werden Vektoren $\bm{\eta}$ und Matrizen $\bm{A}$ jeweils in ``Boldface''  geschrieben. Die Komponenten werden dann mit dem selben Buchstaben und einem tiefgestellten Index bezeichnet, zum Beispiel $A_{ij}$. Jedoch sind nicht alle indizierten Größen Einträge eines Vektors oder einer Matrix, wie zum Beispiel die klassischen Zufallsvariablen $\sigma_i$.
Lineare und billineare Abbildungen werden immer mit einem Großbuchstaben bezeichnet und die zugehörige Darstellende Matrix wird dem selben Buchsatben in ``Boldface'' notiertet. Speziell Permutation werden immer mit $P$ und die zugehörige Permutationsmatrix mit $\bm{P}$ bezeichnet. 
