
Die Ergebnisse \eqref{eq: result magnetisation low} und \eqref{eq: result magnetisation high} für die Magnetisierung unterhalb und oberhalb der kritischen Temperatur $T_C$ ergeben zusammengefasst die, erstmals von Lars Onsager aufgestellte, Formel für die spontanen Magnetisierung in Abhängigkeit der Temperatur.

\begin{grayframe}[frametitle = {Spontane Magnetisierung des 2d Ising-Modells}]
\begin{equation} \label{eq: result Magnetisation}
\mathcal{M}_S = \left\{ \begin{array}{cr} n \mu \left(1-\frac{1}{\sinh(\frac{2J}{k_B\,T})^4}\right)^{\frac{1}{8}} & \text{für } T < T_c \\ 0 &\text{für } T > T_c   \end{array} \right.
\end{equation}
\begin{equation} 
T_C  := \frac{2 J}{ \ln{1+\sqrt{2}} k_B}
\end{equation}
\end{grayframe}

\noindent Der Beweis ist damit abgeschlossen. Nebenbei wurde zudem ein Ausdruck für die Zustandssumme abgeleitet, der eine leichte, explizite Berechnung dieser und der Freien Energie des Modells ermöglicht. Dies wird in Appendix \ref{Appendix: Zustandsumme} vorgeführt.\\

\noindent Der Ansatz das Problem der Zählung geschlossener Graphen auf dem rechteckigen Gitter, mithilfe einer graphischen Überlegung, in ein algebraisches Problem für Graßmann Zahlen zu übersetzen, bietet nicht nur einen anschaulichen Zugang, sondern ermöglicht die Anwendung leicht handhabbarer algebraischer Methoden und Hilfsmittel. Zudem gelingt es mit diesem Ansatz nicht nur die Zustandsumme, sondern auch die Magnetisierung mit absehbarem Aufwand zu berechnen. 
% Der Erfolg des Ansatzes beruht dabei darauf die physikalische Größen, wie die Zustandssumme oder Spinn-Spin-Korrelationen, welche eine physikalische Interpretation besitzen aber mathemathisch schwer handhabbar sind, mit leichter zugänglichen mathemathischen Hilfsgrößen, wie dem Spurintegral einer geeigneten Graßmann-Dichte und Korreltationen geeigneter Graßmann-Funktionen, in Verbindung zu bringen. 
\\
Aus Sicht der Quantenfeldtheorie wurde das Problem mithilfe der Graßmann Variablen in ein Freies Fermionen Modell übersetzt, das dann mit den, in der QFT üblichen Methoden, behandelt werden kann. Dies war auch Stuart Samuels ursprünglicher Zugang. Dieser Ansatz ist zudem nicht nur auf das Ising-Modell beschränkt, sondern kann auch zu exakten Lösung andere Probleme wie ``planar closed-Packed Dimer Problems'' genutzt werden \cite{StuartSamuel1} \cite{StuartSamuel2}. \\ 
Der starke Grenzwertsatz von Szegö erweist sich als mächtiges Hilfsmittel wenn es um die Berechnung von Grenzwerten von Folgen von Töplitz-Determinanten geht. Töplitz-Matrizen stehen dabei in einem natürlichen Zusammenhang mit den in der Physik häufig auftretenden periodischen Randbedingungen. Die Notwendigkeit des Übergangs in ein unendlich großes System, um physikalische Aussagen treffen zu können, stellt dabei eine häufige Schwierigkeit dar.\\

\noindent Die erbrachte Herleitung weist jedoch einen Mangel auf, welche den hier gelieferten Beweis aus mathematischer Sicht unvollständig macht. Dies wurde in Abschnitt \ref{sec: Wahl der übrigen Koeffizienten} bereits angesprochen und in Appendix \ref{Appendix: Zustandsumme} genauer ausgeführt. Im Temperaturbereich $T < T_C$ hat der, über die Hochtemperatur-Darstellung hergeleitete, Ausdruck für die Zustandssumme zwar den richtigen Absolutbetrag aber das falsche Vorzeichen. Am kritischen Punkt für $T = T_C$ wechselt der berechnete Ausdruck für die Zustandssumme das Vorzeichen. Der Vorzeichenwechsel geht dabei stetig von statten. Eine negative Zustandssumme ist jedoch weder aus mathematischer, noch physikalischer Sicht, sinnvoll, sodass sich hier um einen Fehler handeln muss. Tatsächlich scheint es, als würde es am kritischen Punkt zu einer Art Perkolations-Phänomen kommen, das dafür sorgt, dass die Beiträge der, in Abschnitt \ref{sec: Wahl der übrigen Koeffizienten} eingeführten, äußere Graphen nicht mehr im Thermodynamischen Limes verschwinden. Denn diese wurden als einzige mit negativer Gewichtung gezählt. Konkret fehlt es hier an einer qualitativen Abschätzung für die Geschwindigkeit mit der die Beiträge dieser äußeren Graphen verschwinden und einer Abschätzung des Temperaturbereiches in dem diese tatsächlich verschwinden. Dem Autor sind bislang keine Arbeiten bekannt, die diese Themen behandeln würden und auch in anderen Herleitungen, welche die hier verwendeten Methoden zur Lösung des Ising-Modells benutzen, wie \cite{PostBac} \cite{Gandhi}, oder in den ursprunglichen Arbeiten von Stuart Samuel \cite{StuartSamuel1} \cite{StuartSamuel2}, wird dieses Detail nicht erwähnt. \\
Für die Berechnung der spontanen Magnetisierung scheint dieses Verhalten jedoch keinen Unterschied zu machen, da diese als Quotient zweier Zustandssummen ausgedrückt werden kann und das Vorzeichen somit keine Rolle spielt. Zudem kann mit den hier besprochenen Methoden auch ein für $T < T_C$ gültiger Ausdruck für die Zustandssumme unter Hinzunahme der sogenannten Tieftemperatur-Darstellung abgeleitet werden. Dies wird ebenfalls in Appendix \ref{Appendix: Zustandsumme} gezeigt und stellt keinen wirklichen Mehraufwand dar. 

