
\noindent Ziel dieser Arbeit ist die Angabe einer möglichst eleganten Herleitung für die exakte Formel der spontanen Magnetisierung des 2d Ising-Modells in Abhängigkeit von der Temperatur, welche erstmals von Lars Onsager aufgestellt wurde. Dies wird bewerkstelligt indem das graphisch-kombinatorische Problem, welches bei der Beschreibung der Zustandssumme und der Magnetisierung auftritt, mithilfe von Graßmann-Zahlen in ein handhabbares algebraisches Problem übersetzt wird. Mathematische Theoreme wie der starke Szegö Grenzwertsatz dienen zudem als praktische Werkzeuge um Grenzübergänge im Thermodynamischen Limes elegant zu behandeln. Die Arbeit beschränkt sich dabei auf ein homogenes isotropes zweidimensionales Ising-Gitter mit periodischen Randbedingungen. Die Ideen zu dieser Herleitung gehen auf die Arbeiten von Stuart Samuel, Elliot W.Montroll, Renfrey B. Potts und John C. Ward zurück. 