\subsection{Ferromagnetisches Ising-Modell} \label{sec: Ferromagnetisches Ising-Modell}

Dem Ising-Modell liegt die Vorstellung eines endlichen Gitters zugrunde, auf dessen Gitterpunkten permanente magnetische Momente $\mu_i$ lokalisiert sind. Diese sind alle entlang der selben Achse ausgerichtet, die Orientierung ist jedoch zufällig. Dies ist zum Beispiel ein sehr stark vereinfachtes Modell für einen magnetischen Isolator. Es kann als halbwegs realistisches Modell betrachtet, falls das Material eine stark uniaxiale Symmetrie aufweist, da dann eine Richtung für die magnetischen Momente im Material ausgezeichnet ist \cite{StatPhys_Nolting_K4}. Die magnetischen Momente können dabei zum Beispiel von nicht vollbesetzten Hüllen der, an den Gitterplätzen lokalisierten, Atomen eines Kristalls herrühren. In Anlehnung daran werden die magnetische Momente über $\mu_i = \mu \sigma_i$  mit Spin-Variablen $\sigma_i \in \{-1, 1\}$ in Verbindung gebracht, welche als klassische Zufalls-Variablen mit zwei möglichen Einstellungen betrachtet werden. Das als Ising-Modell bekannte System wird dann durch die in \eqref{H_ising_general} angegebene Hamiltonfunktion beschrieben. Betrachtet man ein Gitter mit $N$ Gitterpunkten, so gibt es insgesamt $2^N$ verschieden Konfigurationen welche die Gesamtheit aller Spin-Variablen einnehmen kann. Als klassische Beschreibung, eines eigentlich quantenmechanischen Systems, ist die Hamiltonfunktion diskret und kann höchstens $2^N$ verschiedene Werte annehmen. 

\begin{equation} \label{H_ising_general}
H_{Ising}(\sigma_1, \dots, \sigma_N) = - \sum_{(i,j)} J_{i,j} \,\sigma_i \sigma_j \;- \mu H_0 \sum_{i} \sigma_i 
\end{equation}

\noindent Die Größen $J_{i,j}$ werden als Austausch-Integrale bezeichnet. Sie beschreiben den Energieanteil, der aus der gegenseitigen Wechselwirkungen des i-ten und j-ten magnetischen Moments resultiert. Zu dieser Austauschenergie tragen im allgemeinen primär quantenmechanische Effekte bei, welche in dieser Arbeit nicht genauer erläutert werden sollen. Ein vernachlässigbarer kleiner Anteil resultiert aber zum Beispiel auch aus der gegenseitigen magnetischen Wechselwirkung. Die erste Summe in \eqref{H_ising_general} beschreibt somit die mit der Spin-Spin-Wechselwirkung in Verbindung gebrachten Energieanteil. Ist $J_{i,j}$ positiv, so führt eine parallele Spinausrichtung an den Gitterpunkten $i$ und $j$ zu einer Absenkung der Gesamtenergie und eine antiparallele Spinausrichtung zu einer Erhöhung. Ist $J_{i,j}$ negativ, so verhält es sich genau umgekehrt. Es gilt weiteres $\forall i : J_{i,i} = 0$. \\
\noindent Die zweite Summe in \eqref{H_ising_general} beschreibt den Anteil der Energie, welcher durch Anlegen eines externen Magnetfeldes an das Modell-System entsteht.
Für eine detaillierte Herleitung des quantenmechanischen Ising Hamiltonopertors und eine ausführlichere Beschreibung siehe auch \cite{MarxGross2014}.\\

\noindent Für den Rest der Arbeit sollen weitere Vereinfachungen vereinbart werden:
\begin{itemize}
\item[i)] Nur nächste Nachbar Wechselwirkung, d.h. $\forall i : J_{i,j} = 0$ für $\vert i-j \vert > 1$
\item[ii)] Isotropie und Homogenität des Modells, d.h. $ \forall i\neq j : J_{i,j} = J_{j,i} = J_i = J $
\item[iii)] Kein externes Magnetfeld, d.h. $H_0 = 0$
\end{itemize}

\noindent Zudem beschränkt sich die Arbeit auf ein zweidimensionales Gitter, welches o.B.d.A in der x-y-Ebene liegen soll. Die Einstellungen der Spins kann man sich dann als Ausrichtung der magnetischen Momente entlang der z-Achse vorstellen. Zudem soll $J > 0$ gelten, um überhaupt eine spontane Magnetisierung erwarten zu können. Ein solches Ising-System, bei dem alle Austausch-Integrale nicht negativ sind, bezeichnet man auch als ferromagnetisch. Die Hamiltonfunktion, welche das System für ein quadratisches Gitter mit Seitenlänge $2M+1$ und somit $N = (2M+1)^2$ Gitterpunkten beschreibt, ist in \eqref{H_ising_2d} angegeben.

\begin{grayframe}[frametitle = {2d Ising-Modell ohne externes Magnetfeld}]
\begin{align} 
H(\sigma_1, \dots, \sigma_N) &= - J  \sum_{(i,j)} \sigma_i \sigma_j \label{H_ising_2d}\\
  &= - J \sum_{x = -M}^M \sum_{y = -M}^M \sigma_{x, y} \sigma_{x+1, y} + \sigma_{x, y} \sigma_{x,y+1} \label{H_ising_2d_exp} 
\end{align}
\end{grayframe}

\noindent Die Summation in \eqref{H_ising_2d} läuft dabei über alle $2N$ Paare nächster Nachbarn auf dem Gitter. Diese Notation soll für die gesamte Arbeit vereinbart werden.

\begin{equation}
\sum_{(i,j)} \;\text{bzw.}\; \prod_{(i,j)} \iff  \begin{array}{ll} \text{Summe bzw. Produkt über alle Paare} \\ \text{nächster Nachbarn auf dem Gitter} \end{array}
\end{equation}

\noindent Im folgenden sollen die Beschreibung des Gitters über eine geeignete Indizierung, wie in \eqref{H_ising_2d}, gegenüber einer expliziten Angabe der Koordinaten der Gitterpunkte in der Ebene, wie in \eqref{H_ising_2d_exp}, bevorzugt werden. Die Indizierung wird dabei so gewählt, dass sie auch für eine unendliche Menge gültig wäre. Die Darstellung in \eqref{H_ising_2d_exp} wirft die Frage nach den Randbedingungen auf. Es werden periodische Randbedingungen gemäß \eqref{eq: Periodische Randbedingungen} festgelegt. 

\begin{equation} \label{eq: Periodische Randbedingungen}
\begin{aligned} 
\forall y & : \sigma_{M+1, y} =  \sigma_{-M, y} \\
\forall x & : \sigma_{x, M+1} =  \sigma_{x, -M}
\end{aligned}
\end{equation}

\begin{figure}[h]
    \centering
    \begin{tikzpicture}[scale = 1.5]
\begin{scope}
    \node[draw = none] at (0,0)   (p0p0) {$\sigma$};
\node[draw = none] at (0,1)   (p0p1) {$\sigma$};
\node[draw = none] at (1,0)   (p1p0) {$\sigma$};
\node[draw = none] at (0,-1)  (p0m1) {$\sigma$};
\node[draw = none] at (-1,0)  (m1p0) {$\sigma$};
\node[draw = none] at (1,-1)  (p1m1) {$\sigma$};
\node[draw = none] at (-1,1)  (m1p1) {$\sigma$};
\node[draw = none] at (-1,-1) (m1m1) {$\sigma$};
\node[draw = none] at (1,1)   (p1p1) {$\sigma$};

\node[draw = none] at (1.7,0)   (p2p0) {};
\node[draw = none] at (1.7,1)   (p2p1) {};
\node[draw = none] at (1,1.7)   (p1p2) {};
\node[draw = none] at (0,1.7)   (p0p2) {};
\node[draw = none] at (-1,1.7)  (m1p2) {};
\node[draw = none] at (-1.7,1)  (m2p1) {};
\node[draw = none] at (-1.7,0)  (m2p0) {};
\node[draw = none] at (-1.7,-1) (m2m1) {};
\node[draw = none] at (-1,-1.7) (m1m2) {};
\node[draw = none] at (0,-1.7)  (p0m2) {};
\node[draw = none] at (1,-1.7)  (p1m2) {};
\node[draw = none] at (1.7,-1)  (p2m1) {};

%% arrows
\draw[arrow_grid_in] (p0p0) -- (p0p1);
\draw[arrow_grid_in] (p0p0) -- (p1p0);

\draw[arrow_grid_in] (p1p0) -- (p1p1);


\draw[arrow_grid_in] (p0p1) -- (p1p1);
\draw[arrow_grid_out] (p0p1) -- (p0p2);

\draw[arrow_grid_out] (p1p1) -- (p1p2);


\draw[arrow_grid_in] (p1m1) -- (p1p0);


\draw[arrow_grid_in] (p0m1) -- (p0p0);
\draw[arrow_grid_in] (p0m1) -- (p1m1);

\draw[arrow_grid_in] (m1m1) -- (m1p0);
\draw[arrow_grid_in] (m1m1) -- (p0m1);

\draw[arrow_grid_in] (m1p0) -- (m1p1);
\draw[arrow_grid_in] (m1p0) -- (p0p0);

\draw[arrow_grid_in] (m1p1) -- (p0p1);
\draw[arrow_grid_out] (m1p1) -- (m1p2);



\draw[arrow_grid_out] (m1m2) -- (m1m1);
\draw[arrow_grid_out] (p0m2) -- (p0m1);
\draw[arrow_grid_out] (p1m2) -- (p1m1);


    \draw[arrow_grid_out] (m2p1) -- (m1p1);
    \draw[arrow_grid_out] (m2p0) -- (m1p0);
    \draw[arrow_grid_out] (m2m1) -- (m1m1);

\end{scope}
\begin{scope}[shift = {(2,0)}]
    \node[draw = none] at (0,0)   (p0p0) {$\sigma$};
\node[draw = none] at (0,1)   (p0p1) {$\sigma$};
\node[draw = none] at (1,0)   (p1p0) {$\sigma$};
\node[draw = none] at (0,-1)  (p0m1) {$\sigma$};
\node[draw = none] at (-1,0)  (m1p0) {$\sigma$};
\node[draw = none] at (1,-1)  (p1m1) {$\sigma$};
\node[draw = none] at (-1,1)  (m1p1) {$\sigma$};
\node[draw = none] at (-1,-1) (m1m1) {$\sigma$};
\node[draw = none] at (1,1)   (p1p1) {$\sigma$};

\node[draw = none] at (1.7,0)   (p2p0) {};
\node[draw = none] at (1.7,1)   (p2p1) {};
\node[draw = none] at (1,1.7)   (p1p2) {};
\node[draw = none] at (0,1.7)   (p0p2) {};
\node[draw = none] at (-1,1.7)  (m1p2) {};
\node[draw = none] at (-1.7,1)  (m2p1) {};
\node[draw = none] at (-1.7,0)  (m2p0) {};
\node[draw = none] at (-1.7,-1) (m2m1) {};
\node[draw = none] at (-1,-1.7) (m1m2) {};
\node[draw = none] at (0,-1.7)  (p0m2) {};
\node[draw = none] at (1,-1.7)  (p1m2) {};
\node[draw = none] at (1.7,-1)  (p2m1) {};

%% arrows
\draw[arrow_grid_in] (p0p0) -- (p0p1);
\draw[arrow_grid_in] (p0p0) -- (p1p0);

\draw[arrow_grid_in] (p1p0) -- (p1p1);


\draw[arrow_grid_in] (p0p1) -- (p1p1);
\draw[arrow_grid_out] (p0p1) -- (p0p2);

\draw[arrow_grid_out] (p1p1) -- (p1p2);


\draw[arrow_grid_in] (p1m1) -- (p1p0);


\draw[arrow_grid_in] (p0m1) -- (p0p0);
\draw[arrow_grid_in] (p0m1) -- (p1m1);

\draw[arrow_grid_in] (m1m1) -- (m1p0);
\draw[arrow_grid_in] (m1m1) -- (p0m1);

\draw[arrow_grid_in] (m1p0) -- (m1p1);
\draw[arrow_grid_in] (m1p0) -- (p0p0);

\draw[arrow_grid_in] (m1p1) -- (p0p1);
\draw[arrow_grid_out] (m1p1) -- (m1p2);



\draw[arrow_grid_out] (m1m2) -- (m1m1);
\draw[arrow_grid_out] (p0m2) -- (p0m1);
\draw[arrow_grid_out] (p1m2) -- (p1m1);


    \node[draw = none, scale = 1] at (-0.34,0.5)   (J) {$J_{i,j}$};
    \node[draw = none, scale = 1] at ( 0.5,-0.28)   (J) {$J_{i,k}$};
    \node[draw = none, scale = 0.8] at (0.1, -0.1)   (J) {$i$};
    \node[draw = none, scale = 0.8] at (1.1, -0.1)   (J) {$k$};
    \node[draw = none, scale = 0.8] at (0.1,  0.9)   (J) {$j$};
    %\node[draw = none, scale = 1] at (-0.25, -0.2)   (J) {$\sigma_i$};
\end{scope}
\begin{scope}[shift = {(4,0)}]
    \node[draw = none] at (0,0)   (p0p0) {$\sigma$};
\node[draw = none] at (0,1)   (p0p1) {$\sigma$};
\node[draw = none] at (1,0)   (p1p0) {$\sigma$};
\node[draw = none] at (0,-1)  (p0m1) {$\sigma$};
\node[draw = none] at (-1,0)  (m1p0) {$\sigma$};
\node[draw = none] at (1,-1)  (p1m1) {$\sigma$};
\node[draw = none] at (-1,1)  (m1p1) {$\sigma$};
\node[draw = none] at (-1,-1) (m1m1) {$\sigma$};
\node[draw = none] at (1,1)   (p1p1) {$\sigma$};

\node[draw = none] at (1.7,0)   (p2p0) {};
\node[draw = none] at (1.7,1)   (p2p1) {};
\node[draw = none] at (1,1.7)   (p1p2) {};
\node[draw = none] at (0,1.7)   (p0p2) {};
\node[draw = none] at (-1,1.7)  (m1p2) {};
\node[draw = none] at (-1.7,1)  (m2p1) {};
\node[draw = none] at (-1.7,0)  (m2p0) {};
\node[draw = none] at (-1.7,-1) (m2m1) {};
\node[draw = none] at (-1,-1.7) (m1m2) {};
\node[draw = none] at (0,-1.7)  (p0m2) {};
\node[draw = none] at (1,-1.7)  (p1m2) {};
\node[draw = none] at (1.7,-1)  (p2m1) {};

%% arrows
\draw[arrow_grid_in] (p0p0) -- (p0p1);
\draw[arrow_grid_in] (p0p0) -- (p1p0);

\draw[arrow_grid_in] (p1p0) -- (p1p1);


\draw[arrow_grid_in] (p0p1) -- (p1p1);
\draw[arrow_grid_out] (p0p1) -- (p0p2);

\draw[arrow_grid_out] (p1p1) -- (p1p2);


\draw[arrow_grid_in] (p1m1) -- (p1p0);


\draw[arrow_grid_in] (p0m1) -- (p0p0);
\draw[arrow_grid_in] (p0m1) -- (p1m1);

\draw[arrow_grid_in] (m1m1) -- (m1p0);
\draw[arrow_grid_in] (m1m1) -- (p0m1);

\draw[arrow_grid_in] (m1p0) -- (m1p1);
\draw[arrow_grid_in] (m1p0) -- (p0p0);

\draw[arrow_grid_in] (m1p1) -- (p0p1);
\draw[arrow_grid_out] (m1p1) -- (m1p2);



\draw[arrow_grid_out] (m1m2) -- (m1m1);
\draw[arrow_grid_out] (p0m2) -- (p0m1);
\draw[arrow_grid_out] (p1m2) -- (p1m1);


    \draw[arrow_grid_out] (p1p0) -- (p2p0);
    \draw[arrow_grid_out] (p1p1) -- (p2p1);
    \draw[arrow_grid_out] (p1m1) -- (p2m1);
\end{scope}

\end{tikzpicture}
    \caption{Darstellung der Paarweisen Austauschwechselwirkungen $J_{i,j}$. Unter der Berücksichtigung periodischer Randbedingungen (Hellgraue Pfeile) gibt es genau zwei Austasuchintegrale pro Gitterpunkt.}  \label{Abb: grid}
\end{figure}

\subsection{Elemente der Statistischen Mechanik}

Das im vorherigen Abschnitt beschriebene Ising-Gitter dient nun als statistisches Assemble von N Teilchen, beschrieben durch die Hamiltonfunktion \eqref{H_ising_2d}. Wir gehen davon aus, dass das System in thermischen Kontakt mit einem Wärmebad konstanter Temperatur $T$ steht. Zudem gilt dass sowohl die Ausdehnung $V$, als auch die an Anzahl N der an den Gitterplätzen lokalisierten Momente, konstant sind für jedes Ising-Gitter. Das Assemble kann aus Sicht der statistischen Physik somit als sogenannte kanonische Gesamtheit beschrieben werden. Die Berechnung vieler Größen solcher Systeme können durch die Berechnung der Zustandssumme $Z$ und Erwartungswerte bzw. Korrelationen $\corr{.}$ berechnet werden.

\begin{equation} \label{def: sp_zustandssumme}
 Z(T,N,V) = \sum_{\{S\}} e^{-\beta H( S ) } 
\end{equation}

\noindent Die Zustandssumme ist dabei gemäß \eqref{def: sp_zustandssumme} definiert, wobei der Faktor $\beta = \beta\left(T\right) = \frac{1}{k_B T} $ eingeführt wurde, welcher die Einheit einer reziproken Energie hat. $k_B$ ist hierbei die Boltzmann-Konstante. Die Summation erfolgt über die Menge alle möglichen Spinkonfigurationen des Systems, geschrieben als $\{S\} = \{S = (\sigma_1, \sigma_2, \dots,\sigma_N) \,\vert\, \forall\,i : \sigma_i \in \{-1, 1\}\}$. Die Wahrscheinlichkeit dass eine Konfiguration $S$ eingenommen wird ist dann wie in \eqref{eq: konfigProb} gegeben.

\begin{equation} \label{eq: konfigProb}
    P(S) = \frac{1}{Z} e^{-\beta H( S ) } 
\end{equation}

\noindent Der Spin-Spin-Korrelation ist der Erwartungswert für das Produkt zweier Spinvariablen und ist wie in \eqref{def: sp_erwartungswert} definiert.

\begin{equation} \label{def: sp_erwartungswert}
    \corr{\sigma_{p} \sigma_{q}}_{T,N,V} = \frac{1}{Z} \sum_{\{S\}} e^{-\beta H( S ) }  \sigma_{p}( S ) \sigma_{q}( S )
\end{equation}

 \noindent Es soll nun ein expliziter Ausdruck für die Zustandssumme und die Spin-Spin-Korrelation des Ising-Modells abgeleitet werden. Dazu wird mit der Berechnung des Ausdrucks $\exp{\beta J \sigma_i \, \sigma_j} $ für $i \neq j$ begonnen.

\begin{align}
\exp{\beta J \sigma_i \sigma_j} & = \sum_{k=0}^{\infty} \frac{(\beta J)^{2k}}{(2k)!}(\sigma_i \sigma_j)^{2k} + \sum_{k=0}^{\infty} \frac{(\beta J)^{2k+1}}{(2k+1)!}(\sigma_i \sigma_j)^{2k+1}  &\\
& = \sum_{k=0}^{\infty} \frac{(\beta J)^{2k}}{(2k)!} + (\sigma_i \sigma_j) \sum_{k=0}^{\infty} \frac{(\beta J)^{2k+1}}{(2k+1)!} &\\
& = cosh(\beta J) + (\sigma_i \sigma_j) sinh(\beta J) &\\
& = cosh(\beta J) \; (1 +  tanh(\beta J) \; (\sigma_i \sigma_j)) \label{eq: exp(beta J sig sig)}
\end{align}

\noindent Dabei wurde benutzt dass $(\sigma_i \sigma_j)^{2k} = 1$ und $(\sigma_i \sigma_j)^{2k+1} = (\sigma_i \sigma_j)$ gilt. Durch Einsetzten der Definition der Hamiltonfunktion lässt sich ein expliziter Ausdruck für die Zustandssumme des Ising-Modells finden. 

\begin{align} 
    Z(T,N,V) 
    &  = \sum_{\{S\}} \exp{\beta J \sum_{(i,j)} \sigma_i \sigma_j } \\
    &  = \sum_{\{S\}} \prod_{(i,j)} \exp{\beta J \sigma_i \sigma_j } \\
    &  = cosh(\beta J)^{2N} \sum_{\{S\}} \prod_{(i,j)} (1 +  tanh(\beta J) \; (\sigma_i \sigma_j)) \\
    &  = cosh(\beta J)^{2N} \sum_{\{S\}} \prod_{(i,j)} (1 +  t \; (\sigma_i \sigma_j))
\end{align}

\noindent Bei dieser Ableitung wurde die Größe $t = tanh(\beta J)$ eingeführt. Die Potenz $2N$ rührt von der Anzahl der Paare nächster Nachbarn auf dem Gitter her. Siehe Abbildung \ref{Abb: grid}. Analog lässt sich ein expliziter Ausdruck für die Korrelation für beliebe Paare von Spin-Operatoren des Ising-Modells berechnen.

\begin{align} 
    \corr{\hat{\sigma}_{p} \hat{\sigma}_{q} }_{T,N,V}  
    & = \frac{1}{Z} \sum_{\{S\}} \exp{\beta J \sum_{(i,j)} \sigma_i \sigma_j } \sigma_{p} \sigma_{q} \\
    & = \frac{1}{Z} cosh(\beta J)^{2N} \sum_{\{S\}} \left(\prod_{(i,j)} (1 +  t \; (\sigma_i \sigma_j) )\right) \sigma_{q} \sigma_{q} 
\end{align}

\noindent Die hier abgeleiteten Ausdrücke, zusammengefasst in  \eqref{Ising_Zustandssumme} und \eqref{Ising_SpinSpinCorrelation}, werden in Abschnitt \ref{Sec: Exakte Berechnung der Zustandssumme} und \ref{Sec: Exakte Berechnung der (horizontalen) Spin-Spin Korrelation} als Ausgangspunkt für die weiter Berechnungen mithilfe antikommutativer Variablen dienen.

\begin{grayframe}[frametitle = {Zustandssumme und Spin-Spin-Korrelation für 2d Ising-Modell}]
\begin{align}
 Z(T,N,V)  
  & = cosh(\beta J)^{2N} \sum_{\{S\}} \prod_{(i,j)} (1 +  t \; (\sigma_i \sigma_j)) \label{Ising_Zustandssumme} \\
\corr{\hat{\sigma}_{p} \hat{\sigma}_{q} }_{T,N,V} 
  & = \frac{1}{Z} cosh(\beta J)^{2N} \sum_{\{S\}} \left(\prod_{(i,j)} (1 +  t \; (\sigma_i \sigma_j) )\right) \sigma_{q} \sigma_{q} \label{Ising_SpinSpinCorrelation}
\end{align}
\centering
\noindent Die Summation läuft dabei über alle $2^N$ Spinkonfigurationen des Gitters
$$\{S\} = \{S = (\sigma_1, \sigma_2, \dots,\sigma_N) \,\vert\, \forall\,i : \sigma_i \in \{-1, 1\}\}$$
\end{grayframe}

\noindent Um das System mit den Mittel der Statistischen Physik auf Phasenübergänge untersuchen zu können, muss das Ising-Modell im sogenannten Thermodynamik Limes untersucht werden. Dazu wird eine entsprechende Größe, wie die freie Energie, für ein System endlicher Ausdehnung $V$ und Teilchenzahl $N$ berechnet und anschließend mithilfe des Grenzübergangs \eqref{Thermodynamischer Limes} auf ein unendliches System extrapoliert. 

\begin{equation} \label{Thermodynamischer Limes}
\centering
\begin{tabular}{lccccc}
        &  &                 &    &      $N \longrightarrow \infty$\\
$\tlim$ &  &  $\iff$           &   &      $V \longrightarrow \infty$ \\
        &  &                 &   & $n =  N/V = konst. < \infty$
\end{tabular}
\end{equation}

\noindent Der Übergang in den Thermodynamischen Limes ist notwendig, um  Phasenübergänge erkennen zu können, da diese sich als Diskontinuität, Sigularität oder Nicht-Analytizität eines Thermodynamischen Potenzials, wie der Freien Energie, bemerkbar machen. Für endliche Systeme sind diese Potenziale aber immer analytisch. Für das unendlich große System können nur Größen pro Volumen oder Teilchen sinnvoll betrachtet werden. Die Freie Energie pro Teilchen $f$ lässt sich z.B. als Thermodynamischer Limes der Freien Energien $F_N$ endlicher Systemen mit Teilchenzahl $N$ ausdrücken. $F_N$ wiederum kann mit der Zustandsumme des endlichen Systems berechnet werden. 

\begin{equation} \label{SP_FreieEnergie}
f(T,n = \frac{N}{V}) = \tlim \frac{F_N}{N} = \tlim \frac{-k_B T}{N} \ln Z(T,V,N) 
\end{equation}

\subsection{Spontane Magnetisierung}

Abschließend soll noch die Definition des Begriffs der Spontane Magnetisierung besprochen werden. Die Magnetisierung $M$ eines magnetischen Festkörpers wird normalerweise als das mittlere magnetische Moment pro Volumen definiert. Für ein endliches Ising-Gitter $\Lambda_N$ ergibt sich Ausdruck  \eqref{Classic_Magnatization} als eine Anschauliche Definition der Magnetisierung des Ising-Systems. 

\begin{equation} \label{Classic_Magnatization}
M(N,T,V,H_0)  = \frac{1}{V}\corr{ \mu \sum_i \sigma_i\;}_{\Lambda_N, H_0} = \frac{\mu N}{V}\corr{\sigma_0}_{\Lambda_N, H_0} = n \mu \corr{\sigma_0}_{\Lambda_N, H_0} 
\end{equation}

\noindent Dabei wurde die, durch die angenommene Homogenität des periodischen Gitters implizierte, translationsinvarianz der Erwartungswerte benutzt. Die Magnetisierung ergibt sich letztlich durch Extrapolation mithilfe des Thermodynamischen Limes. Zur Notation: $\Lambda_N$ gibt an dass ein endliches System mit $N$ Teilchen betrachtet wird und $H_0$ verweist auf die Anwesenheit eines externen Magnetfeldes $H_0 > 0$. Als Spontane Magnetisierung $M_S$ wird nun die Magnetisierung eines Festkörpers bei verschwindendem externen Magnetfeld bezeichnet. 

\begin{equation} \label{def: classic_spontaneMagnetisierung}
M_S(T) = \lim_{H_0 \rightarrow 0} M(T, H_0) = \lim_{H_0 \rightarrow 0} \tlim M(T,N,V,H_0) =  n \mu \lim_{H_0 \rightarrow 0} \corr{\sigma_0}_{H_0} 
\end{equation}

\noindent Die Definition in \eqref{def: classic_spontaneMagnetisierung} wirft dabei jedoch zwei Probleme für die Berechnung der Spontanen  Magnetisierung für das hier untersuchte Ising-System \eqref{H_ising_2d} auf:

\begin{itemize}
\item [i)] Die Grenzprozesse in \eqref{def: classic_spontaneMagnetisierung} lassen sich im Allgemeinen nicht vertauschen. Somit kann diese Definition für die Spontane Magnetisierung möglicherweise nicht angewandt werden. Für das untersuchte System wurde schließlich $H_0 = 0$ festgelegt. 
\item [ii)] Es konnte bislang keine analytische Berechnung für $\corr{\sigma_0}$ gefunden werden.  (Zumindest ist eine solche dem Autor nicht bekannt). 
\end{itemize}

\noindent Um diese Probleme zu umgehen, soll die Spontane Magnetisierung stattdessen über die langreichweitige Spin-Spin-Korrelation \eqref{longrangeCorrelation} gemäß \eqref{spontaneMagnetisierung} definiert werden. 

\begin{grayframe}[frametitle = {Spontane Magnetisierung}]
\begin{align}
l^* & = \lim_{||i-j||_1 \rightarrow \infty} \tlim \sqrt{\corr{\sigma_i\sigma_j}_{\Lambda_N}} \label{longrangeCorrelation}\\ 
M_S   & = n \mu l^* \label{spontaneMagnetisierung}
\end{align}
\end{grayframe}

\noindent Um einsehen zu können dass dies eine äquivalente Definition ist, soll zuerst gezeigt dass die langreichweitige Spin-Spin-Korrelation in Anwesenheit eines beliebig kleinen externen Magnetfeldes in das Produkt der Erwartungswerte der einzelnen Spinvariablen faktorisiert. Dazu nehmen wir an dass die magnetische Suzeptiliät $\chi$ für ein Ising-System mit externen Magnetfeld \eqref{def: suszeptibilitaet} einen endlichen Wert besitzt. Dies ist eine gerechtfertigte Annahme für alle Temperaturen die außerhalb einer beliebig kleinen Umgebung um einen Kritische Temperatur liegen, an der ein Phasenübergang stattfindet.

\begin{equation} \label{def: suszeptibilitaet}
\chi = \tlim\; \frac{\partial M}{\partial H_0} = n \mu \tlim\; \frac{\partial \corr{\sigma_0}_{\Lambda_N, H_0}}{\partial H_0} 
\end{equation}

\noindent Der folgende Beweis benutzt mehrmals die sogenannten Griffiths-Kelley-Sherman-Ungleichungen (GKS-Ugl.), welche für allgemeine klassiche Spinsysteme $H_{\mathcal{S}}$ der Form \eqref{Def: generalSpinSystem} auf einem Gitter $\Lambda$ gilt. Eine Formulierung dieser ist in \eqref{eq: GKS1} und \eqref{eq: GKS2} ohne Beweis angegeben. Für einen Beweis siehe zum Beispiel.

\begin{grayframe}[frametitle = {Griffiths-Kelley-Sherman-Ungelichungen}]
\begin{equation}\label{Def: generalSpinSystem} 
H_{\mathcal{S}}  = \sum_{A \subseteq \Lambda} J_A \sigma^A \;\;\;\text{wobei}\;\;\; \sigma^A = \prod_{i \in A} \sigma_i
\end{equation}
\begin{align}
&\forall\; A,B \subseteq \Lambda \;\text{gilt : }  \\
&\text{1. Ungleichung:}\;   \corr{\sigma^A}    \geq 0  \label{eq: GKS1}\\
&\text{2. Ungleichung:}\;   \corr{\sigma^A \sigma^B}\geq \corr{\sigma^A}\corr{\sigma^B} \label{eq: GKS2}\\
\end{align}
\end{grayframe}

\noindent Um die Rechnungen möglichst kurz zu halten, sollen einige Folgerungen aus diesen GKS Ungleichung für algemeineren Spinsysteme, wie in \eqref{Def: generalSpinSystem},abgeleitet werden. Dazu wird die Änderung der Erwartungswerte in Abhängigkeit der verallgemeinerten Austauschintegrale untersucht.

\begin{equation}
\frac{\partial \corr{\sigma^A}}{\partial J_B} = \frac{\partial }{\partial J_B} \frac{1}{Z} \sum_{\{S\}} \exp{\sum_{C \subseteq \Lambda} J_C\, \sigma^C } \sigma^A = \beta \left(\corr{\sigma^A \sigma^B} - \corr{\sigma^A }\corr{\sigma^B} \right) \geq 0 \label{eq: GKS3}
\end{equation}

\noindent Dabei muss beachtet werden dass auch $Z$ von den Größen $J_C$ abhängt. Die Ungleichung \eqref{eq: GKS3} folgt aus der 2.ten GKS-Ungleichung \eqref{eq: GKS2}. Die Ungleichung bringt für $A=\{i,j \}$ für $i,j \in \Lambda$ eine intuitive Eigenschaft des Ising-Modells zum Ausdruck, nämlich dass sich der Erwartungswert der Spin-Spin-Korrelation erhöht, wenn man die Austauschwechselwirkung vergrößert. Für $B =  \{ i \} $ mit $i \in \Lambda$ und $ J_B = H_0 > 0$ folgt \eqref{eq:  suszeptibilitaet} für die magnetische Suzeptibilität des Ising-Modells in \eqref{H_ising_general} .  

\begin{align} 
\frac{\chi}{n \mu}
&=  \tlim\; \frac{\partial \corr{\sigma_0}_{\Lambda_N, H_0}}{\partial H_0}  
 = \tlim\; \sum_{i \in \Lambda} \frac{\partial \corr{\sigma_0}_{\Lambda_N, H_0}}{\partial J_i} \\
&= \beta \sum_{i \in \mathbb{Z}^2} \corr{\sigma_0 \, \sigma_i}_{\Lambda_N, H_0} - \corr{\sigma_0}_{\Lambda_N, H_0}\corr{\sigma_i}_{\Lambda_N, H_0} \;\;<\;\; \infty \label{eq:  suszeptibilitaet}
\end{align}

\noindent Da alle Summanden positiv sind, müssen diese für große $i$ verschwindend klein werden. Daher muss die langreichweitige Spin-Spin-Korreltation in ein Produkt von Erwartungswerten faktorisieren, falls $H_0>0$ gilt.

\begin{equation} \label{eq: corrFactorize}
n \mu \tlim \corr{\sigma_0 \,\sigma_j}_{\Lambda_N, H_0} \;\; \underset{||j||_1 \rightarrow \infty}{\longrightarrow} \;\;n \mu \tlim \corr{\sigma_0}_{\Lambda_N, H_0}\langle\sigma_j\rangle_{\Lambda_N, H_0} =  n \mu \corr{\sigma_0}_{H_0}^2 % = M
\end{equation}

\noindent Somit lässt sich für $H_0 > 0$ die Magnetisierung mithilfe der langreichweitige Spin-Spin-Korrelation ausdrücken. Für $H=0$ kann noch immer keine Aussage getroffen werden, da nachwievor die Grenzprozesse in \eqref{def: classic_spontaneMagnetisierung} im allgemeinen nicht vertauscht werden dürfen. Aus \eqref{eq: GKS3} folgt aber für das hier untersuchte Ising-System, dass die Erwartungswerte und Spin-Spin-Korrelationen monoton wachsend in den Austauschintergralen $J_{i,j}$ und dem Betrag des Externen Magnetfeldes $H_0$ sind. Wenn ein Austauschintegral unendlich groß gemacht wird, so nehmen die zugehörigen Spin-Variablen immer den gleichen Wert an, da die Wahrscheinlichkeit dass sie unterschiedliche Werte annehmen nach \eqref{eq: konfigProb} verschwindet. Unter Berücksichtigung der Periodischen Randbedingungen, verhält sich somit ein Gitter, bei dem entsprechende Autauschintegrale auf einen unendlich hohen Wert angehoben werden, wie ein kleineres Gitter. Diese Monotonie-Eigenschaft wird in \eqref{eq: ErwartungswertMonotnie} festgehalten und wird in Abbildung \ref{Abb: monotnie} graphisch veranschaulicht.

\begin{equation} \label{eq: ErwartungswertMonotnie}
    \corr{\sigma^A}_{\Lambda'} \geq \corr{\sigma^A}_{\Lambda''} \;\;\; \text{für} \;\;\; \Lambda' \subseteq \Lambda'' \subseteq \Lambda
\end{equation}

\noindent Aus \eqref{eq: ErwartungswertMonotnie} folgt, dass die Erwartungswerte in folge des Thermodynamischen Limes mit wachsendem Gitter monoton fallen. Wegen der ersten GKS-Ungleichung \eqref{eq: GKS1} sind die Folgen 

\begin{equation}\left(\corr{\sigma^A}_{\Lambda_N,H_0}\right)_{N\in\mathbb N} \;\;\;\text{und}\;\;\;  \left( \corr{\sigma^A}_{\Lambda_N}\right)_{N\in\mathbb N} \end{equation} zudem beschränkt und die Grenzwerte 

\begin{equation} \corr{\sigma^A} = \tlim \corr{\sigma^A}_{\Lambda_N} \;\;\;\text{und}\;\;\; \corr{\sigma^A}_{H_0} = \tlim \corr{\sigma^A}_{\Lambda_N,H_0} \label{eq: limesExist}
\end{equation} existieren daher. Dabei gilt wegen der Monotonie in $N$

\begin{equation} 
\forall N\in\mathbb{N}: \corr{\sigma^A}_{H_0} \leq  \corr{\sigma^A}_{\Lambda_N,H_0}
\end{equation} Somit folgt 

\begin{equation} 
\lim_{H_0 \rightarrow 0} \corr{\sigma^A}_{H_0} = \tlim \lim_{H_0 \rightarrow 0}  \corr{\sigma^A}_{H_0} \leq \tlim \lim_{H_0 \rightarrow 0}  \corr{\sigma^A}_{\Lambda_N,H_0} = \corr{\sigma^A}
\end{equation} Aufgrund der Monotonie in H folgt jedoch ebenso 

\begin{equation} 
\lim_{H_0 \rightarrow 0} \corr{\sigma^A}_{H_0} \geq \corr{\sigma^A}
\end{equation} Und somit folgt die Gleichheit 

\begin{equation} 
\lim_{H_0 \rightarrow 0} \tlim \corr{\sigma^A}_{H_0} = \lim_{H_0 \rightarrow 0} \corr{\sigma^A}_{H_0} =  \corr{\sigma^A} = \tlim \lim_{H_0 \rightarrow 0}  \corr{\sigma^A}_{H_0} \label{eq: limesSwap}
\end{equation} 

\noindent Da \eqref{eq: limesExist} und \eqref{eq: limesSwap} für beliebige Produkte von unterschiedlichen Ising-Spins gezeigt wurde, folgt gemeinsam mit \eqref{eq: corrFactorize} die Äquivalenz der Definitionen \eqref{def: classic_spontaneMagnetisierung} und \eqref{spontaneMagnetisierung} der Sponanten Magnetisierung. Die Eigenschaft das die Spin-Spin-Korrelation für $H_0 = 0$ ebenfalls faktorisiert folgt aus der Existenz der einzelne Grenzwerte und den Grenzwertsätzen für Folgen reeler Zahlen.  

\begin{figure}[h]
    \centering
    \begin{tikzpicture}%[node distance=0.15, scale = 2.2]

\begin{scope}[scale = 2]
\node[grid_point] at (1,0)             (p1) {};
\node[grid_point] at (0.707,0.707)   (p2) {};
\node[grid_point] at (0,1)             (p3) {};
\node[grid_point] at (-0.707,0.707)  (p4) {};
\node[grid_point] at (-1,0)            (p5) {};
\node[grid_point] at (-0.707,-0.707) (p6) {};
\node[grid_point] at (0,-1)            (p7) {};
\node[grid_point] at (0.707,-0.707)  (p8) {};

\node[draw = none] at (1.1548, 0.4784)   (c1) {$J_{0,1}$};
\node[draw = none] at (0.4784, 1.1548)   (c2) {$J_{1,2}$};
\node[draw = none, text width=5em] at (-0.4784, 1.1548)  (c3) {$J_{2,3}\uparrow \infty$};
\node[draw = none,, text width=6em] at (-1.1548, 0.4784)  (c4) {$J_{3,4}\uparrow \infty$};
\node[draw = none, text width=7em] at (-1.1548, -0.4784) (c5) {$J_{-3,4}\uparrow \infty$};
\node[draw = none, text width=6em] at (-0.4784, -1.1548) (c6) {$J_{-2,-3}\uparrow \infty$};
\node[draw = none] at (0.4784, -1.1548)  (c7) {$J_{-1,-2}$};
\node[draw = none] at (1.1548, -0.4784)  (c8) {$J_{0,-1} $};

\draw[-, dashed] (p3) -- (p6);
\node[draw = none, text width=6em] at (0.3, -0.3)  (J) {$\hat{J}_{-3,2} \uparrow \infty$};
\draw[color = black!80] (0,0) circle (1);

\node[draw = none, scale = 2] at (2,0) (arr) {$\rightsquigarrow$};
\end{scope}

\begin{scope}[scale = 2, shift={(4,0)}]
\node[grid_point] at (1,0)              (p1) {};
\node[grid_point] at ( 0.5000,  0.8660)  (p2) {};
\node[grid_point] at (-0.5000,  0.8660) (p3) {};
\node[grid_point] at (-1, 0)            (p4) {};
\node[grid_point] at (-0.5000, -0.8660) (p5) {};
\node[grid_point] at ( 0.5000, -0.8660) (p6) {};

\node[draw = none] at ( 1.0825,  0.6250)  (c8) {$J_{0,1} $};
\node[draw = none] at ( 1.0825, -0.6250)  (c8) {$J_{1,2} $};
\node[draw = none, text width=6em] at (-1.0825,  0.6250)  (c8) {$J_{-3,2} \uparrow \infty $};
\node[draw = none] at (-1.22, -0.6250)  (c8) {$J_{-2,-3} $};
\node[draw = none] at (0,   1.25)  (c8) {$J_{1,2} $};
\node[draw = none] at (0,  -1.25)  (c8) {$J_{-1,-2} $};

\node[draw = none] at (-0.4,  0.55)  (c8) {$2$};
\node[draw = none] at (-0.72, 0)  (c8) {$-3 $};

\draw[color = black!80] (0,0) circle (1);

\end{scope}
\end{tikzpicture}
    \caption{Veranschaulichung der Reduzierung auf ein kleineres Gitter durch unendliche Erhöhung von Austauschintegralen $J_{i,j}$ für dim = 1. Die Periodischen Randbedingungen werden ebenfalls über ein unendlich großes Austauschintegral dargestellt sodass, sich 2 und -3 gemeinsam wie ein Gitterpunkt verhalten.} \label{Abb: monotnie}
\end{figure}






