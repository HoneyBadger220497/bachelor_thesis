 \noindent Es soll nun ein expliziter Ausdruck für die Zustandssumme und die Spin-Spin-Korrelation des Ising-Modells abgeleitet werden. Dazu wird mit der Berechnung des Ausdrucks $\exp{\beta J \sigma_i \, \sigma_j} $ für $i \neq j$ begonnen.

\begin{align}
\exp{\beta J \sigma_i \sigma_j} & = \sum_{k=0}^{\infty} \frac{(\beta J)^{2k}}{(2k)!}(\sigma_i \sigma_j)^{2k} + \sum_{k=0}^{\infty} \frac{(\beta J)^{2k+1}}{(2k+1)!}(\sigma_i \sigma_j)^{2k+1}  &\\
& = \sum_{k=0}^{\infty} \frac{(\beta J)^{2k}}{(2k)!} + (\sigma_i \sigma_j) \sum_{k=0}^{\infty} \frac{(\beta J)^{2k+1}}{(2k+1)!} &\\
& = cosh(\beta J) + (\sigma_i \sigma_j) sinh(\beta J) &\\
& = cosh(\beta J) \; (1 +  tanh(\beta J) \; (\sigma_i \sigma_j)) \label{eq: exp(beta J sig sig)}
\end{align}

\noindent Dabei wurde benutzt dass $(\sigma_i \sigma_j)^{2k} = 1$ und $(\sigma_i \sigma_j)^{2k+1} = (\sigma_i \sigma_j)$ gilt. Durch Einsetzten der Definition der Hamiltonfunktion lässt sich ein expliziter Ausdruck für die Zustandssumme des Ising-Modells finden. 

\begin{align} 
    Z(T,N,V) 
    &  = \sum_{\{S\}} \exp{\beta J \sum_{(i,j)} \sigma_i \sigma_j } \\
    &  = \sum_{\{S\}} \prod_{(i,j)} \exp{\beta J \sigma_i \sigma_j } \\
    &  = cosh(\beta J)^{2N} \sum_{\{S\}} \prod_{(i,j)} (1 +  tanh(\beta J) \; (\sigma_i \sigma_j)) \\
    &  = cosh(\beta J)^{2N} \sum_{\{S\}} \prod_{(i,j)} (1 +  t \; (\sigma_i \sigma_j))
\end{align}

\noindent Bei dieser Ableitung wurde die Größe $t = tanh(\beta J)$ eingeführt. Die Potenz $2N$ rührt von der Anzahl der Paare nächster Nachbarn auf dem Gitter her. Siehe Abbildung \ref{Abb: grid}. Analog lässt sich ein expliziter Ausdruck für die Korrelation für beliebe Paare von Spin-Operatoren des Ising-Modells berechnen.

\begin{align} 
    \corr{\hat{\sigma}_{p} \hat{\sigma}_{q} }_{T,N,V}  
    & = \frac{1}{Z} \sum_{\{S\}} \exp{\beta J \sum_{(i,j)} \sigma_i \sigma_j } \sigma_{p} \sigma_{q} \\
    & = \frac{1}{Z} cosh(\beta J)^{2N} \sum_{\{S\}} \left(\prod_{(i,j)} (1 +  t \; (\sigma_i \sigma_j) )\right) \sigma_{q} \sigma_{q} 
\end{align}

\noindent Die hier abgeleiteten Ausdrücke, zusammengefasst in  \eqref{Ising_Zustandssumme} und \eqref{Ising_SpinSpinCorrelation}, werden in Abschnitt \ref{Sec: Exakte Berechnung der Zustandssumme} und \ref{Sec: Exakte Berechnung der (horizontalen) Spin-Spin Korrelation} als Ausgangspunkt für die weiter Berechnungen mithilfe antikommutativer Variablen dienen.

\begin{grayframe}[frametitle = {Zustandssumme und Spin-Spin-Korrelation für 2d Ising-Modell}]
\begin{align}
 Z(T,N,V)  
  & = cosh(\beta J)^{2N} \sum_{\{S\}} \prod_{(i,j)} (1 +  t \; (\sigma_i \sigma_j)) \label{Ising_Zustandssumme} \\
\corr{\hat{\sigma}_{p} \hat{\sigma}_{q} }_{T,N,V} 
  & = \frac{1}{Z} cosh(\beta J)^{2N} \sum_{\{S\}} \left(\prod_{(i,j)} (1 +  t \; (\sigma_i \sigma_j) )\right) \sigma_{q} \sigma_{q} \label{Ising_SpinSpinCorrelation}
\end{align}
\centering
\noindent Die Summation läuft dabei über alle $2^N$ Spinkonfigurationen des Gitters
$$\{S\} = \{S = (\sigma_1, \sigma_2, \dots,\sigma_N) \,\vert\, \forall\,i : \sigma_i \in \{-1, 1\}\}$$
\end{grayframe}


Mithilfe einer graphischen Überlegung soll eine alternative Darstellung der Zustandssumme als Berezin-Integral über Graßmann Variablen hergeleitet werden. Diese Darstellung erlaubt die Anwendung des mathematischen Apparats der fermionischen Quantenfeldtheorie, welche eine einfache Berechnung der betrachteten Größen ermöglicht. Diese Idee geht auf die Arbeiten von Stuart Samuel zurück. 

\subsection{Korrelation als Zustandssumme auf modifizierten Gitter} 
Zuerst soll eine Verbindung zwischen der Spin-Korrelation und der Zustandssumme des Ising-Modells hergestellt werden, um die allgemeine Form der Herleitung in Abschnitt \ref{sec: GR_Zustandssumme} rechtzufertigen.

\noindent Man betrachte 2 Punkte $p$, $q$ auf dem Gitter. Diese können immer durch eine rechteckige Strecke auf dem Gitter verbunden werden. Diese Strecke ist nicht eindeutig und soll mit $L$ bezeichnet werden. Weiteres soll $P_L = \{ (i,j) \,|\, i,j \in L\ \wedge |i-j| = 1 \}$ die Menge der Paare nächster Nachbarn in $L$ sein. Nutzt man die Eigenschaft  $\sigma_{i}^{2} = 1$ dann gilt

$$ \sigma_{p} \sigma_{q} = \prod_{\bm{x}_i \in L} \sigma_{\bm{x}_i}^{2} \,\sigma_{p}\,\sigma_{q} = \prod_{(i,j) \in P_L} \sigma_{i}\sigma_{j} $$

\noindent Somit kann die Summe in \eqref{Ising_SpinSpinCorrelation} umgeschrieben werden. 

\begin{align} 
 \sum_{\{S\}} \left(\prod_{(i,j)} (1 +  t \; (\sigma_i \sigma_j))\textbf{ }\sigma_{q} \sigma_{q}\right) 
  & = \sum_{\{S\}} \left(\prod_{(i,j) \notin P_L } ((1 +  t \; (\sigma_i \sigma_j))\right) \left(\prod_{(i,j) \in P_L} \sigma_i \sigma_j\right) \\
  & = \sum_{\{S\}} \left(\prod_{(i,j) \notin P_L} ((1 +  t \; (\sigma_i \sigma_j))\right) \left(\prod_{(i,j) \in P_L} ((1 +  t \; (\sigma_i \sigma_j))\sigma_i \sigma_j \right) \\
  & = \sum_{\{S\}} \left(\prod_{(i,j) \notin P_L} ((1 +  t \; (\sigma_i \sigma_j))\right) \left(\prod_{(i,j) \in P_L} (((\sigma_i \sigma_j) +  t)\right)  \\
  & = t^m \sum_{\{S\}} \left(\prod_{(i,j) \notin P_L} ((1 +  t \; (\sigma_i \sigma_j))\right) \left(\prod_{(i,j) \in P_L} (((1 + \frac{1}{t}\sigma_i \sigma_j))\right)\\
  &= t^m \sum_{\{S\}} \left(\prod_{(i,j)} (1 + \hat{t}_{(i,j)} \; (\sigma_i \sigma_j)) \right) \label{eq: corr_sum}
\end{align}

\noindent Hierbei ist $m$ die Anzahl der Gitterpunkte auf der Strecke $L$. Bis auf den Vorfaktor lässt sich in \eqref{eq: corr_sum} die Summe als Zustandssumme des Ising-Modells mit modifiziertem ortsabhängigem $t$ identifizieren. Für die Reformulierung des Ising-modells muss also nur ein Ausdruck für die Summe $\hat{Z}[\hat{t}_{i,j}]$, wie in \eqref{eq: zustandsfunktional} angegeben, gefunden werden.  

\begin{grayframe}[frametitle = {Spin Korrelation als Zustandsumme auf defektem Gitter}]
\begin{equation}
\corr{\hat{\sigma}_{p} \hat{\sigma}_{q} }_{T,N,V} = \frac{t^m \hat{Z}[\hat{t}_L]}{\hat{Z}[t]}
\end{equation}

\begin{align}
\hat{Z}[\hat{t}] &=\sum_{\{S\}} \left(\prod_{(i,j)} (1 + \hat{t}_{(i,j)} \; (\sigma_i \sigma_j)) \right) \label{eq: zustandsfunktional}\\
\hat{t}_{L,(i,j)} &= \left\{\begin{array}{ll} t^{-1} & \text{für}\; (i,j)\notin P_L \\
          t & sonst \end{array} \right.
\end{align}

\end{grayframe}

\subsection{Graphische Repräsentation der Zustandssumme} \label{sec: GR_Zustandssumme}

Als nächstes soll eine Darstellung des Summe $\hat{Z}[\hat{t}_{i,j}]$ gefunden werden, welche eine Graphische Interpretation erlaubt. Bei der Auswertung des Produkts in \eqref{sec: GR_Zustandssumme} erhält man eine Summe, bei der die Summand aus der Entscheidung für $1$ oder $\hat{t}_{i,j}\,\sigma_i \sigma_j$ in jeden Multiplikanten herrührt. Es handelt sich also um eine Summation über alle Produkte von Isingspin Paaren unterschiedlicher Länge $n \in {0,1,\dots ,2N}$. 

\begin{align}
\sum_{\{S\}} \prod_{(i,j)} (1 +  \hat{t}_{i,j} \; (\sigma_i \sigma_j)) 
&= 
\sum_{\{S\}} 1 + \sum_{(i,j)} \hat{t}_{i,j}\,(\sigma_i \sigma_j) + \sum_{(i,j)}\sum_{(k,l)} \hat{t}_{i,j}\, \hat{t}_{k,l}\,(\sigma_i \sigma_j)  (\sigma_k \sigma_l) \dots \\
&= \sum_{\{S\}} \, \sum_{\{(i_1,j_1,...,i_n,j_n)\}} \hat{t}_{i_1,j_1} \cdots \hat{t}_{i_n,j_n} \, (\sigma_{i_1} \sigma_{j_1})(\sigma_{i_2} \sigma_{i_2}) \cdots (\sigma_{i_n} \sigma_{j_n})
\end{align}

\noindent Aufgrund der Summation über alle Spinkonfigurationen $S$ tragen alle Produkte, in denen zumindest eine Spinvariable $\sigma_k$ in ungerader Anzahl vorkommt, nicht zum Gesamtergebnisse bei. Denn für jede Konfiguration $S = (\sigma_1, \dots, \sigma_k, \dots, \sigma_{2N})$ gibt es genau eine weitere Konfiguration $S' = (\sigma_1, \dots, -\sigma_k, \dots, \sigma_{2N})$ sodass sich diese dann aufheben. Eine Produkt von Isingsspin Paaren, in dem jede Spinvariable in gerader Anzahl vorkommt, hat hingegen immer den Wert 1, unabhängig von der Spinkonfiguration. Ein solches Produkt besitzt eine Repräsentation als geschlossener Graph $\mathcal{G}=(K,\Lambda')$ auf dem Gitter, wie in Abb. \ref{Abb: erlaubte Graphen} gezeigt. Zudem ist es möglich den Graphen $\mathcal{G}$ zu durchlaufen, ohne eine Kante zweimal zu passieren, da jedes Isingspin Paar nur höchstens einmal vorkommt. Gitterpunkte $i\in\Lambda'\subseteq\Lambda$ die in Graphen $\mathcal G$ enthalten sind bezeichnet man als Vertizes. Die Paare nächster Nachbarn, welche im Graphen enthalten sind, repräsentieren die Kanten des Graphen und werden in der Menge $K$ zusammengefasst. Die Summe $\hat{Z}[\hat{t}_{i,j}]$ lässt sich dann als Summe über alle geeigneten Graphen schreiben. 
\begin{grayframe}[frametitle = {Graphische Representation der Summe $\hat{Z}[\hat{t}_{i,j}]$} ]
\begin{equation} \label{eq: GR_pseudoZustanssumme}
\hat{Z}[\hat{t}_{i,j}] = 2^N \sum_{\mathcal G} \prod_{(i,j)\in K} \hat{t}_{i,j}
\end{equation} Die geeigneten Graphen $\mathcal G$ 
\begin{itemize}
\item[i)] sind geschlossen
\item[ii)] können durchlaufen werden ohne eine Kante zweimal zu nutzen.
\end{itemize}
\end{grayframe}

\noindent Die Summanden ergeben sich als die Gewichte der Kannten und durch die Summation über alle Spinkonfigurationen entsteht ein Gesamtfaktor von $2^N$. Sind alle $\hat{t}_{i,j} = t$ gleich, wie das für die Zustandssumme des Ising-Modells der Falls ist, ergibt sich \eqref{eq: GR_Zustandssumme}. Der Exponent $N(\mathcal G)$ ist dabei die Anzahl der Kanten des Graphen $\mathcal G$

\begin{equation} \label{eq: GR_Zustandssumme}
\sum_{\{S\}} \prod_{(i,j)} (1 +  t \; (\sigma_i \sigma_j)) = 2^N \sum_{\{ \mathcal G\}} t^{N(\mathcal G)}
\end{equation}

\begin{figure}
    \centering
     \draw[step=1cm,gray, ultra thin] (-3.5,-1.5) grid (4.5,1.5);

\node[grid_point] at (-3,1)   (m3p1) {};
\node[grid_point] at (-3,0)   (m3p0) {};
\node[grid_point] at (-3,-1)  (m3m1) {};

\node[grid_point] at (-2,1)   (m2p1) {};
\node[grid_point] at (-2,0)   (m2p0) {};
\node[grid_point] at (-2,-1)  (m2m1) {};

\node[grid_point] at (-1,1)   (m1p1) {};
\node[grid_point] at (-1,0)   (m1p0) {};
\node[grid_point] at (-1,-1)  (m1m1) {};

\node[grid_point] at (0,1)   (p0p1) {};
\node[grid_point] at (0,0)   (p0p0) {};
\node[grid_point] at (0,-1)  (p0m1) {};

\node[grid_point] at (3,1)   (p3p1) {};
\node[grid_point] at (3,0)   (p3p0) {};
\node[grid_point] at (3,-1)  (p3m1) {};

\node[grid_point] at (2,1)   (p2p1) {};
\node[grid_point] at (2,0)   (p2p0) {};
\node[grid_point] at (2,-1)  (p2m1) {};

\node[grid_point] at (1,1)   (p1p1) {};
\node[grid_point] at (1,0)   (p1p0) {};
\node[grid_point] at (1,-1)  (p1m1) {};

\node[grid_point] at (4,1)   (p4p1) {};
\node[grid_point] at (4,0)   (p4p0) {};
\node[grid_point] at (4,-1)  (p4m1) {};

%% Graph
\draw[-, black!100, very thick] (m3p1) -- (m2p1) ;
\draw[-, black!100, very thick] (m2p1) -- (m1p1) ;
\draw[-, black!100, very thick] (m1p1) -- (m1p0) ;
\draw[-, black!100, very thick] (m1p0) -- (p0p0) ;
\draw[-, black!100, very thick] (p0p0) -- (p0m1) ;
\draw[-, black!100, very thick] (p0m1) -- (m1m1) ;
\draw[-, black!100, very thick] (m1m1) -- (m2m1) ;
\draw[-, black!100, very thick] (m2m1) -- (m3m1) ;
\draw[-, black!100, very thick] (m3m1) -- (m3p0) ;
\draw[-, black!100, very thick] (m3p0) -- (m3p1) ;

\draw[-, black!100, very thick] (p1p1) -- (p2p1) ;
\draw[-, black!100, very thick] (p2p1) -- (p3p1) ;
\draw[-, black!100, very thick] (p3p1) -- (p3p0) ;
\draw[-, black!100, very thick] (p3p0) -- (p4p0) ;
\draw[-, black!100, very thick] (p4p0) -- (p4m1) ;
\draw[-, black!100, very thick] (p4m1) -- (p3m1) ;
\draw[-, black!100, very thick] (p3m1) -- (p3p0) ;
\draw[-, black!100, very thick] (p3p0) -- (p2p0) ;
\draw[-, black!100, very thick] (p2p0) -- (p1p0) ;
\draw[-, black!100, very thick] (p1p0) -- (p1p1) ;

    \caption{Ein erlaubter Graph auf dem Gitter mit einer Selbstüberschneidung. Die Graphen auf dem Gitter müssen nicht zusammenhängend sein.}
    \label{Abb: erlaubte Graphen}
\end{figure}

\subsection{Graphen ausgedrückt durch Graßmann Variablen } \label{sec: GraßmanGraphs}

\noindent Für ein Gitter $\Lambda$ mit $N$ Gitterpunkten, soll an jedem Punkt $i \in \Lambda$ vier Graßmann-Variablen $h_{i}^o, h_{i}^x, v_{i}^o, v_{i}^x$ eingeführt werden. Siehe Abbildung \ref{Abb: graßmanVariableBeiI}. Jede dieser Variablen besitzt neben der Position am Gitter zwei weitere Eigenschaften. Einerseits eine Ausrichtung, gekennzeichnet mit $h$ für `Horizontal' und $v$ für `Vertikal' und einen `Flavor', gekennzeichnet mit $o$ oder $x$. Zwei zum selben Gitterpunkt gehörige Variablen mit gleicher Ausichtung und unterschiedlichem Flavor, zum Beispiel  $h_{i}^o, h_{i}^x $ oder $ v_{i}^o, v_{i}^x$ werden als zueinander konjugiert bezeichnet.

\begin{figure}
\centering
\begin{tikzpicture}[scale = 0.5]
    \node[draw = none] at (0,0) (center) {$i$} ;
    \node[draw, circle, fill=none, scale = 0.5, very thick] at (-1,0) (center) {} ;
    \node[draw, circle, fill=none, scale = 0.5, very thick] at (0,-1)  {} ;
    \node[draw, cross out, very thick, scale = 0.6] at (1,0)  {} ;
    \node[draw, cross out, very thick, scale = 0.6] at (0,1)  {} ;
    \node[draw=none] at (-2,0)  {$h_{i}^o$} ;
    \node[draw=none] at (0,-2)  {$v_{i}^o$} ;
    \node[draw=none] at (2,0) {$h_{i}^x$} ;
    \node[draw=none] at (0,2) {$v_{i}^x$} ;
\end{tikzpicture}
\caption{Graßmann Variablen assoziiert mit Gitterpunkt $i\in\Lambda$}
\label{Abb: graßmanVariableBeiI}
\end{figure}

\noindent Die Idee ist es nun die Summe als Berezin-Integral einer geeigneten Funktion von Graßmann Variablen zu schreiben. Dazu wird eine quadratische Wirkung $A$ gemäß \eqref{eq: Ansatz Wirkung} eingeführt.  

\begin{equation} \label{eq: Ansatz Wirkung}
    \begin{aligned}
        A(h_{i}^o, h_{i}^x, v_{i}^o, v_{i}^x)  &= A_{bond} + A_{corner} + A_{monomer} \\
                 &\\
        A_{bond} &= \sum_{i \in \Lambda} a_{g,i} \; h_{i}^x \,h_{i+\vec{e}_x}^o
                                            + a_{h,i} \; v_{i}^x \,v_{i+\vec{e}_y}^o \\
        A_{corner} &= \sum_{i \in \Lambda} a_{a}\; h_{i}^x \,v_{i}^o 
                                            + a_{b}\; v_{i}^x\, h_{i}^o
                                            + a_{c}\; v_{i}^x \,h_{i}^x 
                                            + a_{d}\; v_{i}^x \,h_{i}^x\\
        A_{monomer} &= \sum_{i \in \Lambda} a_{e}\, h_{i}^o \,h_{i}^x
                                            + a_{f}\; v_{i}^o \,v_{i}^x
    \end{aligned}
\end{equation}

\noindent Die Pseudozustandssumme $\hat{Z}[\hat{t}]$ wird dann als Integral über die exponentierte Wirkung  geschrieben. Nach Abschnitt \ref{sec: Graßman-Diff-Int} ist die Reihenfolge der Variablen im ``Integrationsmaß'' entscheident für das Vorzeichen. In Anlehnung an die Statistische Physik soll die in \eqref{eq: Ansatz Zustandssumme} angegeben Integrationsoperation als `Spur' bezeichnet werden und mit $\Sp{.}$ notiert werden. 

\begin{equation} \label{eq: Ansatz Zustandssumme}
    \begin{aligned}
        \hat{Z}[\hat{t}] &= C\;\Sp{e^{A}} = C\;\int D_{h,v}\;e^{A}\\
        D_{h,v}& = \prod_{i \in \Lambda} dh_{i}^x\,dh_{i}^o\,dv_{i}^x\,dv_{i}^o \\
    \end{aligned}
\end{equation}

\noindent Es gilt die Konstanten $C$ und $a_{i, \alpha}$ mit $\alpha \in {a,b,c,d,e,f,g,h}$ zu bestimmen. Dazu wird jedem Paar von Graßmann Variablen eine graphische Interpretation zugeordnet. Diese sind in Abb. \ref{Abb: Graphische Interpretation Graßmann Paare} gezeigt. Die Richtung des Pfeil repräsentiert dabei die Reihenfolge der Variablen in den Paaren. Ein Vertauschen der Variablen resultiert, graphisch interpretiert, somit in einer Richtungsumkehrung des Pfeils. Im folgenden sollen diese acht Paare immer als $P_{i,\alpha}$ abgekürzt werden, wobei $\alpha$ der Buchstabe aus Abb. \ref{Abb: Graphische Interpretation Graßmann Paare} bzw. der Index des zugehörigen Koeffizienten ist. $i$ gibt die Position auf dem Gitter an.

\begin{figure}[h!]
    \centering
    \begin{tikzpicture}[node distance=0.15, scale = 2.2]
    
    %% corner a) hx vo
    \draw[step=1cm,gray, ultra thin] (-0.5,4.5) grid (0.5,5.5);
    \node[draw = none] at (0,5) (center) {} ;
    \node[draw, circle, fill=none, scale = 0.5, very thick] (ho) [left=of center]  {} ;
    \node[draw, circle, fill=none, scale = 0.5, very thick] (vo) [below=of center]  {} ;
    \node[draw, cross out, very thick, scale = 0.6] (hx) [right=of center] {} ;
    \node[draw, cross out, very thick, scale = 0.6] (vx) [above=of center] {} ;
    \draw[arrow_outer] (hx) .. controls (0.5, 4.65) and (0.35, 4.5) .. (vo);
    \node[draw = none, scale = 1.5, text width=10em] at (-0.5, 5.5) {a) };
    \node[draw = none, scale = 1.5, text width=7em] at (-0.5, 5.0) {$h_{\bm{x}_i}^x \,v_{\bm{x}_i}^o$};

    %% corner b) vx ho
    \draw[step=1cm,gray, ultra thin] (2.5,4.5) grid (3.5,5.5);
    \node[draw = none] at (3,5) (center) {} ;
    \node[draw, circle, fill=none, scale = 0.5, very thick] (ho) [left=of center]  {} ;
    \node[draw, circle, fill=none, scale = 0.5, very thick] (vo) [below=of center]  {} ;
    \node[draw, cross out, very thick, scale = 0.6] (hx) [right=of center] {} ;
    \node[draw, cross out, very thick, scale = 0.6] (vx) [above=of center] {} ;
    \draw[arrow_outer] (vx) .. controls (2.65, 5.5) and (2.6, 5.35) .. (ho);
    \node[draw = none, scale = 1.5, text width=10em] at (2.5, 5.5) {b) };
    \node[draw = none, scale = 1.5, text width=7em] at (2.5, 5.0) {$v_{\bm{x}_i}^x\, h_{\bm{x}_i}^o$};
    
    %% corner c) vx hx
    \draw[step=1cm,gray, ultra thin] (-0.5,3.5) grid (0.5,4.5);
    \node[draw = none] at (0,4) (center) {} ;
    \node[draw, circle, fill=none, scale = 0.5, very thick] (ho) [left=of center]  {} ;
    \node[draw, circle, fill=none, scale = 0.5, very thick] (vo) [below=of center]  {} ;
    \node[draw, cross out, very thick, scale = 0.6] (hx) [right=of center] {} ;
    \node[draw, cross out, very thick, scale = 0.6] (vx) [above=of center] {} ;
    \draw[arrow_outer] (vo) .. controls (-0.35, 3.5) and (-0.5, 3.65) .. (ho);
    \node[draw = none, scale = 1.5, text width=10em] at (-0.5, 4.5) {c) };
    \node[draw = none, scale = 1.5, text width=7em] at (-0.5, 4.0) {$v_{\bm{x}_i}^o \,h_{\bm{x}_i}^o$};

    %% corner d) vo ho
    \draw[step=1cm,gray, ultra thin] (2.5,3.5) grid (3.5,4.5);
    \node[draw = none] at (3,4) (center) {} ;
    \node[draw, circle, fill=none, scale = 0.5, very thick] (ho) [left=of center]  {} ;
    \node[draw, circle, fill=none, scale = 0.5, very thick] (vo) [below=of center]  {} ;
    \node[draw, cross out, very thick, scale = 0.6] (hx) [right=of center] {} ;
    \node[draw, cross out, very thick, scale = 0.6] (vx) [above=of center] {} ;
    \draw[arrow_outer] (vx) .. controls (3.35, 4.5) and (3.5, 4.35) .. (hx);
    \node[draw = none, scale = 1.5, text width=10em] at (2.5, 4.5) {d) };
    \node[draw = none, scale = 1.5, text width=7em] at (2.5, 4.0) {$v_{\bm{x}_i}^x \,h_{\bm{x}_i}^x$};
    
    %% in_conn e) ho hx
    \draw[step=1cm,gray, ultra thin] (-0.5,2.5) grid (0.5,3.5);
    \node[draw = none] at (0,3) (center) {} ;
    \node[draw, circle, fill=none, scale = 0.5, very thick] (ho) [left=of center]  {} ;
    \node[draw, circle, fill=none, scale = 0.5, very thick] (vo) [below=of center]  {} ;
    \node[draw, cross out, very thick, scale = 0.6] (hx) [right=of center] {} ;
    \node[draw, cross out, very thick, scale = 0.6] (vx) [above=of center] {} ;
    \draw[arrow_inner] (ho) -- (hx);
    \node[draw = none, scale = 1.5, text width=10em] at (-0.5, 3.5) {e) };
    \node[draw = none, scale = 1.5, text width=7em] at (-0.5, 3.0) {$h_{\bm{x}_i}^o \,h_{\bm{x}_i}^x$};

    %% in_conn f) vo vx
    \draw[step=1cm,gray, ultra thin] (2.5,2.5) grid (3.5,3.5);
    \node[draw = none] at (3,3) (center) {} ;
    \node[draw, circle, fill=none, scale = 0.5, very thick] (ho) [left=of center]  {} ;
    \node[draw, circle, fill=none, scale = 0.5, very thick] (vo) [below=of center]  {} ;
    \node[draw, cross out, very thick, scale = 0.6] (hx) [right=of center] {} ;
    \node[draw, cross out, very thick, scale = 0.6] (vx) [above=of center] {} ;
    \draw[arrow_inner] (vo) -- (vx);
    \node[draw = none, scale = 1.5, text width=10em] at (2.5, 3.5) {f) };
    \node[draw = none, scale = 1.5, text width=7em] at (2.5, 3.0) {$v_{\bm{x}_i}^o \,v_{\bm{x}_i}^x$};
    
    %% out_conn g) + f)
    \draw[step=1cm,gray, ultra thin] (-0.5,0.5) grid (1.5,1.5);
    \draw[step=1cm,gray, ultra thin] (-0.5,-0.5) grid (0.5,0.5);
    % gridpoint (x,y)
    \node[draw = none] at (0,1) (0center) {} ;
    \node[draw, circle, fill=none, scale = 0.5, very thick] (0ho) [left=of 0center]  {} ;
    \node[draw, circle, fill=none, scale = 0.5, very thick] (0vo) [below=of 0center]  {} ;
    \node[draw, cross out, very thick, scale = 0.6] (0hx) [right=of 0center] {} ;
    \node[draw, cross out, very thick, scale = 0.6] (0vx) [above=of 0center] {} ;
    % gridpoint (x,y) + e_x
    \node[draw = none] at (1,1) (1center) {} ;
    \node[draw, circle, fill=none, scale = 0.5, very thick] (1ho) [left=of 1center]  {} ;
    \node[draw, circle, fill=none, scale = 0.5, very thick] (1vo) [below=of 1center]  {} ;
    \node[draw, cross out, very thick, scale = 0.6] (1hx) [right=of 1center] {} ;
    \node[draw, cross out, very thick, scale = 0.6] (1vx) [above=of 1center] {} ;
     % gridpoint (x,y) - e_y
    \node[draw = none] at (0,0) (2center) {} ;
    \node[draw, circle, fill=none, scale = 0.5, very thick] (2ho) [left=of 2center]  {} ;
    \node[draw, circle, fill=none, scale = 0.5, very thick] (2vo) [below=of 2center]  {} ;
    \node[draw, cross out, very thick, scale = 0.6] (2hx) [right=of 2center] {} ;
    \node[draw, cross out, very thick, scale = 0.6] (2vx) [above=of 2center] {} ;
    % g) hx ho+1
    \draw[arrow_outer] (0hx)--(1ho);
    \node[draw = none, scale = 1.5, text width=10em] at (-0.5, 2.5) {g) };
    \node[draw = none, scale = 1.5, text width=7em] at (1.2, 1.85) {$h_{\bm{x}_i}^x\, h_{\bm{x}_i+\bm{e_{x}}}^o$};
    % h) vx vo+1
     \draw[arrow_outer] (2vx)--(0vo);
    \node[draw = none, scale = 1.5, text width=10em] at (-0.5, 1.5) {h) };
    \node[draw = none, scale = 1.5, text width=7em] at (-0.5, 0.5) {$v_{\bm{x}_i}^x\, v_{\bm{x}_i+\bm{e_{y}}}^o$};
    

    
\end{tikzpicture}
    \caption{Graphische Darstellung der Paare von Graßmann Variablen in \eqref{} }
    \label{Abb: Graphische Interpretation Graßmann Paare}
\end{figure}

\noindent Im folgenden sollen ungerichtete Graphen, wie sie in der Summe \eqref{eq: GR_pseudoZustanssumme} auftauchen untersucht werden. Jeder geeignete Graph lässt sich allein durch die graphischen Elemente \ref{Abb: Graphische Interpretation Graßmann Paare}\textit{g)} und \ref{Abb: Graphische Interpretation Graßmann Paare}\textit{g)} eindeutig festlegen. Somit repräsentiert eine Produkt der Paare $P_{i,g}$ oder $P_{i,h}$ einen Graphen auf dem Gitter. Die Richtung der Pfeile geben dabei jedoch nicht die Durchlaufrichtung des Graphen an, sondern die Reihenfolge der Variablen in den assoziierten Paaren. Wird jedoch über ein solches Produkt integriert, so verschwindet das Integral, da höchstens zwei von vier Graßmann Variablen pro Gritterpunkt auftauchen können. Die anderen Paare aus Abb. \ref{Abb: Graphische Interpretation Graßmann Paare} werden somit hinzugefügt, sodass jede Variable einmal vorkommt und das Integral nicht verschwindet. Produkte von $2N$ Paaren $P_{i,\alpha}$ repräsentieren somit einen Graphen auf dem Gitter. Gitterpunkte, die nicht Teil eines Graphen sind, sollen in Anlehnung an die Arbeit von Stuat Samuel als `Monomer' bezeichnet werden und können durch die Produkte $P_a \cdot P_b$, $P_c \cdot P_d$, $P_e \cdot P_f$ dargestellt werden. Betrachtet man die Definition der Exponentialfunktion genauer, so erkennt man dass $e^A$ eine Summe aller Produkte von Paaren $P_i,a$ bis $P_i,h$ beliebiger Paar-Anzahl und Reihenfolge darstellt. 

\begin{align}
\exp{A} &= \exp{\sum_{i \in\Lambda} \sum_{l \in \{a,\dots, h\}} a_{l, i} P_{l, i }} \\
        &=\sum_{k\in\mathbb N \setminus \{2N\}} \frac{1}{k!}\left( \sum_{l \in \Lambda \times {a,\dots,h}} a_{l} P_{l}\right)^k + \frac{1}{(2N)!} \left(\sum_{l \in \Lambda \times {a,\dots,h}} a_{l} P_{l}\right)^{2N} \label{eq: exp_sum_combi}
\end{align}


%\begin{alignat}{2}
%\exp{A} &= \exp{\sum_{i \in\Lambda} \sum_{l \in \{a,\dots, h\}} a_{l, i} P_{l, i }} \; & &= %\prod_{i \in \Lambda \times \{a,\dots, h\}} \exp{a_i P_i} \\  
%    &= \prod_{i \in \Lambda \times \{a,\dots, h\}} (1 + a_i P_i) & &= \sum_{I\subseteq \Lambda %\times \{a,\dots, h\} }  \prod_{i \in J} a_i P_i \\
%\end{alignat}

\noindent Die Integration liefert nur für Produkte von $2N$ Paaren einen von Null verschiedenen Wert, die linke Doppelsumme verschwindet somit unter der Spurbildung. Die rechte Summe besteht aus alle Produkten aus $2N$ Paaren die auf dem Gittermöglich sind. Von diesen liefern nur jenen einen von null verschiedenen Beitrag, wenn jede Graßmann Variable exakt einmal vorkommt. Diese Produkte entsprechen genau den gewünschten Graphen auf dem Gitter. Der Wert der Integration ist dann als das Produkt der Koeffizienten $a_l$ gegeben. Das Vorzeichen jedes Summanden ergibt sich durch umordnen der Variablen in die richtige Reihenfolge und ist durch das ``Integrationsmaß'' in \eqref{eq: Ansatz Zustandssumme} festgelegt. Da die Graßmann Paare, im Gegensatz zu den einzelnen Variablen, kommutieren, gibt es genau $(2N)!$ Produkte die, unter der Spurbildung, den selben Wert liefern. Diese Multiplizität hebt sich dann mit dem Koeffizient der Exponentialfunktion auf. Da nur die Kanten der Graphen zum Wert beitragen sollen, werden die Koeffizienten der Verbindgsstrecken als $a_{g,i} = \hat{t}_{i+1, i} \;\text{bzw.} =  t$ und $a_{h,i} = \hat{t}_{i, i+1} \;\text{bzw.} = t$  gewählt. Für die übrigen Koeffizienten bleibt nur mehr $a_{i,\alpha} \in \{-1,1\}$ für $\alpha \in \{a,\dots,f\}$. Diese müssen so gewählt werden dass das Vorzeichen für jeden Summanden gleich ist. Diese Bestimmung erweist sich als langwierig und wird in Appendix \ref{Appendix: Vorzeichen} abgehandelt. Als Resultat erhält man dass sich mit der Wahl $a_{i,\alpha} = -1 $ für $\alpha \in \{a,\dots,f\}$ für jeden Summanden das Vorzeichen $(-1)^N$ ergibt, sodass für den übrig gebliebene Koeffizienten $C = (-2)^N$ gelten muss. 

\begin{grayframe}[frametitle = {Ausdruck für Pseudo-Zustandssumme mit Graßmann Variablen}]

\begin{equation}
    \begin{aligned}
        \hat{Z}[t_{i,j}]  & = (-2)^N \int \cdots \int\prod_{j=1}^{N-1} dh_{j}^x\,dh_{j}^o\,dv_{j}^x,dv_{j}^o \;\;e^{A[t_{i,j}] } = (-2)^N \Sp{e^{A[t_{i,j}] }} \nonumber\\
       %D_{h,v}& =  \\
        %\\
        %A[t_{i,j}]   &= A_{bond} + A_{corner} + A_{monomer} \\
    \end{aligned}
\end{equation}
\begin{alignat}{2}
        & A[t_{i,j}]   &&= A_{bond} + A_{corner} + A_{monomer} \nonumber \\
        \label{eq: ZustandsBerezinIntegral} \\
        &A_{bond} &&= \; \sum_{j = 0}^{N-1} \hat{t}_{j+1, j} \; h_{\bm{x}_j}^o \,h_{\bm{x}_j+\bm{e}_x}^x
                                            + \hat{t}_{j, j+1} \; v_{\bm{x}_j}^x \,v_{\bm{x}_j+\bm{e}_y}^o \nonumber\\
       & A_{corner} &&= -\sum_{j = 0}^{N-1} h_{\bm{x}_j}^x \,v_{\bm{x}_j}^o 
                                            + v_{\bm{x}_j}^x\, h_{\bm{x}_j}^o
                                            + v_{\bm{x}_j}^x \,h_{\bm{x}_j}^x 
                                            + v_{\bm{x}_j}^o \,h_{\bm{x}_j}^o \nonumber\\
       &    A_{monomer}&&= -\sum_{j = 0}^{N-1} \, h_{\bm{x}_j}^o \,h_{\bm{x}_j}^x
                                            +  v_{\bm{x}_j}^o \,v_{\bm{x}_j}^x \nonumber
\end{alignat}
\end{grayframe}


Im Folgenden werden zunächst nur die in Abschnitt \ref{sec: GraßmanGraphs} eingeführten Graßmann-Paare $P_{i,\alpha}$ betrachtet. Die Koeffizienten werden außer acht gelassen und sollen am Ende so gewählt werden, dass die Spur für jedes Produkte von Graßmann-Paaren das gleiche Vorzeichen ergibt. Für die Graphische Bedeutung der Graßmann-Paare sei an Abb. \ref{Abb: Graphische Interpretation Graßmann Paare} errinert. 
 
\subsection{Vorzeichen der Monomer-Paare} \label{sec: vorzeicehnMonomer}

Für die Gitterpunkte die zu keinem Graphen gehören gibt es, wie in Abschnitt \ref{sec: GraßmanGraphs} beschrieben, drei mögliche Darstellungen. Somit kann dreimal der selbe Graph in der Summe \eqref{eq: exp_sum_combi} auftauchen, wodurch sich der Beitrag zum Vorzeichen des Graphen aus der Summe der drei einzelnen Vorzeichen ergibt. Die Monomer-Paare $P_{i,e}$ und $P_{i,f}$ sind bereits Richtig angeordnet. 

\begin{align}
\iint \,dh_{i}^x\,dh_{i}^o \; P_{i,e} &= \iint \,dh_{i}^x\,dh_{i}^o h_{i}^o\,h_{i}^x\;  = 1\\
\iint \,dv_{i}^x\,dv_{i}^o \; P_{i,f} &= \iint \,dv_{i}^x\,dv_{i}^o v_{i}^o\,v_{i}^x\;  = 1
\end{align} Daher ergibt sich für die 3 Darstellungen der Monomer

\begin{equation}
-\underbrace{h_{i}^x\, v_{i}^o \,v_{i}^x\,h_{i}^o}_{= P_{i,a}\cdot P_{i,b}} 
= h_{i}^x\, v_{i}^o\, h_{i}^o\,\,v_{i}^x 
= \underbrace{h_{i}^o\,h_{i}^x\, v_{i}^o\,v_{i}^x }_{= P_{i,e}\cdot P_{i,f}}= - h_{i}^o\,v_{i}^o\,h_{i}^x\,v_{i}^x 
= -\underbrace{v_{i}^o\,h_{i}^o\,v_{i}^x\,h_{i}^x}_{= P_{i,c}\cdot P_{i,d}}
\end{equation} Somit trägt jeder Gitterpunkt, der nicht teil eines Graphen ist, mit einem Faktor -1 zum Vorzeichen bei.

\subsection{Vorzeichen eines zusammenhängenden Graphen ohne Selbstüberschneidung}

Um das Vorzeichen geschlossener Graphen auswerten zu können, muss man die Graßmann Variablen derart umordnen, sodass die Integration am Ende 1 ergibt. Das Vorzeichen ergibt sich aus den notwendigen Vertauschungen. Hierzu soll ein graphisches Vorgehen die Rechnung ersetzen. Um das Vorzeichen bestimmen zu können, müssen die Paare für die Ecken $P_a$ bis $P_d$ und Verbindungen $P_g$, $P_h$ betrachtet werden. Die einzelnen Monomer-Paare sind jeweils schon richtig geordnet und können weggelassen werden. Zu Bestimmung des Vorzeichen eines geschlossenen Graphen geht man dann in folgenden Schritten vor:

\begin{itemize}
\item[0)] Man wählt ein Paar vom Typ $g$ oder $h$ als das Erste. Graphisch betrachtet, wählt man so auf dem Graphen einen Seite eines Gitterknotens als Startpunkt und geht in eine Richtung zum nächsten Gitterknoten. Diese Richtung legt die Durchlaufrichtung, des Graphen fest. 
\item[2)] Man wählt ein Paar als Nachfolger, welches eine zur zweiten Variable des Vorgängerpaares konjugierte Variable enthält. Die Konjungiertheit in diesem kontext wurde in Abschnitt \ref{sec: GraßmanGraphs} eingeführt. Damit ist das Nachfolgerpaar eindeutig festgelegt. Anschließend werden die beiden Variablen im Nachfolger Paar vertauscht, falls die konjungierte Variable nicht die erste variable ist. Nun kann ein weiterer Nachfolger bestimmt werden. Graphisch erfolgt ein Vorzeichen wechsel für jeden Pfeil im Graphen der entgegen der Durchlaufrichtung liegt.
Der Algorithmus wird fortgesetzt bis das nächste Paar, das erste wäre. Ist dies nie der Fall, ist der Graph nicht geschlossen und somit nicht relevant.
\item[3)] Nun wird die letzte Variable an den Anfang gehängt und die Paare um geklammert. Da die erste Variable eine mit $x$ Flavor ist, muss im nächsten schritt eine zusätzliche Vertauschung vorgenommen werden und es ergibt sich ein zusätzlicher Faktor -1. Dieses Vorzeichen geht auf die Schließung des Graphen zurück.
\item[4)] Durch Umklammern erhält man lauter Paare konjungierter Variablen. Wenn in jedem Paar $o$ vor $x$ kommt, ist das Integral positiv. Somit müssen Paare, wo dies nicht gilt, vertauscht werden. Graphisch fast man dafür bei jedem Knoten die Ports gleicher Orientierung zusammen, wie im Abb.\ref{Abb: VorzeichenBestimmung} mit roten Ellipsen gekennzeichnet.  Kommt hier in Durchlaufrichtung ein $x$ vor einem $o$ führt dies zu einem Vorzeichen wechsel.  
\end{itemize}


\begin{figure}
    \centering
    \captionsetup[subfigure]{labelformat=empty}
    \begin{subfigure}[c]{0.4\textwidth}
        \centering
        \begin{tikzpicture}[node distance=0.1, scale = 2.0]
    \draw[step=1cm,gray, ultra thin] (-1.5,-1.5) grid (1.4,1.5);
    
    %% 3x3 Grid
% gridpoint 1 = (0,0) 
\node[draw = none] at (0,0) (1center) {} ;
\node[draw, circle, fill=none, scale = 0.5, very thick] (1ho) [left=of 1center]  {} ;
\node[draw, circle, fill=none, scale = 0.5, very thick] (1vo) [below=of 1center]  {} ;
\node[draw, cross out, very thick, scale = 0.6] (1hx) [right=of 1center] {} ;
\node[draw, cross out, very thick, scale = 0.6] (1vx) [above=of 1center] {} ;

% gridpoint 2 = (1,0) 
\node[draw = none] at (1,0) (2center) {} ;
\node[draw, circle, fill=none, scale = 0.5, very thick] (2ho) [left=of 2center]  {} ;
\node[draw, circle, fill=none, scale = 0.5, very thick] (2vo) [below=of 2center]  {} ;
\node[draw, cross out, very thick, scale = 0.6] (2hx) [right=of 2center] {} ;
\node[draw, cross out, very thick, scale = 0.6] (2vx) [above=of 2center] {} ;
    
% gridpoint 3 = (1,1) 
\node[draw = none] at (1,1) (3center) {} ;
\node[draw, circle, fill=none, scale = 0.5, very thick] (3ho) [left=of 3center]  {} ;
\node[draw, circle, fill=none, scale = 0.5, very thick] (3vo) [below=of 3center]  {} ;
\node[draw, cross out, very thick, scale = 0.6] (3hx) [right=of 3center] {} ;
\node[draw, cross out, very thick, scale = 0.6] (3vx) [above=of 3center] {} ;

% gridpoint 4 = (0,1) 
\node[draw = none] at (0,1) (4center) {} ;
\node[draw, circle, fill=none, scale = 0.5, very thick] (4ho) [left=of 4center]  {} ;
\node[draw, circle, fill=none, scale = 0.5, very thick] (4vo) [below=of 4center]  {} ;
\node[draw, cross out, very thick, scale = 0.6] (4hx) [right=of 4center] {} ;
\node[draw, cross out, very thick, scale = 0.6] (4vx) [above=of 4center] {} ;

% gridpoint 5 = (-1,1) 
\node[draw = none] at (-1,1) (5center) {} ;
\node[draw, circle, fill=none, scale = 0.5, very thick] (5ho) [left=of 5center]  {} ;
\node[draw, circle, fill=none, scale = 0.5, very thick] (5vo) [below=of 5center]  {} ;
\node[draw, cross out, very thick, scale = 0.6] (5hx) [right=of 5center] {} ;
\node[draw, cross out, very thick, scale = 0.6] (5vx) [above=of 5center] {} ;

% gridpoint 6 = (-1,0) 
\node[draw = none] at (-1,0) (6center) {} ;
\node[draw, circle, fill=none, scale = 0.5, very thick] (6ho) [left=of 6center]  {} ;
\node[draw, circle, fill=none, scale = 0.5, very thick] (6vo) [below=of 6center]  {} ;
\node[draw, cross out, very thick, scale = 0.6] (6hx) [right=of 6center] {} ;
\node[draw, cross out, very thick, scale = 0.6] (6vx) [above=of 6center] {} ;
    
% gridpoint 7 = (-1,-1) 
\node[draw = none] at (-1,-1) (7center) {} ;
\node[draw, circle, fill=none, scale = 0.5, very thick] (7ho) [left=of 7center]  {} ;
\node[draw, circle, fill=none, scale = 0.5, very thick] (7vo) [below=of 7center]  {} ;
\node[draw, cross out, very thick, scale = 0.6] (7hx) [right=of 7center] {} ;
\node[draw, cross out, very thick, scale = 0.6] (7vx) [above=of 7center] {} ;
    
% gridpoint 8 = (0,-1) 
\node[draw = none] at (0,-1) (8center) {} ;
\node[draw, circle, fill=none, scale = 0.5, very thick] (8ho) [left=of 8center]  {} ;
\node[draw, circle, fill=none, scale = 0.5, very thick] (8vo) [below=of 8center]  {} ;
\node[draw, cross out, very thick, scale = 0.6] (8hx) [right=of 8center] {} ;
\node[draw, cross out, very thick, scale = 0.6] (8vx) [above=of 8center] {} ;    
    
% gridpoint 9 = (1,-1) 
\node[draw = none] at (1,-1) (9center) {} ;
\node[draw, circle, fill=none, scale = 0.5, very thick] (9ho) [left=of 9center]  {} ;
\node[draw, circle, fill=none, scale = 0.5, very thick] (9vo) [below=of 9center]  {} ;
\node[draw, cross out, very thick, scale = 0.6] (9hx) [right=of 9center] {} ;
\node[draw, cross out, very thick, scale = 0.6] (9vx) [above=of 9center] {} ;

%% graph
% connections
\draw[arrow_outer] (7hx) -- (8ho);
\draw[arrow_outer] (8hx) -- (9ho);
\draw[arrow_outer] (9vx) -- (2vo);
\draw[arrow_outer] (2vx) -- (3vo);
\draw[arrow_outer] (4hx) -- (3ho);
\draw[arrow_outer] (5hx) -- (4ho);
\draw[arrow_outer] (6vx) -- (5vo);
\draw[arrow_outer] (7vx) -- (6vo);
% corners
%\draw[->, blue!80, very thick] (1.3, -1.1)  arc[radius=0.2, start angle=0, end angle= -90];
\draw[arrow_outer] (9hx) .. controls (1.4, -1.25) and (1.25, -1.4)    .. (9vo); % corner a) at 9
\draw[arrow_outer] (5vx) .. controls (-1.25, 1.4) and (-1.4, 1.25)    .. (5ho); % corner b) at 5
\draw[arrow_outer] (3vx) .. controls (1.25, 1.4) and (1.4, 1.25)      .. (3hx); % corner c) at 3 
\draw[arrow_outer] (7vo) .. controls (-1.25, -1.4) and (-1.4, -1.25)  .. (7ho); % corner d) at 7
    
%% pairs
% at 2
\draw (1, 0) ellipse[x radius = 0.265, y radius = 0.1, rotate = 90, color = red!100];
% at 3
\draw (1, 1) ellipse[x radius = 0.265, y radius = 0.1, rotate = 90, color = red!100];
\draw (1, 1) ellipse[x radius = 0.265, y radius = 0.1, rotate = 0, color = red!100];
% at 4
\draw (0, 1) ellipse[x radius = 0.265, y radius = 0.1, rotate = 0, color = red!100];
% at 5
\draw (-1, 1) ellipse[x radius = 0.265, y radius = 0.1, rotate = 90, color = red!100];
\draw (-1, 1) ellipse[x radius = 0.265, y radius = 0.1, rotate = 0, color = red!100];
% at 6
\draw (-1, 0) ellipse[x radius = 0.265, y radius = 0.1, rotate = 90, color = red!100];
% at 7
\draw (-1, -1) ellipse[x radius = 0.265, y radius = 0.1, rotate = 90, color = red!100];
\draw (-1, -1) ellipse[x radius = 0.265, y radius = 0.1, rotate = 0, color = red!100];
% at 8
\draw (0, -1) ellipse[x radius = 0.265, y radius = 0.1, rotate = 0, color = red!100];
% at 9 
\draw (1, -1) ellipse[x radius = 0.265, y radius = 0.1, rotate = 90, color = red!100];
\draw (1, -1) ellipse[x radius = 0.265, y radius = 0.1, rotate = 0, color = red!100];

    
    \draw[arrow_start] (-1.5, -0.8) -- (-1.1,-0.8);
    \node[draw = none, scale = 0.8, ] at (-1.6, -0.7) {Start};
\end{tikzpicture}
    \end{subfigure}
    \hspace{0.1\textwidth}
    \begin{subfigure}[c]{0.4\textwidth}
        \subcaption{Um das Vorzeichen des Graphen zu bestimmen wählt man einen Startpunkt, hier ein $x$. Dann bewegt man sich in die Richtung des ersten Pfeiles vom Startpunkt weg durch den Graphen. Dabei bewegt man sich bei den Ellipsen immer entlang der längeren Achse. Für jeden Pfeil der entgegen der Bewegungsrichtung zeigt, sowie immer wenn in einer Ellipse $x$ vor $o$ kommt, erhält man einen Vorzeichenwechsel. Hier erhält man 5 Vorzeichenwechsel für die Pfeilrichtung, 6 Vorzeichenwechsel für die $xo$ Paare und einen zusätzlich für die Schließung des Graphen. Insgesamt also $+1$ als Vorzeichen.  }
    \end{subfigure}
    \caption{Vorzeichenbestimmung geschlossener zusammenhängender Graphen } \label{Abb: VorzeichenBestimmung}
\end{figure}

\noindent Mithilfe dieser graphischen Regel kann nun leicht das Vorzeichen eines beliebigen geschlossenen Graphen bestimmt werden. Die Regel ermöglicht jedoch auch eine Segmentierung eines Graphen, in unterschiedliche Bausteine wie in Abb. \ref{Abb: directedElemets}. Aus der in Abb. \ref{Abb: Segmentierung} gezeigten Segmentierung in entgegengesetzte Richtungen lassen sich 12 Bausteine ermitteln, mit denen jeder geschlossene Graph ohne Selbstüberschneidung gebaut werden kann. Die Vorzeichen dieser Bausteine sind in Abb. \ref{Abb: directedElemets} angegeben. Für $b$) ergibt sich zum Beispiel ein Vorzeichenwechsel für das Zweite $xo$ Paar und den zweiten Pfeil der gegen die Durchlaufrichtung zeigt. Insgesamt also ein positives Vorzeichen. Das Vorzeichen eines geschlossenen Graphen ohne Selbstüberschneidung ergibt sich dann als Produkt der Vorzeichen der Segmente und dem Vorzeichen für die Schließung des Graphen. Für den geschlossenen Graphen in Abb. \ref{Abb: VorzeichenBestimmung} ergibt sich insgesamt ein Positives Vorzeichen. Durch hinzufügen von Elementen $a$) bis $\bar h$) lässt sich jeder beliebige geschlossene, zusammenhängende Graph ohne Selbsüberschneidung erstellen. Dabei kommen die Elemente $a$) bis $d$) immer paarweise mit ihrer komplementären Sequenz $\bar a$) bis $\bar d$) vor. Somit ändert sich das Vorzeichen nicht und alle geschlossenen zusammenhängenden Graphen ohne Selbsüberschneidung haben positives Vorzeichen. 

\begin{figure}
    %\begin{tikzpicture}[node distance=0.1, scale = 2.0]
    \draw[step=1cm,gray, ultra thin] (-1.5,-1.5) grid (1.4,1.5);
    
    %% 3x3 Grid
% gridpoint 1 = (0,0) 
\node[draw = none] at (0,0) (1center) {} ;
\node[draw, circle, fill=none, scale = 0.5, very thick] (1ho) [left=of 1center]  {} ;
\node[draw, circle, fill=none, scale = 0.5, very thick] (1vo) [below=of 1center]  {} ;
\node[draw, cross out, very thick, scale = 0.6] (1hx) [right=of 1center] {} ;
\node[draw, cross out, very thick, scale = 0.6] (1vx) [above=of 1center] {} ;

% gridpoint 2 = (1,0) 
\node[draw = none] at (1,0) (2center) {} ;
\node[draw, circle, fill=none, scale = 0.5, very thick] (2ho) [left=of 2center]  {} ;
\node[draw, circle, fill=none, scale = 0.5, very thick] (2vo) [below=of 2center]  {} ;
\node[draw, cross out, very thick, scale = 0.6] (2hx) [right=of 2center] {} ;
\node[draw, cross out, very thick, scale = 0.6] (2vx) [above=of 2center] {} ;
    
% gridpoint 3 = (1,1) 
\node[draw = none] at (1,1) (3center) {} ;
\node[draw, circle, fill=none, scale = 0.5, very thick] (3ho) [left=of 3center]  {} ;
\node[draw, circle, fill=none, scale = 0.5, very thick] (3vo) [below=of 3center]  {} ;
\node[draw, cross out, very thick, scale = 0.6] (3hx) [right=of 3center] {} ;
\node[draw, cross out, very thick, scale = 0.6] (3vx) [above=of 3center] {} ;

% gridpoint 4 = (0,1) 
\node[draw = none] at (0,1) (4center) {} ;
\node[draw, circle, fill=none, scale = 0.5, very thick] (4ho) [left=of 4center]  {} ;
\node[draw, circle, fill=none, scale = 0.5, very thick] (4vo) [below=of 4center]  {} ;
\node[draw, cross out, very thick, scale = 0.6] (4hx) [right=of 4center] {} ;
\node[draw, cross out, very thick, scale = 0.6] (4vx) [above=of 4center] {} ;

% gridpoint 5 = (-1,1) 
\node[draw = none] at (-1,1) (5center) {} ;
\node[draw, circle, fill=none, scale = 0.5, very thick] (5ho) [left=of 5center]  {} ;
\node[draw, circle, fill=none, scale = 0.5, very thick] (5vo) [below=of 5center]  {} ;
\node[draw, cross out, very thick, scale = 0.6] (5hx) [right=of 5center] {} ;
\node[draw, cross out, very thick, scale = 0.6] (5vx) [above=of 5center] {} ;

% gridpoint 6 = (-1,0) 
\node[draw = none] at (-1,0) (6center) {} ;
\node[draw, circle, fill=none, scale = 0.5, very thick] (6ho) [left=of 6center]  {} ;
\node[draw, circle, fill=none, scale = 0.5, very thick] (6vo) [below=of 6center]  {} ;
\node[draw, cross out, very thick, scale = 0.6] (6hx) [right=of 6center] {} ;
\node[draw, cross out, very thick, scale = 0.6] (6vx) [above=of 6center] {} ;
    
% gridpoint 7 = (-1,-1) 
\node[draw = none] at (-1,-1) (7center) {} ;
\node[draw, circle, fill=none, scale = 0.5, very thick] (7ho) [left=of 7center]  {} ;
\node[draw, circle, fill=none, scale = 0.5, very thick] (7vo) [below=of 7center]  {} ;
\node[draw, cross out, very thick, scale = 0.6] (7hx) [right=of 7center] {} ;
\node[draw, cross out, very thick, scale = 0.6] (7vx) [above=of 7center] {} ;
    
% gridpoint 8 = (0,-1) 
\node[draw = none] at (0,-1) (8center) {} ;
\node[draw, circle, fill=none, scale = 0.5, very thick] (8ho) [left=of 8center]  {} ;
\node[draw, circle, fill=none, scale = 0.5, very thick] (8vo) [below=of 8center]  {} ;
\node[draw, cross out, very thick, scale = 0.6] (8hx) [right=of 8center] {} ;
\node[draw, cross out, very thick, scale = 0.6] (8vx) [above=of 8center] {} ;    
    
% gridpoint 9 = (1,-1) 
\node[draw = none] at (1,-1) (9center) {} ;
\node[draw, circle, fill=none, scale = 0.5, very thick] (9ho) [left=of 9center]  {} ;
\node[draw, circle, fill=none, scale = 0.5, very thick] (9vo) [below=of 9center]  {} ;
\node[draw, cross out, very thick, scale = 0.6] (9hx) [right=of 9center] {} ;
\node[draw, cross out, very thick, scale = 0.6] (9vx) [above=of 9center] {} ;

%% graph
% connections
\draw[arrow_outer] (7hx) -- (8ho);
\draw[arrow_outer] (8hx) -- (9ho);
\draw[arrow_outer] (9vx) -- (2vo);
\draw[arrow_outer] (2vx) -- (3vo);
\draw[arrow_outer] (4hx) -- (3ho);
\draw[arrow_outer] (5hx) -- (4ho);
\draw[arrow_outer] (6vx) -- (5vo);
\draw[arrow_outer] (7vx) -- (6vo);
% corners
%\draw[->, blue!80, very thick] (1.3, -1.1)  arc[radius=0.2, start angle=0, end angle= -90];
\draw[arrow_outer] (9hx) .. controls (1.4, -1.25) and (1.25, -1.4)    .. (9vo); % corner a) at 9
\draw[arrow_outer] (5vx) .. controls (-1.25, 1.4) and (-1.4, 1.25)    .. (5ho); % corner b) at 5
\draw[arrow_outer] (3vx) .. controls (1.25, 1.4) and (1.4, 1.25)      .. (3hx); % corner c) at 3 
\draw[arrow_outer] (7vo) .. controls (-1.25, -1.4) and (-1.4, -1.25)  .. (7ho); % corner d) at 7
    
%% pairs
% at 2
\draw (1, 0) ellipse[x radius = 0.265, y radius = 0.1, rotate = 90, color = red!100];
% at 3
\draw (1, 1) ellipse[x radius = 0.265, y radius = 0.1, rotate = 90, color = red!100];
\draw (1, 1) ellipse[x radius = 0.265, y radius = 0.1, rotate = 0, color = red!100];
% at 4
\draw (0, 1) ellipse[x radius = 0.265, y radius = 0.1, rotate = 0, color = red!100];
% at 5
\draw (-1, 1) ellipse[x radius = 0.265, y radius = 0.1, rotate = 90, color = red!100];
\draw (-1, 1) ellipse[x radius = 0.265, y radius = 0.1, rotate = 0, color = red!100];
% at 6
\draw (-1, 0) ellipse[x radius = 0.265, y radius = 0.1, rotate = 90, color = red!100];
% at 7
\draw (-1, -1) ellipse[x radius = 0.265, y radius = 0.1, rotate = 90, color = red!100];
\draw (-1, -1) ellipse[x radius = 0.265, y radius = 0.1, rotate = 0, color = red!100];
% at 8
\draw (0, -1) ellipse[x radius = 0.265, y radius = 0.1, rotate = 0, color = red!100];
% at 9 
\draw (1, -1) ellipse[x radius = 0.265, y radius = 0.1, rotate = 90, color = red!100];
\draw (1, -1) ellipse[x radius = 0.265, y radius = 0.1, rotate = 0, color = red!100];

    
    \draw[arrow_start] (-1.5, -0.8) -- (-1.1,-0.8);
    \node[draw = none, scale = 0.8, ] at (-1.6, -0.7) {Start};
\end{tikzpicture}
    \begin{subfigure}[c]{0.3\textwidth}
    \begin{tikzpicture}[node distance=0.1, scale = 2.25]
    %\draw[step=1cm,gray, ultra thin] (-1.5,-1.5) grid (1.5,1.5);
    %\draw[step=1cm,gray, ultra thin] ( 1.6,-1.5) grid (4.4,1.5);
    
    %% 3x3 Grid
% gridpoint 1 = (0,0) 
\node[draw = none] at (0,0) (1center) {} ;
\node[draw, circle, fill=none, scale = 0.5, very thick] (1ho) [left=of 1center]  {} ;
\node[draw, circle, fill=none, scale = 0.5, very thick] (1vo) [below=of 1center]  {} ;
\node[draw, cross out, very thick, scale = 0.6] (1hx) [right=of 1center] {} ;
\node[draw, cross out, very thick, scale = 0.6] (1vx) [above=of 1center] {} ;

% gridpoint 2 = (1,0) 
\node[draw = none] at (1,0) (2center) {} ;
\node[draw, circle, fill=none, scale = 0.5, very thick] (2ho) [left=of 2center]  {} ;
\node[draw, circle, fill=none, scale = 0.5, very thick] (2vo) [below=of 2center]  {} ;
\node[draw, cross out, very thick, scale = 0.6] (2hx) [right=of 2center] {} ;
\node[draw, cross out, very thick, scale = 0.6] (2vx) [above=of 2center] {} ;
    
% gridpoint 3 = (1,1) 
\node[draw = none] at (1,1) (3center) {} ;
\node[draw, circle, fill=none, scale = 0.5, very thick] (3ho) [left=of 3center]  {} ;
\node[draw, circle, fill=none, scale = 0.5, very thick] (3vo) [below=of 3center]  {} ;
\node[draw, cross out, very thick, scale = 0.6] (3hx) [right=of 3center] {} ;
\node[draw, cross out, very thick, scale = 0.6] (3vx) [above=of 3center] {} ;

% gridpoint 4 = (0,1) 
\node[draw = none] at (0,1) (4center) {} ;
\node[draw, circle, fill=none, scale = 0.5, very thick] (4ho) [left=of 4center]  {} ;
\node[draw, circle, fill=none, scale = 0.5, very thick] (4vo) [below=of 4center]  {} ;
\node[draw, cross out, very thick, scale = 0.6] (4hx) [right=of 4center] {} ;
\node[draw, cross out, very thick, scale = 0.6] (4vx) [above=of 4center] {} ;

% gridpoint 5 = (-1,1) 
\node[draw = none] at (-1,1) (5center) {} ;
\node[draw, circle, fill=none, scale = 0.5, very thick] (5ho) [left=of 5center]  {} ;
\node[draw, circle, fill=none, scale = 0.5, very thick] (5vo) [below=of 5center]  {} ;
\node[draw, cross out, very thick, scale = 0.6] (5hx) [right=of 5center] {} ;
\node[draw, cross out, very thick, scale = 0.6] (5vx) [above=of 5center] {} ;

% gridpoint 6 = (-1,0) 
\node[draw = none] at (-1,0) (6center) {} ;
\node[draw, circle, fill=none, scale = 0.5, very thick] (6ho) [left=of 6center]  {} ;
\node[draw, circle, fill=none, scale = 0.5, very thick] (6vo) [below=of 6center]  {} ;
\node[draw, cross out, very thick, scale = 0.6] (6hx) [right=of 6center] {} ;
\node[draw, cross out, very thick, scale = 0.6] (6vx) [above=of 6center] {} ;
    
% gridpoint 7 = (-1,-1) 
\node[draw = none] at (-1,-1) (7center) {} ;
\node[draw, circle, fill=none, scale = 0.5, very thick] (7ho) [left=of 7center]  {} ;
\node[draw, circle, fill=none, scale = 0.5, very thick] (7vo) [below=of 7center]  {} ;
\node[draw, cross out, very thick, scale = 0.6] (7hx) [right=of 7center] {} ;
\node[draw, cross out, very thick, scale = 0.6] (7vx) [above=of 7center] {} ;
    
% gridpoint 8 = (0,-1) 
\node[draw = none] at (0,-1) (8center) {} ;
\node[draw, circle, fill=none, scale = 0.5, very thick] (8ho) [left=of 8center]  {} ;
\node[draw, circle, fill=none, scale = 0.5, very thick] (8vo) [below=of 8center]  {} ;
\node[draw, cross out, very thick, scale = 0.6] (8hx) [right=of 8center] {} ;
\node[draw, cross out, very thick, scale = 0.6] (8vx) [above=of 8center] {} ;    
    
% gridpoint 9 = (1,-1) 
\node[draw = none] at (1,-1) (9center) {} ;
\node[draw, circle, fill=none, scale = 0.5, very thick] (9ho) [left=of 9center]  {} ;
\node[draw, circle, fill=none, scale = 0.5, very thick] (9vo) [below=of 9center]  {} ;
\node[draw, cross out, very thick, scale = 0.6] (9hx) [right=of 9center] {} ;
\node[draw, cross out, very thick, scale = 0.6] (9vx) [above=of 9center] {} ;

%% graph
% connections
\draw[arrow_outer] (7hx) -- (8ho);
\draw[arrow_outer] (8hx) -- (9ho);
\draw[arrow_outer] (9vx) -- (2vo);
\draw[arrow_outer] (2vx) -- (3vo);
\draw[arrow_outer] (4hx) -- (3ho);
\draw[arrow_outer] (5hx) -- (4ho);
\draw[arrow_outer] (6vx) -- (5vo);
\draw[arrow_outer] (7vx) -- (6vo);
% corners
%\draw[->, blue!80, very thick] (1.3, -1.1)  arc[radius=0.2, start angle=0, end angle= -90];
\draw[arrow_outer] (9hx) .. controls (1.4, -1.25) and (1.25, -1.4)    .. (9vo); % corner a) at 9
\draw[arrow_outer] (5vx) .. controls (-1.25, 1.4) and (-1.4, 1.25)    .. (5ho); % corner b) at 5
\draw[arrow_outer] (3vx) .. controls (1.25, 1.4) and (1.4, 1.25)      .. (3hx); % corner c) at 3 
\draw[arrow_outer] (7vo) .. controls (-1.25, -1.4) and (-1.4, -1.25)  .. (7ho); % corner d) at 7
    
%% pairs
% at 2
\draw (1, 0) ellipse[x radius = 0.265, y radius = 0.1, rotate = 90, color = red!100];
% at 3
\draw (1, 1) ellipse[x radius = 0.265, y radius = 0.1, rotate = 90, color = red!100];
\draw (1, 1) ellipse[x radius = 0.265, y radius = 0.1, rotate = 0, color = red!100];
% at 4
\draw (0, 1) ellipse[x radius = 0.265, y radius = 0.1, rotate = 0, color = red!100];
% at 5
\draw (-1, 1) ellipse[x radius = 0.265, y radius = 0.1, rotate = 90, color = red!100];
\draw (-1, 1) ellipse[x radius = 0.265, y radius = 0.1, rotate = 0, color = red!100];
% at 6
\draw (-1, 0) ellipse[x radius = 0.265, y radius = 0.1, rotate = 90, color = red!100];
% at 7
\draw (-1, -1) ellipse[x radius = 0.265, y radius = 0.1, rotate = 90, color = red!100];
\draw (-1, -1) ellipse[x radius = 0.265, y radius = 0.1, rotate = 0, color = red!100];
% at 8
\draw (0, -1) ellipse[x radius = 0.265, y radius = 0.1, rotate = 0, color = red!100];
% at 9 
\draw (1, -1) ellipse[x radius = 0.265, y radius = 0.1, rotate = 90, color = red!100];
\draw (1, -1) ellipse[x radius = 0.265, y radius = 0.1, rotate = 0, color = red!100];

    
    % direction
    %\draw[->, black!60, ultra thick] (0,0.4)  arc[radius=0.4, start angle= 90, end angle= 420];
    
    % draw boxes
    \draw[thick] (0.75,-1.4) -- (1.4,-1.4) -- (1.4, -0.25) -- (0.75,-0.25) -- (0.75,-1.4);
    \draw[thick] (0.75,-0.25) -- (1.4, -0.25) -- (1.4, 0.75) -- (0.75, 0.75) -- (0.75, -0.25);
    \draw[thick] (0.25, 0.75) -- (1.4, 0.75) -- (1.4, 1.4) -- (0.25, 1.4) -- (0.25, 0.75);
    \draw[thick] (-0.75, 0.75) -- (0.25, 0.75) -- (0.25, 1.4) -- (-0.75, 1.4) -- (-0.75, 0.75);
    \draw[thick] (-1.4, 0.25) -- (-0.75, 0.25) -- (-0.75, 1.4) -- (-1.4, 1.4) -- (-1.4, 0.25);
    \draw[thick] (-1.4, -0.75) -- (-0.75, -0.75) -- (-0.75, 0.25) -- (-1.4, 0.25) -- (-1.4, -0.75);
    \draw[thick] (-1.4, -1.4) -- (-0.25, -1.4) -- (-0.25, -0.75) -- (-1.4, -0.75) -- (-1.4, -1.4);
    \draw[thick] (-0.25,-1.4) -- (0.75, -1.4) -- (0.75, -0.75) -- (-0.25, -0.75) -- (-0.25,-1.4);
    
    % arrows
    \draw[->, blue!80, very thick] (1.65, 0) -- (1.65, 0.5);
    \draw[->, blue!80, very thick] (0, 1.65) -- (-0.5, 1.65);  
    \draw[->, blue!80, very thick] (-1.65, 0) -- (-1.65, -0.5);
    \draw[->, blue!80, very thick] (0, -1.65) -- (0.5, -1.65);  
    \draw[->, blue!80, very thick] (1.65, 1)  arc[radius=0.65, start angle=0 , end angle= 90];
    \draw[->, blue!80, very thick] (-1, 1.65)  arc[radius=0.65, start angle=90 , end angle= 180];
    \draw[->, blue!80, very thick] (-1.65, -1)  arc[radius=0.65, start angle=180 , end angle= 270];
    \draw[->, blue!80, very thick] (1, -1.65)  arc[radius=0.65, start angle=270 , end angle= 360];
\end{tikzpicture}
    \subcaption{Segmentierung gegen den Uhrzeigersin}
    \end{subfigure}
    \hspace{0.2\textwidth}
    \begin{subfigure}[c]{0.3\textwidth}
    \input{Diagrams/Evaluation/clockwiseSquareGraph.tex}
    \subcaption{Segmentierung in Uhrzeigersinn}
    \end{subfigure}
    \caption{Segmentierung des Graphen aus Abb. \ref{Abb: VorzeichenBestimmung}}
    \label{Abb: Segmentierung}
\end{figure}

\begin{figure}[h!]
    \centering
    \begin{tikzpicture}[scale = 1.5]

\begin{scope}
\draw[step=1cm, white, ultra thin] (-2.5,0) grid (5,0);
\draw[->, blue!80, very thick] (0,-0.325)  arc[radius=0.65, start angle=270 , end angle= 360];
\draw[->, blue!80, very thick] (4,-0.325)  arc[radius=0.65, start angle=-180 , end angle= -270];
\node[draw = none, text width=2em, scale = 1.5 ] at (-0.02,0) (Num1) {$a$)};
\node[draw = none, text width=11em, scale = 1 ] at (2.5,0) (Num1) {+\;+\;+\;+ = +1};
\node[draw = none, text width=2em, scale = 1.5 ] at (3.98,0) (Num1) {$\bar a$)};
\node[draw = none, text width=11em, scale = 1 ] at (6.5,0) (Num1) {+\;+\;+\;+ = +1};
\end{scope}

\begin{scope}[shift = {(0,-1)}]
\draw[->, blue!80, very thick] (0.65,-0.325)  arc[radius=0.65, start angle=0 , end angle= 90];
\draw[->, blue!80, very thick] (4.65,-0.325)  arc[radius=0.65, start angle=-90 , end angle= -180];
\node[draw = none, text width=2em, scale = 1.5 ] at (-0.02,0) (Num1) {$b$)};
\node[draw = none, text width=11em, scale = 1 ] at (2.5,0) (Num1) {+\;+\;-\;- = +1};
\node[draw = none, text width=2em, scale = 1.5 ] at (3.98,0) (Num1) {$\bar b$)};
\node[draw = none, text width=11em, scale = 1 ] at (6.5,0) (Num1) {-\;-\;+\;+ = +1};
\end{scope}

\begin{scope}[shift = {(0,-2)}]
\draw[->, blue!80, very thick] (0.65, 0.35)  arc[radius=0.65, start angle=90 , end angle= 180];
\draw[->, blue!80, very thick] (4.65, 0.35)  arc[radius=0.65, start angle=0 , end angle= -90];
\node[draw = none, text width=2em, scale = 1.5 ] at (-0.02,0) (Num1) {$c$)};
\node[draw = none, text width=11em, scale = 1 ] at (2.5,0) (Num1) {-\;-\;-\;- = +1};
\node[draw = none, text width=2em, scale = 1.5 ] at (3.98,0) (Num1) {$\bar c$)};
\node[draw = none, text width=11em, scale = 1 ] at (6.5,0) (Num1) {-\;-\;-\;- = +1};
\end{scope}

\begin{scope}[shift = {(0,-3)}]
\draw[->, blue!80, very thick] (0, 0.35)  arc[radius=0.65, start angle=170 , end angle= 270];
\draw[->, blue!80, very thick] (4, 0.35)  arc[radius=0.65, start angle=-270 , end angle= -360];
\node[draw = none, text width=2em, scale = 1.5 ] at (-0.02,0) (Num1) {$d$)};
\node[draw = none, text width=11em, scale = 1 ] at (2.5,0) (Num1) {-\;-\;-\;+ = -1};
\node[draw = none, text width=2em, scale = 1.5 ] at (3.98,0) (Num1) {$\bar d$)};
\node[draw = none, text width=11em, scale = 1 ] at (6.5,0) (Num1) {+\;-\;-\;- = -1};
\end{scope}

\begin{scope}[shift = {(0,-4)}]
\draw[->, blue!80, very thick] (0.25, -0.25) -- (0.25, 0.25); 
\draw[->, blue!80, very thick] (4.25,  0.25) -- (4.25, -0.25);
\node[draw = none, text width=2em, scale = 1.5 ] at (-0.02,0) (Num1) {$g$)};
\node[draw = none, text width=11em, scale = 1 ] at (2.5,0) (Num1) {+\;+ = +1};
\node[draw = none, text width=2em, scale = 1.5 ] at (3.98,0) (Num1) {$\bar g$)};
\node[draw = none, text width=11em, scale = 1 ] at (6.5,0) (Num1) {+\;+ = +1};
\end{scope}

\begin{scope}[shift = {(0,-5)}]
\draw[->, blue!80, very thick] (0, 0) -- (0.5, 0); % right 
\draw[->, blue!80, very thick] (4.5, 0) -- (4, 0); % left
\node[draw = none, text width=2em, scale = 1.5 ] at (-0.02,0) (Num1) {$h$)};
\node[draw = none, text width=11em, scale = 1 ] at (2.5,0) (Num1) {+\;+ = +1};
\node[draw = none, text width=2em, scale = 1.5 ] at (3.98,0) (Num1) {$\bar h$)};
\node[draw = none, text width=11em, scale = 1 ] at (6.5,0) (Num1) {+\;+ = +1};
\end{scope}

\end{tikzpicture}
    \caption{Alle Teilsequenzen eines beliebigen gerichteten Graphen }
    \label{Abb: directedElemets}
\end{figure}


\subsection{Vorzeichen von Überschneidungen}

Das Vorzeichen eines geschlossenen, zusammenhängenden Graphen mit einer Selbstüberschneidung kann mit der selben Methode bestimmt werden, wie für geschlossenen Graphen ohne Selbstüberschneidung. Das Vorzeichen ergibt sich dann als Produkt des Vorzeichens $V_c$ für die Schließung, des Vorzeichens $V_k$ für die Kreuzung und des Vorzeichens $V_s$ aufgrund der Vorkommenden Segmente $a)$ bis $\bar h)$. Die Kreuzung kann durch keines der Segmente $a)$ bis $\bar h)$ dargestellt werden. Der Algorithmus zur Bestimmung des Vorzeichens durch Umordnen legt aber eindeutig fest, wie die Kreuzung durchlaufen wird. Dadurch kann der Graph in zwei Schleifen mit entgegengesetzter Durchlaufrichtung aufgeteilt werden. Ersetzt man die Kreuzung durch zwei Elemente $x$ und $\bar x$, wobei x eines der Segmente  $a)$, $b)$, $c)$ und $d)$ ist, erhält man zwei geschlossenen Graphen mit positiven Vorzeichen, welches sich als Produkt $V_{1,s}\cdot V_{1,c}$ bzw. $V_{2,s}\cdot V_{2,c}$. Damit die Ersetzung der Kreuzung durch der Segmentpaare $x-\bar x$ eine gültige Darstellung als Produkt von Graßman Paaren hat müssen die Getrennten Graphen auf dem Gitter außeinandergeschoben werden und die Terme $P_a \cdot P_b$ oder $P_c \cdot P_d$ hinzugefügt werden. Die translation am gitter ändert das Vorzeichen nicht, aber die zusätzlichen paare führen jedoch zu einem negativen Vorzeichen. Das Auftauschen der Kreuzung trägt also mit einem negativen Vorzeichen bei. Dies erkennt man auch aus der Rechnung . 

\begin{equation}
\begin{aligned}
&V_c \cdot V_k \cdot V_s = V_c \cdot V_k \cdot V_{1,s} \cdot V_{2,s}   \overset{!}{=} V_{1,c} \cdot V_{1,s}  \cdot V_{2,c} \cdot V_{2,s} \\
\\
& V_{c} =  V_{1,c} = V_{2,c} = -1  \;\;\;\;\Rightarrow\;\;\;\;   V_k = -1
\end{aligned}
\end{equation}

\noindent Die Argumentation lässt sich nun leicht auf Graphen mit $N_k$ Selbstüberschneidungen ausweiten, indem man diese induktiv wie oben in Graphen ohne Selbstüberschneidung zerlegt. Dabei liefert jede Kreuzung ein negatives Vorzeichen. 

\subsection{Wahl der Koeffizienten}
In der Arbeit wurden alle Koeffizienten $a_{i,\alpha}$ für $\alpha \in \{a,b,c,d,e,f\}$ zu $-1$ gewählt. Mit dieser Wahl tragen alle Monomer nach wie vor mit $-1$ ein zum Gesamtvorzeichen bei, da diese immer mit einem Produkt ${a_{i,\alpha} \cdot a_{i,\alpha'} = 1}$ auftreten. Der Faktor $-1$ kommt dabei durch die Spurbildung, wie in Abschnitt \ref{sec: vorzeicehnMonomer} beschrieben, zustande. Die Kreuzungen tragen, unabhängig von der Wahl der $a_{i,\alpha}$, einem Faktor $-1$ bei. Jedes Segment $a)$ bis $\bar{h})$ eines Graphen kommt mit einen Koeffizienten $a_{i,\alpha}$ und liefert somit einen Faktor $-1$. Insgesamt trägt dann jeder Punkt auf dem Gitter mit einem Faktor $-1$ bei, sodass sich das gesamte Vorzeichen für jeden Graphen bei der Spurbildung unter Berücksichtigung der Koeffizienten zu $(-1)^N$ ergibt.
