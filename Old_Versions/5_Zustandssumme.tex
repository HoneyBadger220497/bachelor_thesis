
Da die Zustandssumme in der Statistischen Physik eine zentrale Rolle spielt und ein Teil dieser Rechnung später ohne hin benötigt wird, soll in diesem Abschnitt ein analytischer Ausdruck für die Zustandssumme des 2D - Ising-Modells angegeben werden.
Ausgangspunkt ist die Darstellung \eqref{eq: ZustandsBerezinIntegral} der Zustandssummer als Berezin-Integral.
Die antisymmetrische Darstellende Matrix $\bm A$ der Graßmann Wirkung $A$ ist eine komplexe $4N \times 4N$ Matrix. Um die Notation übersichtlich zu halten bezeichnet $\bm{\eta}$ den Vektor der Graßmann Variablen an den Gitterpunkten. Die Reihenfolge der Variablen im Vektor ist die im vorherigen Kapitel in \eqref{eq: ZustandsBerezinIntegral} vereinbarte Reihenfolge der Graßmann Variablen.
\begin{equation}
\bm{\eta} = \left(h_{\bm{x}_1}^o, h_{\bm{x}_1}^x, v_{\bm{x}_1}^o, v_{\bm{x}_1}^x, \dots, h_{\bm{x}_N}^o, h_{\bm{x}_N}^x, v_{\bm{x}_N}^o, v_{\bm{x}_N}^x \right)
\end{equation}

 \noindent Dabei ist die Art wie die Gitterpunkte durchnummeriert werden grundsätzlich egal, da Paare von Graßmann Variablen miteinander kommutieren. Um die Darstellenden Matrix jedoch möglichst einfach zu gestalten, soll die Numerierung der Gitterpunkt nun wie in Abbildung \ref{Abb: Numerierung} festgelegt werden. Diese Nummerierung besitzt die Eigenschaft, dass ein Verschiebung der Indizierung um $2M(M+1)$ durch eine Punktspiegelung erreicht wird. 

\begin{figure}[h!]
\centering
\begin{tikzpicture}[scale = 1.2]

\begin{scope}[shift = {(-3,0)}]


\node[small_grid_point] at (0,0) (0){}; 

% right grid point
\node[small_grid_point_right] at (0,1)  {}; 
\node[small_grid_point_right] at (0,2) {}; 
\node[small_grid_point_right] at (1,-2) {}; 
\node[small_grid_point_right] at (1,-1) {}; 
\node[small_grid_point_right] at (1,0) (10){}; 
\node[small_grid_point_right] at (1,1) (1) {}; 
\node[small_grid_point_right] at (1,2) {}; 
\node[small_grid_point_right] at (2,-2) {}; 
\node[small_grid_point_right] at (2,-1) (3){}; 
\node[small_grid_point_right] at (2,0) (20){}; 
\node[small_grid_point_right] at (2,1) {}; 
\node[small_grid_point_right] at (2,2) {}; 

%%% left grid
\node[small_grid_point_left] at (0,-1) {}; 
\node[small_grid_point_left] at (0,-2) {}; 
\node[small_grid_point_left] at (-1,-2) {}; 
\node[small_grid_point_left] at (-1,-1) (2){}; 
\node[small_grid_point_left] at (-1,0) {}; 
\node[small_grid_point_left] at (-1,1) {}; 
\node[small_grid_point_left] at (-1,2) {}; 
\node[small_grid_point_left] at (-2,-2) {}; 
\node[small_grid_point_left] at (-2,-1) {}; 
\node[small_grid_point_left] at (-2,0) {}; 
\node[small_grid_point_left] at (-2,1) (4){}; 
\node[small_grid_point_left] at (-2,2) {}; 


\node[draw = none, scale = 1] at (0.3,0.3) {0};
%%% right grid
\node[draw = none, scale = 1] at (0.3,1.3) {1};
\node[draw = none, scale = 1] at (0.3,2.3) {2};
\node[draw = none, scale = 1] at (1.3,-1.7) {3};
\node[draw = none, scale = 1] at (1.3,-0.7) {4};
\node[draw = none, scale = 1] at (1.3,0.3) {5};
\node[draw = none, scale = 1] at (1.3,1.3) {6};
\node[draw = none, scale = 1] at (1.3,2.3) {7};
\node[draw = none, scale = 1] at (2.3,-1.7) {8};
\node[draw = none, scale = 1] at (2.3,-0.7) {9};
\node[draw = none, scale = 1] at (2.3,0.3) {10};
\node[draw = none, scale = 1] at (2.3,1.3) {11};
\node[draw = none, scale = 1] at (2.3,2.3) {12};
%%% left grid
\node[draw = none, scale = 1] at (0.3,-0.7) {13};
\node[draw = none, scale = 1] at (0.3,-1.7) {14};
\node[draw = none, scale = 1] at (-0.7,-1.7) {19};
\node[draw = none, scale = 1] at (-0.7,-0.7) {18};
\node[draw = none, scale = 1] at (-0.7,0.3) {17};
\node[draw = none, scale = 1] at (-0.7,1.3) {16};
\node[draw = none, scale = 1] at (-0.7,2.3) {15};
\node[draw = none, scale = 1] at (-1.7,-1.7) {24};
\node[draw = none, scale = 1] at (-1.7,-0.7) {23};
\node[draw = none, scale = 1] at (-1.7,0.3) {22}; 
\node[draw = none, scale = 1] at (-1.7,1.3) {21};
\node[draw = none, scale = 1] at (-1.7,2.3) {20};

\end{scope}

\begin{scope}[shift = {(3,0)}]


\node[small_grid_point] at (0,0) (0){}; 

% right grid point
\node[small_grid_point_right] at (0,1)  {}; 
\node[small_grid_point_right] at (0,2) {}; 
\node[small_grid_point_right] at (1,-2) {}; 
\node[small_grid_point_right] at (1,-1) {}; 
\node[small_grid_point_right] at (1,0) (10){}; 
\node[small_grid_point_right] at (1,1) (1) {}; 
\node[small_grid_point_right] at (1,2) {}; 
\node[small_grid_point_right] at (2,-2) {}; 
\node[small_grid_point_right] at (2,-1) (3){}; 
\node[small_grid_point_right] at (2,0) (20){}; 
\node[small_grid_point_right] at (2,1) {}; 
\node[small_grid_point_right] at (2,2) {}; 

%%% left grid
\node[small_grid_point_left] at (0,-1) {}; 
\node[small_grid_point_left] at (0,-2) {}; 
\node[small_grid_point_left] at (-1,-2) {}; 
\node[small_grid_point_left] at (-1,-1) (2){}; 
\node[small_grid_point_left] at (-1,0) {}; 
\node[small_grid_point_left] at (-1,1) {}; 
\node[small_grid_point_left] at (-1,2) {}; 
\node[small_grid_point_left] at (-2,-2) {}; 
\node[small_grid_point_left] at (-2,-1) {}; 
\node[small_grid_point_left] at (-2,0) {}; 
\node[small_grid_point_left] at (-2,1) (4){}; 
\node[small_grid_point_left] at (-2,2) {}; 

%%% arrows from 6 to 18 and 9 to 21 
\draw[arrow_grid_out] (1) -- (2);
\draw[arrow_grid_out] (3) -- (4);

%%% right grid number
\node[draw = none, scale = 1] at (1.3,1.3) {6};
\node[draw = none, scale = 1] at (2.3,-0.7) {9};

%%% left grid number
\node[draw = none, scale = 1] at (-1.3,-0.7) {18};
\node[draw = none, scale = 1] at (-1.7,1.3) {21};


\end{scope}

\end{tikzpicture}
\caption{Nummerierung eines Gitters für $M=5$ und Veranschaulichung der Punktspiegelung}
\label{Abb: Numerierung}
\end{figure}

\subsection{Fouriertransformation der Graßmann-Wirkung}

\noindent Die Darstellende Matrix $\bm A$ der Graßmann Wirkung $A$ ist im allgemeinen voll besetzt. Dies macht es schwierig diese aufzustellen beziehungsweise die Pfaffsche Determinante zu berechnen. Da Periodische Randbedingungen angenommen wurden, ist zu erwarten dass eine diskrete Fourier Transformation der Darstellenden Matrix $\bm A$ in einer Art Block-Diagonalmatrix $\bm{\hat{A}}$ resultiert. Um diese Transformation durchzuführen werden alle Variablen mit gleicher Ausrichtung und gleichem Flavor zu einer Familie von Variablen, gekennzeichnet durch den Index $\nu \in \{(h,x), (h,o), (v,x), (v,o)\}$, zusammengefasst. Somit bezeichnet zum Beispiel $(\nu_i^{(v,x)})_{i \in \Lambda} = (v^x_i)_{i \in \Lambda}$ die Familie aller vertikal ausgerichteten Graßmann Variablen mit Flavor $x$ auf dem Gitter $\Lambda$. Es gibt dann insgesamt vier solcher Familien. Die Diskrete Fouriertransformation wird nun für jede Familie von Graßmann-Variablen getrennt durchgeführt. Dabei bezeichnet $\bm{x} = (x_1, x_2)$ die expliziten Koordinaten auf dem Gitter und $\bm{k} = (k_1,k_2)$ die Koordinaten im reziproken Raum welche innerhalb der ersten Brillouin Zone $\bar{\Lambda}$ liegen. 
\begin{equation}
\bar{\Lambda} = \left\{ \frac{2\pi}{2M+1}(q1, q2) \;\; |\;\; q_1,q_2 \in \{-M,\dots,M\} \right\} 
\end{equation}

\noindent Die Transformationsvorschrift ist in \eqref{eq: Fourer2D invers} und \eqref{eq: Fourer2D} angegeben. Da es sich bei der Fouriertransformation um eine unitäre, und damit invertierbare lineare Transformation handelt, bilden die fouriertransformierten Variablen einen äquivalenten Satz von Graßmann Variablen. 
\begin{align}
\eta_{\bm{x}}^{\nu} &= \frac{1}{\sqrt{N}} \sum_{\bm{k} \in \bar{\Lambda}} \hat{\eta}_{\bm{k}}^{\nu}\; e^{-i \bm{k} \cdot \bm{x}}  \label{eq: Fourer2D invers}\\
\hat{\eta}_{\bm{k}}^{\nu} &= \frac{1}{\sqrt{N}} \sum_{\bm{x} \in \Lambda} \hat{\eta}_{\bm{x}}^{\nu}\; e^{\,i \bm{k} \cdot \bm{x}} \label{eq: Fourer2D}
\end{align}

\noindent Bezeichnet $\bm{\hat{\eta}}^{\nu}$ den Vektor der fouriertransformierten Variablen einer Familie so kann diese Transformation auch mithilfe einer Matrix $\bm{W}$ geschrieben werden.
\begin{align}
\bm{\hat{\eta}}^{\nu} &= \bm{W} \bm{\eta} ^{\nu} \\
\bm{\eta}^{\nu} &= \bm{W}^{\dagger} \bm{\hat{\eta}}^{\nu}
\end{align}

\noindent Die Nummerierung der Gitterpunkte im K-Raum ist letztlich ebenfalls maßgeblich für die Form der Darstellenden Matrizen der Fouriertransformation und der transformieten Wirkung $\hat{A}$. Da das Gitter quadratisch ist, ist auch das reziproke Gitter quadratisch. Die Nummerierung in Punkte innerhalb der Brillouin Zone soll daher genau wie auf dem ursprünglichen Gitter erfolgen.

\noindent Als nächstes soll der Vorfaktor berechnet werden der durch diese lineare Transformation entsteht. Das Integrationsmaß für die fouriertransformierten Graßmann Variablen ist in \eqref{def: fouriertransformiertes Graßmann Maß} angegeben.
\begin{equation} \label{def: fouriertransformiertes Graßmann Maß}
\hat{D}_{h,v} = \prod_{i = 1}^N d\hat{h}_{i}^x\,d\hat{h}_{i}^o\,d\hat{v}_{i}^x\,d\hat{v}_{i}^o
\end{equation}

\noindent Um den Vorfaktor für diese Transformation zu bestimmen, muss beachtet werden, dass jede Familie separat transformiert wird. Die Reihenfolge der Variablen der Algebra muss dazu umgeordnet werden. Es sei $\bm{P}$ die Darstellende Matrix der Permutation, welche dieser Umordnung entspricht. Dann ergibt sich für die Fouriertransformation die in angegebene Vorschrift.
\begin{equation}
\bm{\eta} = \bm{P}^{-1} \left(\begin{array}{cccc} 
               \bm{W}^{\dagger}  &0&0&0 \\
        0&     \bm{W}^{\dagger}  &0&0 \\
        0&0&   \bm{W}^{\dagger}  &0 \\
        0&0&0& \bm{W}^{\dagger}   \\
    \end{array}\right) 
    \bm{P} \, \bm{\hat{\eta}}
\end{equation}

\noindent Mithilfe der Substitutionsformel \eqref{} für Berezin Integrale ergibt sich dann Zusammenhang \eqref{eq: FT Integral wd} für das Integral der Fouriertransformierten $hat{f}(\bm{\hat{\eta}}) = f(\bm{\eta}(\bm{\hat{\eta}}))$ einer Graßmann-Funktion. 
\begin{equation} \label{eq: FT Integral wd}
 \int D_{h,v} \, f  = det(\bm{P})\,det(\bm{W})^4\, det(\bm{P}^{-1}) \int \hat{D}_{h,v} \, \hat{f}  =  det(\bm{W})^4 \int \hat{D}_{h,v} \, \hat{f}
\end{equation} 

\noindent Im Anhang wird gezeigt dass $det(\bm{W}) = \pm 1$. Daher stimmen die Integrale sogar überein.  
\begin{equation} \label{eq: FT Integral}
 \int D_{h,v} \, f  = \int \hat{D}_{h,v} \, \hat{f}
\end{equation}

\noindent Als nächsten Schritt muss eine darstellende Matrix für die fouriertransformierte Wirkung gefunden werden. Dazu wird einfach der Ausdruck  \eqref{} in \eqref{} eingesetzt. Dies sei zunächst etwas allgemeiner vorgeführt.

\begin{align}
\sum_{j = 0}^{N-1} \; \eta^{\nu}_{\bm{x}_j} \,\eta^{\nu'}_{\bm{x}_j+\bm{b}}
& = \sum_{\bm{x}_j \in \Lambda} \; \frac{1}{\sqrt{N}} \sum_{\bm{k} \in \bar{\Lambda}} \hat{\eta}_{\bm{k}}^{\nu}\; e^{-i \bm{k} \cdot \bm{x}_j} \frac{1}{\sqrt{N}} \sum_{\bm{k'} \in \bar{\Lambda}} \hat{\eta}_{\bm{k'}}^{\nu'}\; e^{-i  \bm{k'} \cdot (\bm{x}_j+\bm{b})} \nonumber \\
&  = \sum_{\bm{k} \in \bar{\Lambda}} \sum_{\bm{k'} \in \bar{\Lambda}} e^{-i  \bm{k'} \cdot \bm{b}} \, \hat{\eta}_{\bm{k}}^{\nu}\, \hat{\eta}_{\bm{k'}}^{\nu'} \, \frac{1}{N} \sum_{\bm{x}_j \in \Lambda} e^{-i (\bm{k} + \bm{k'})\cdot \bm{x}_j } \nonumber \\ 
&  = \sum_{\bm{k} \in \bar{\Lambda}} \sum_{\bm{k'} \in \bar{\Lambda}} e^{-i  \bm{k'} \cdot \bm{b}} \, \hat{\eta}_{\bm{k}}^{\nu} \, \hat{\eta}_{\bm{k'}}^{\nu'} \; \delta(\bm{k} + \bm{k'}) \nonumber \\
& = \sum_{\bm{k} \in \bar{\Lambda}}  e^{\,i  \bm{k} \cdot \bm{b}} \,\hat{\eta}_{\bm{k}}^{\nu}\, \hat{\eta}_{-\bm{k}}^{\nu'} \nonumber
\end{align}

\noindent Entsprechende Substitutionen für $\bm{b}$, $\eta^{\nu}$ und $\eta^{\nu'}$ ergeben alle Terme der Transformierten Graßmann-Wirkung $\hat{A}$.
\begin{alignat}{2}
        & \hat{A}   &&= \hat{A}_{bond} + \hat{A}_{corner} + \hat{A}_{monomer} \nonumber \\
        \nonumber \\
        &\hat{A}_{bond} &&= \;\;\; \sum_{\bm{k} \in \bar{\Lambda}}  
            t e^{\,i k_1} \hat{h}_{\bm{k}}^{x} \, \hat{h}_{-\bm{k}}^{o} 
            + t e^{\,i k_2} \hat{v}_{\bm{k}}^{x} \, \hat{h}_{-\bm{k}}^{o}  \nonumber\\
       & \hat{A}_{corner} &&= -\sum_{\bm{k} \in \bar{\Lambda}} 
            h_{\bm{k}}^x \,v_{-\bm{k}}^o 
            + v_{\bm{k}}^x\, h_{-\bm{k}}^o
            + v_{\bm{k}}^x \,h_{-\bm{k}}^x 
            + v_{\bm{k}}^o \,h_{-\bm{k}}^o \nonumber\\
       &  \hat{A}_{monomer}&&= -\sum_{\bm{k} \in \bar{\Lambda}} 
            \, h_{\bm{k}}^o \,h_{-\bm{k}}^x
            +  v_{\bm{k}}^o \,v_{-\bm{k}}^x \nonumber
\end{alignat}

\noindent Der Monomer-Anteil $\hat{A}_{monomer}$ kann, unter Beachtung der Antikommutierende Eigenschaften und einer Umindizierung $\bm{k'} \rightarrow -\bm{k}$, umgeschrieben werden.
\begin{equation}
-\sum_{\bm{k} \in \bar{\Lambda}} h_{\bm{k}}^o \,h_{-\bm{k}}^x = \sum_{\bm{k} \in \bar{\Lambda}} h_{-\bm{k}}^x \, h_{\bm{k}}^o  = 
\sum_{\bm{k'} \in \bar{\Lambda}} h_{\bm{k'}}^x \, h_{-\bm{k'}}^o  
\end{equation}

\noindent Daher können die Terme von $A_{bond}$ und $\hat{A}_{monomer}$ unter Einführung der Koeffizienten $\xi(k)$ zusammengefasst werden. 
\begin{equation} 
\xi(k) = 1 + t\,e^{\,ik} 
\end{equation}

\noindent Durch Verdoppelung und anschließender Umindizierung $\bm{k'} \rightarrow -\bm{k}$ aller Terme, kann eine Antisymmetrische Darstellung der Wirkung, wie in \eqref{} angegeben, gefunden werden.

\begin{align}
    2\,\hat{A}  
        &= \sum_{\bm{k} \in \bar{\Lambda}}  
        \xi(k_1) \; \hat{h}_{\bm{k}}^{x} \,\hat{h}_{-\bm{k}}^{o} 
        + \xi(k_2) \; \hat{v}_{\bm{k}}^{x} \,\hat{v}_{-\bm{k}}^{o}  
        - \xi(-k_1) \;  \hat{h}_{\bm{k}}^{o} \,\hat{h}_{-\bm{k}}^{x}
        - \xi(-k_2) \; \hat{v}_{\bm{k}}^{o} \,\hat{v}_{-\bm{k}}^{x}   \nonumber\\
        & + \sum_{\bm{k} \in \bar{\Lambda}}  
        - h_{\bm{k}}^x \,v_{-\bm{k}}^o 
        - v_{\bm{k}}^x\, h_{-\bm{k}}^o
        - v_{\bm{k}}^x \,h_{-\bm{k}}^x 
        - v_{\bm{k}}^o \,h_{-\bm{k}}^o 
        + v_{\bm{k}}^o \, h_{-\bm{k}}^x  
        + h_{\bm{k}}^o \, v_{-\bm{k}}^x
        + h_{\bm{k}}^x \, v_{-\bm{k}}^x 
        + h_{\bm{k}}^o \, v_{-\bm{k}}^o  \nonumber 
\end{align}

\noindent Durch Einführung der Vekoren $\bm{\hat{\eta}}_{\bm{k}} = (\hat{h}_{\bm{k}}^{o}, \hat{h}_{\bm{k}}^{x}, \hat{v}_{\bm{k}}^{o}, \hat{v}_{\bm{k}}^{x} )$ und der Antisymmetrischen Matrizen $\bm{\hat{A}}_{\bm{k}}$ lässt sich die Darstellung \eqref{eq: Darstellung Wirkung} erreichen. 
\begin{equation} \label{eq: Darstellung Wirkung}
    \hat{A} = \frac{1}{2} \sum_{\bm{k} \in \bar{\Lambda}} \bm{\hat{\eta}}_{\bm{k}}^T\, \bm{\hat{A}}_{\bm{k}} \bm{\hat{\eta}}_{-\bm{k}}
\end{equation}

\begin{equation}
    \bm{\hat{A}}_{\bm{k}} = \left(\begin{array}{cccc} 
        0         &\xi(k_1)  &  1       & -1        \\
        -\xi(-k_1)&0         &  1       &  1        \\
        -1        &-1        &  0       & \xi(k_2)  \\
        1         &-1        &-\xi(-k_2)&  0        \\
    \end{array}\right) 
\end{equation}

\noindent Der fouriertransformierte Graßmann-Vektor $\bm{\hat{\eta}}$ lässt sich als Zusammenfassung der Vektoren $\bm{\hat{\eta}}_{\bm{k}}$ auffassen.
\begin{equation}
\bm{\hat{\eta}} = \left(\hat{h}_{\bm{k}_1}^o, \hat{h}_{\bm{k}_1}^x, \hat{v}_{\bm{k}_1}^o, \hat{v}_{\bm{k}_1}^x, \dots, \hat{h}_{\bm{k}_N}^o, \hat{h}_{\bm{k}_N}^x, \hat{v}_{\bm{k}_N}^o, \hat{v}_{\bm{k}_N}^x \right) = \left(\bm{\hat{\eta}}_{\bm{k}_0}, \bm{\hat{\eta}}_{\bm{k}_1}, \cdots, \bm{\hat{\eta}}_{\bm{k}_N}  \right)
\end{equation}

\noindent Damit kann man dann aus \eqref{eq: Darstellung Wirkung} die darstellende Matrix $\hat{A}$ der Graßmann-Wirkung $A$ ablesen. 
\begin{grayframe}[frametitle = { Darstellende Matrix der Fouriertransformierten Graßmann-Wikrung $\hat{A}$ }]
\begin{equation}
\renewcommand{\arraystretch}{1.5}
\bm{\hat{A}} = \frac{1}{2}
\left(\begin{array}{c|ccc|ccc}  
\bm{\hat{A}}_{\bm{k}_0}  & \bm{0}   & \cdots    & \bm{0} & \bm{0} & \cdots  & \bm{0} \\ \hline
\bm{0}  & \bm{0}       & \cdots   & \bm{0}    & \bm{\hat{A}}_{\bm{k}_1} & \cdots  & \bm{0} \\
\vdots  &\vdots        & \ddots   & \vdots    & \vdots                  & \ddots  & \vdots \\
\bm{0}  & \bm{0}       & \cdots   & \bm{0} & \bm{0} & \cdots  & \bm{\hat{A}}_{\bm{k}_{2M(M+1)}}  \\ \hline
\bm{0}  & \bm{\hat{A}}_{\bm{k}_{2M(M+1)+1}} & \cdots  & \bm{0} & \bm{0} & \cdots  & \bm{0} \\
\vdots  &\vdots        & \ddots   & \vdots    & \vdots                  & \ddots  & \vdots \\
\bm{0}  & \bm{0}       & \cdots   &\bm{\hat{A}}_{\bm{k}_{N}} & \bm{0} & \cdots  & \bm{0} 
\end{array} \right) 
\end{equation}
\end{grayframe}

\noindent Da $\xi(k) = - \xi(-k)  $ nur für $k = 2\mathbb{N}\pi$ erfüllt ist, kann nur  $\bm{\hat{A}}_{\bm{k}_0}$, mit ${\bm{k}_0} = (0,0)$, antisymmetrisch sein. Für die übrigen $\bm{\hat{A}}_{\bm{k_i}}$ gilt jedoch $-\bm{\hat{A}}_{\bm{k_i}}^T = \bm{\hat{A}}_{-\bm{k_i}}$. Aufgrund der gewählten Nummerierung der Gitterpunkte folgt dann $\bm{\hat{A}}_{-\bm{k}_i} = \bm{\hat{A}}_{\bm{k}_{i+2M(M+1)}}$. 
Daher ist die Matrix $\hat{\bm{A}}$ insgesamt antisymmetrisch und somit die eindeutig bezüglich dieses Satzes von Graßmann-Variablen. 

\noindent Da das Berezin-Integral der fouriertransformierten Graßmann-Dichte nach \eqref{eq: FT Integral} gleich dem ursprünglichen Spurintegral ist, folgt mithilfe des Satzes über Gauß-Berezin-Integrale \eqref{Satz: Gauß-Berezin 2} und den Eigenschaften \eqref{eq:pfaff 4} und \eqref{eq:pfaff 6} Pfaffscher Determinanten der Ausdruck \eqref{eq: expl. Spur}.
\begin{align}
\Sp{e^{A}}
    &= \Sp{e^{\bm{\hat{\eta}}^T \bm{\hat{A}} \,\bm{\hat{\eta}}}} 
    = pf(2\bm{\hat{A}}) \nonumber \\
    &= pf(\bm{\hat{A}}_{\bm{k}_0}) (-1)^{4\frac{N-1}{2} \left(4\frac{N-1}{2}-1 \right) \, \frac{1}{2}}
        \; det\left(\begin{array}{ccc}  
        \bm{\hat{A}}_{\bm{k}_1} & \cdots  & \bm{0} \\
        \vdots                  & \ddots  & \vdots \\
        \bm{0}                  & \cdots  & \bm{\hat{A}}_{\bm{k}_{2M(M+1)}} 
        \end{array} \right)  \nonumber \\
    &= pf(\bm{\hat{A}}_{\bm{k}_0}) \prod_{i = 1}^{2M(M+1)}  det(\bm{\hat{A}}_{\bm{k}_i}) \label{eq: expl. Spur}
\end{align}

\subsection{Berechnung der Zustandssumme} 

Für die explizite Angabe der Zustandssumme müssen letztlich die $4\times4$ Determinanten und pfaffschen Determinanten ausgewertet werden. Für $ \bm{k} \in \{ \bm{k}_1,\dots, \bm{k}_{2M(M+k_1)} \} $ folgt aus dem Laplace'schen Entwicklungssatz
\begin{equation} \nonumber
det(\bm{\hat{A}}_{\bm{k}})
    = -(\xi(k_1)+\xi(-k_1))(\xi(k_2)+\xi(k_2)) +  \xi(k_1)\xi(-k_1)\xi(k_2)\xi(-k_2) + 4 
\end{equation} Mit den Zwischenergebnissen
\begin{align} \nonumber
\xi(k)+\xi(-k) &= 2(1 + t\,cos(k)) \nonumber \\
\xi(k)\xi(-k) &= 1 + 2t\,cos(k) + t^2 \nonumber
\end{align} folgt dann, nach einigen algebraischen Umformungen
\begin{equation} \label{eq: expl. det(A_k)}
det(\bm{\hat{A}}_{\bm{k}})
    = (1+t^2)^2 - 2t(1-t^2)\,(cos(k_x) + cos(k_y))
\end{equation}

\noindent Mit \eqref{eq: expl. pfaffian} folgt für $\bm{k_0} = (0,0)$ dass
\begin{equation}
pf(\bm{\hat{A}}_{\bm{k}_0}) = t^2 + 2t -1
\end{equation}

\noindent Die Terme \eqref{eq: expl. det(A_k)} sind immer positiv. Denn es gilt
\begin{align}
&(1+t^2)^2 - 2t(1-t^2)\,(cos(k_x) + cos(k_y)) \nonumber \\
    &\geq (1+t^2)^2 - 4t(1-t^2) \nonumber \\
    &= t^4 + 4t^3 + 2t^2 -4t +1 \nonumber \\
    &= (t^2+2t-1)^2 \label{eq: pf(A_0)^2 = det(A_0)} \\
    &\geq 0 \nonumber
\end{align}

\noindent Mithilfe von $-\bm{\hat{A}}_{\bm{k_i}}^T = \bm{\hat{A}}_{-\bm{k_i}}  = \bm{\hat{A}}_{\bm{k}_{i+2M(M+1)}}$ und der Tatsache, dass es sich um $4\times4$ Matrizen handelt, kann man Ausdruck \eqref{eq: expl. Spur} symmetrisieren.
\begin{equation} \label{eq: sym. det(A_k)}
det(\bm{\hat{A}}_{\bm{k_i}})
    = \sqrt{det(\bm{\hat{A}}_{\bm{k}_i})^2} 
    = \sqrt{det(\bm{\hat{A}}_{\bm{k}_i})det(\bm{\hat{A}}_{\bm{k}_{i+2M(M+1)}})} 
\end{equation}

\noindent Unter Berücksichtigung der verbliebenen Vorfaktoren \eqref{} und \eqref{eq: expl. Spur} erhält  man dann \eqref{eq: expl. Zustandssumme} für die Zustandssumme.
\begin{align}
Z   &= cosh(\beta J)^{2N} 2^N (-1)^N \Sp{e^A} \nonumber = (2cosh(\beta J)^{2} )^N (-1)^N pf(\bm{\hat{A}}_{\bm{k}_0}) \prod_{\bm{k} \in \bar{\Lambda}\setminus{\bm{0}}}  \sqrt{det(\bm{\hat{A}}_{\bm{k}_i})} \nonumber\\
    &= (2cosh(\beta J)^{2} )^N (-1)^N (t^2+2t-1) \prod_{\bm{k} \in \bar{\Lambda}\setminus{\bm{0}}}  \sqrt{(1+t^2)^2 - 2t(1-t^2)\,(cos(k_1) + cos(k_2))} \nonumber\\
    &= (2cosh(\beta J)^{2} )^N (1-t^2+2t) \prod_{\bm{k} \in \bar{\Lambda}\setminus{\bm{0}}}  \sqrt{(1+t^2)^2 - 2t(1-t^2)\,(cos(k_1) + cos(k_2))} \label{eq: expl. Zustandssumme}
\end{align}

\subsection{Diskussion des Ergebnisses für die Zustandssumme}
\subsection{Übergang Thermodynamischer Limes}