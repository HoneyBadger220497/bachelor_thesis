
Im Folgenden werden zunächst nur die in Abschnitt \ref{sec: GraßmanGraphs} eingeführten Graßmann-Paare $P_{i,\alpha}$ betrachtet. Die Koeffizienten werden außer acht gelassen und sollen am Ende so gewählt werden, dass die Spur für jedes Produkte von Graßmann-Paaren das gleiche Vorzeichen ergibt. Für die Graphische Bedeutung der Graßmann-Paare sei an Abb. \ref{Abb: Graphische Interpretation Graßmann Paare} errinert. 
 
\subsection{Vorzeichen der Monomer-Paare} \label{sec: vorzeicehnMonomer}

Für die Gitterpunkte die zu keinem Graphen gehören gibt es, wie in Abschnitt \ref{sec: GraßmanGraphs} beschrieben, drei mögliche Darstellungen. Somit kann dreimal der selbe Graph in der Summe \eqref{eq: exp_sum_combi} auftauchen, wodurch sich der Beitrag zum Vorzeichen des Graphen aus der Summe der drei einzelnen Vorzeichen ergibt. Die Monomer-Paare $P_{i,e}$ und $P_{i,f}$ sind bereits Richtig angeordnet. 

\begin{align}
\iint \,dh_{i}^x\,dh_{i}^o \; P_{i,e} &= \iint \,dh_{i}^x\,dh_{i}^o h_{i}^o\,h_{i}^x\;  = 1\\
\iint \,dv_{i}^x\,dv_{i}^o \; P_{i,f} &= \iint \,dv_{i}^x\,dv_{i}^o v_{i}^o\,v_{i}^x\;  = 1
\end{align} Daher ergibt sich für die 3 Darstellungen der Monomer

\begin{equation}
-\underbrace{h_{i}^x\, v_{i}^o \,v_{i}^x\,h_{i}^o}_{= P_{i,a}\cdot P_{i,b}} 
= h_{i}^x\, v_{i}^o\, h_{i}^o\,\,v_{i}^x 
= \underbrace{h_{i}^o\,h_{i}^x\, v_{i}^o\,v_{i}^x }_{= P_{i,e}\cdot P_{i,f}}= - h_{i}^o\,v_{i}^o\,h_{i}^x\,v_{i}^x 
= -\underbrace{v_{i}^o\,h_{i}^o\,v_{i}^x\,h_{i}^x}_{= P_{i,c}\cdot P_{i,d}}
\end{equation} Somit trägt jeder Gitterpunkt, der nicht teil eines Graphen ist, mit einem Faktor -1 zum Vorzeichen bei.

\subsection{Vorzeichen eines zusammenhängenden Graphen ohne Selbstüberschneidung}

Um das Vorzeichen geschlossener Graphen auswerten zu können, muss man die Graßmann Variablen derart umordnen, sodass die Integration am Ende 1 ergibt. Das Vorzeichen ergibt sich aus den notwendigen Vertauschungen. Hierzu soll ein graphisches Vorgehen die Rechnung ersetzen. Um das Vorzeichen bestimmen zu können, müssen die Paare für die Ecken $P_a$ bis $P_d$ und Verbindungen $P_g$, $P_h$ betrachtet werden. Die einzelnen Monomer-Paare sind jeweils schon richtig geordnet und können weggelassen werden. Zu Bestimmung des Vorzeichen eines geschlossenen Graphen geht man dann in folgenden Schritten vor:

\begin{itemize}
\item[0)] Man wählt ein Paar vom Typ $g$ oder $h$ als das Erste. Graphisch betrachtet, wählt man so auf dem Graphen einen Seite eines Gitterknotens als Startpunkt und geht in eine Richtung zum nächsten Gitterknoten. Diese Richtung legt die Durchlaufrichtung, des Graphen fest. 
\item[2)] Man wählt ein Paar als Nachfolger, welches eine zur zweiten Variable des Vorgängerpaares konjugierte Variable enthält. Die Konjungiertheit in diesem kontext wurde in Abschnitt \ref{sec: GraßmanGraphs} eingeführt. Damit ist das Nachfolgerpaar eindeutig festgelegt. Anschließend werden die beiden Variablen im Nachfolger Paar vertauscht, falls die konjungierte Variable nicht die erste variable ist. Nun kann ein weiterer Nachfolger bestimmt werden. Graphisch erfolgt ein Vorzeichen wechsel für jeden Pfeil im Graphen der entgegen der Durchlaufrichtung liegt.
Der Algorithmus wird fortgesetzt bis das nächste Paar, das erste wäre. Ist dies nie der Fall, ist der Graph nicht geschlossen und somit nicht relevant.
\item[3)] Nun wird die letzte Variable an den Anfang gehängt und die Paare um geklammert. Da die erste Variable eine mit $x$ Flavor ist, muss im nächsten schritt eine zusätzliche Vertauschung vorgenommen werden und es ergibt sich ein zusätzlicher Faktor -1. Dieses Vorzeichen geht auf die Schließung des Graphen zurück.
\item[4)] Durch Umklammern erhält man lauter Paare konjungierter Variablen. Wenn in jedem Paar $o$ vor $x$ kommt, ist das Integral positiv. Somit müssen Paare, wo dies nicht gilt, vertauscht werden. Graphisch fast man dafür bei jedem Knoten die Ports gleicher Orientierung zusammen, wie im Abb.\ref{Abb: VorzeichenBestimmung} mit roten Ellipsen gekennzeichnet.  Kommt hier in Durchlaufrichtung ein $x$ vor einem $o$ führt dies zu einem Vorzeichen wechsel.  
\end{itemize}


\begin{figure}
    \centering
    \captionsetup[subfigure]{labelformat=empty}
    \begin{subfigure}[c]{0.4\textwidth}
        \centering
        \begin{tikzpicture}[node distance=0.1, scale = 2.0]
    \draw[step=1cm,gray, ultra thin] (-1.5,-1.5) grid (1.4,1.5);
    
    %% 3x3 Grid
% gridpoint 1 = (0,0) 
\node[draw = none] at (0,0) (1center) {} ;
\node[draw, circle, fill=none, scale = 0.5, very thick] (1ho) [left=of 1center]  {} ;
\node[draw, circle, fill=none, scale = 0.5, very thick] (1vo) [below=of 1center]  {} ;
\node[draw, cross out, very thick, scale = 0.6] (1hx) [right=of 1center] {} ;
\node[draw, cross out, very thick, scale = 0.6] (1vx) [above=of 1center] {} ;

% gridpoint 2 = (1,0) 
\node[draw = none] at (1,0) (2center) {} ;
\node[draw, circle, fill=none, scale = 0.5, very thick] (2ho) [left=of 2center]  {} ;
\node[draw, circle, fill=none, scale = 0.5, very thick] (2vo) [below=of 2center]  {} ;
\node[draw, cross out, very thick, scale = 0.6] (2hx) [right=of 2center] {} ;
\node[draw, cross out, very thick, scale = 0.6] (2vx) [above=of 2center] {} ;
    
% gridpoint 3 = (1,1) 
\node[draw = none] at (1,1) (3center) {} ;
\node[draw, circle, fill=none, scale = 0.5, very thick] (3ho) [left=of 3center]  {} ;
\node[draw, circle, fill=none, scale = 0.5, very thick] (3vo) [below=of 3center]  {} ;
\node[draw, cross out, very thick, scale = 0.6] (3hx) [right=of 3center] {} ;
\node[draw, cross out, very thick, scale = 0.6] (3vx) [above=of 3center] {} ;

% gridpoint 4 = (0,1) 
\node[draw = none] at (0,1) (4center) {} ;
\node[draw, circle, fill=none, scale = 0.5, very thick] (4ho) [left=of 4center]  {} ;
\node[draw, circle, fill=none, scale = 0.5, very thick] (4vo) [below=of 4center]  {} ;
\node[draw, cross out, very thick, scale = 0.6] (4hx) [right=of 4center] {} ;
\node[draw, cross out, very thick, scale = 0.6] (4vx) [above=of 4center] {} ;

% gridpoint 5 = (-1,1) 
\node[draw = none] at (-1,1) (5center) {} ;
\node[draw, circle, fill=none, scale = 0.5, very thick] (5ho) [left=of 5center]  {} ;
\node[draw, circle, fill=none, scale = 0.5, very thick] (5vo) [below=of 5center]  {} ;
\node[draw, cross out, very thick, scale = 0.6] (5hx) [right=of 5center] {} ;
\node[draw, cross out, very thick, scale = 0.6] (5vx) [above=of 5center] {} ;

% gridpoint 6 = (-1,0) 
\node[draw = none] at (-1,0) (6center) {} ;
\node[draw, circle, fill=none, scale = 0.5, very thick] (6ho) [left=of 6center]  {} ;
\node[draw, circle, fill=none, scale = 0.5, very thick] (6vo) [below=of 6center]  {} ;
\node[draw, cross out, very thick, scale = 0.6] (6hx) [right=of 6center] {} ;
\node[draw, cross out, very thick, scale = 0.6] (6vx) [above=of 6center] {} ;
    
% gridpoint 7 = (-1,-1) 
\node[draw = none] at (-1,-1) (7center) {} ;
\node[draw, circle, fill=none, scale = 0.5, very thick] (7ho) [left=of 7center]  {} ;
\node[draw, circle, fill=none, scale = 0.5, very thick] (7vo) [below=of 7center]  {} ;
\node[draw, cross out, very thick, scale = 0.6] (7hx) [right=of 7center] {} ;
\node[draw, cross out, very thick, scale = 0.6] (7vx) [above=of 7center] {} ;
    
% gridpoint 8 = (0,-1) 
\node[draw = none] at (0,-1) (8center) {} ;
\node[draw, circle, fill=none, scale = 0.5, very thick] (8ho) [left=of 8center]  {} ;
\node[draw, circle, fill=none, scale = 0.5, very thick] (8vo) [below=of 8center]  {} ;
\node[draw, cross out, very thick, scale = 0.6] (8hx) [right=of 8center] {} ;
\node[draw, cross out, very thick, scale = 0.6] (8vx) [above=of 8center] {} ;    
    
% gridpoint 9 = (1,-1) 
\node[draw = none] at (1,-1) (9center) {} ;
\node[draw, circle, fill=none, scale = 0.5, very thick] (9ho) [left=of 9center]  {} ;
\node[draw, circle, fill=none, scale = 0.5, very thick] (9vo) [below=of 9center]  {} ;
\node[draw, cross out, very thick, scale = 0.6] (9hx) [right=of 9center] {} ;
\node[draw, cross out, very thick, scale = 0.6] (9vx) [above=of 9center] {} ;

%% graph
% connections
\draw[arrow_outer] (7hx) -- (8ho);
\draw[arrow_outer] (8hx) -- (9ho);
\draw[arrow_outer] (9vx) -- (2vo);
\draw[arrow_outer] (2vx) -- (3vo);
\draw[arrow_outer] (4hx) -- (3ho);
\draw[arrow_outer] (5hx) -- (4ho);
\draw[arrow_outer] (6vx) -- (5vo);
\draw[arrow_outer] (7vx) -- (6vo);
% corners
%\draw[->, blue!80, very thick] (1.3, -1.1)  arc[radius=0.2, start angle=0, end angle= -90];
\draw[arrow_outer] (9hx) .. controls (1.4, -1.25) and (1.25, -1.4)    .. (9vo); % corner a) at 9
\draw[arrow_outer] (5vx) .. controls (-1.25, 1.4) and (-1.4, 1.25)    .. (5ho); % corner b) at 5
\draw[arrow_outer] (3vx) .. controls (1.25, 1.4) and (1.4, 1.25)      .. (3hx); % corner c) at 3 
\draw[arrow_outer] (7vo) .. controls (-1.25, -1.4) and (-1.4, -1.25)  .. (7ho); % corner d) at 7
    
%% pairs
% at 2
\draw (1, 0) ellipse[x radius = 0.265, y radius = 0.1, rotate = 90, color = red!100];
% at 3
\draw (1, 1) ellipse[x radius = 0.265, y radius = 0.1, rotate = 90, color = red!100];
\draw (1, 1) ellipse[x radius = 0.265, y radius = 0.1, rotate = 0, color = red!100];
% at 4
\draw (0, 1) ellipse[x radius = 0.265, y radius = 0.1, rotate = 0, color = red!100];
% at 5
\draw (-1, 1) ellipse[x radius = 0.265, y radius = 0.1, rotate = 90, color = red!100];
\draw (-1, 1) ellipse[x radius = 0.265, y radius = 0.1, rotate = 0, color = red!100];
% at 6
\draw (-1, 0) ellipse[x radius = 0.265, y radius = 0.1, rotate = 90, color = red!100];
% at 7
\draw (-1, -1) ellipse[x radius = 0.265, y radius = 0.1, rotate = 90, color = red!100];
\draw (-1, -1) ellipse[x radius = 0.265, y radius = 0.1, rotate = 0, color = red!100];
% at 8
\draw (0, -1) ellipse[x radius = 0.265, y radius = 0.1, rotate = 0, color = red!100];
% at 9 
\draw (1, -1) ellipse[x radius = 0.265, y radius = 0.1, rotate = 90, color = red!100];
\draw (1, -1) ellipse[x radius = 0.265, y radius = 0.1, rotate = 0, color = red!100];

    
    \draw[arrow_start] (-1.5, -0.8) -- (-1.1,-0.8);
    \node[draw = none, scale = 0.8, ] at (-1.6, -0.7) {Start};
\end{tikzpicture}
    \end{subfigure}
    \hspace{0.1\textwidth}
    \begin{subfigure}[c]{0.4\textwidth}
        \subcaption{Um das Vorzeichen des Graphen zu bestimmen wählt man einen Startpunkt, hier ein $x$. Dann bewegt man sich in die Richtung des ersten Pfeiles vom Startpunkt weg durch den Graphen. Dabei bewegt man sich bei den Ellipsen immer entlang der längeren Achse. Für jeden Pfeil der entgegen der Bewegungsrichtung zeigt, sowie immer wenn in einer Ellipse $x$ vor $o$ kommt, erhält man einen Vorzeichenwechsel. Hier erhält man 5 Vorzeichenwechsel für die Pfeilrichtung, 6 Vorzeichenwechsel für die $xo$ Paare und einen zusätzlich für die Schließung des Graphen. Insgesamt also $+1$ als Vorzeichen.  }
    \end{subfigure}
    \caption{Vorzeichenbestimmung geschlossener zusammenhängender Graphen } \label{Abb: VorzeichenBestimmung}
\end{figure}

\noindent Mithilfe dieser graphischen Regel kann nun leicht das Vorzeichen eines beliebigen geschlossenen Graphen bestimmt werden. Die Regel ermöglicht jedoch auch eine Segmentierung eines Graphen, in unterschiedliche Bausteine wie in Abb. \ref{Abb: directedElemets}. Aus der in Abb. \ref{Abb: Segmentierung} gezeigten Segmentierung in entgegengesetzte Richtungen lassen sich 12 Bausteine ermitteln, mit denen jeder geschlossene Graph ohne Selbstüberschneidung gebaut werden kann. Die Vorzeichen dieser Bausteine sind in Abb. \ref{Abb: directedElemets} angegeben. Für $b$) ergibt sich zum Beispiel ein Vorzeichenwechsel für das Zweite $xo$ Paar und den zweiten Pfeil der gegen die Durchlaufrichtung zeigt. Insgesamt also ein positives Vorzeichen. Das Vorzeichen eines geschlossenen Graphen ohne Selbstüberschneidung ergibt sich dann als Produkt der Vorzeichen der Segmente und dem Vorzeichen für die Schließung des Graphen. Für den geschlossenen Graphen in Abb. \ref{Abb: VorzeichenBestimmung} ergibt sich insgesamt ein Positives Vorzeichen. Durch hinzufügen von Elementen $a$) bis $\bar h$) lässt sich jeder beliebige geschlossene, zusammenhängende Graph ohne Selbsüberschneidung erstellen. Dabei kommen die Elemente $a$) bis $d$) immer paarweise mit ihrer komplementären Sequenz $\bar a$) bis $\bar d$) vor. Somit ändert sich das Vorzeichen nicht und alle geschlossenen zusammenhängenden Graphen ohne Selbsüberschneidung haben positives Vorzeichen. 

\begin{figure}
    %\begin{tikzpicture}[node distance=0.1, scale = 2.0]
    \draw[step=1cm,gray, ultra thin] (-1.5,-1.5) grid (1.4,1.5);
    
    %% 3x3 Grid
% gridpoint 1 = (0,0) 
\node[draw = none] at (0,0) (1center) {} ;
\node[draw, circle, fill=none, scale = 0.5, very thick] (1ho) [left=of 1center]  {} ;
\node[draw, circle, fill=none, scale = 0.5, very thick] (1vo) [below=of 1center]  {} ;
\node[draw, cross out, very thick, scale = 0.6] (1hx) [right=of 1center] {} ;
\node[draw, cross out, very thick, scale = 0.6] (1vx) [above=of 1center] {} ;

% gridpoint 2 = (1,0) 
\node[draw = none] at (1,0) (2center) {} ;
\node[draw, circle, fill=none, scale = 0.5, very thick] (2ho) [left=of 2center]  {} ;
\node[draw, circle, fill=none, scale = 0.5, very thick] (2vo) [below=of 2center]  {} ;
\node[draw, cross out, very thick, scale = 0.6] (2hx) [right=of 2center] {} ;
\node[draw, cross out, very thick, scale = 0.6] (2vx) [above=of 2center] {} ;
    
% gridpoint 3 = (1,1) 
\node[draw = none] at (1,1) (3center) {} ;
\node[draw, circle, fill=none, scale = 0.5, very thick] (3ho) [left=of 3center]  {} ;
\node[draw, circle, fill=none, scale = 0.5, very thick] (3vo) [below=of 3center]  {} ;
\node[draw, cross out, very thick, scale = 0.6] (3hx) [right=of 3center] {} ;
\node[draw, cross out, very thick, scale = 0.6] (3vx) [above=of 3center] {} ;

% gridpoint 4 = (0,1) 
\node[draw = none] at (0,1) (4center) {} ;
\node[draw, circle, fill=none, scale = 0.5, very thick] (4ho) [left=of 4center]  {} ;
\node[draw, circle, fill=none, scale = 0.5, very thick] (4vo) [below=of 4center]  {} ;
\node[draw, cross out, very thick, scale = 0.6] (4hx) [right=of 4center] {} ;
\node[draw, cross out, very thick, scale = 0.6] (4vx) [above=of 4center] {} ;

% gridpoint 5 = (-1,1) 
\node[draw = none] at (-1,1) (5center) {} ;
\node[draw, circle, fill=none, scale = 0.5, very thick] (5ho) [left=of 5center]  {} ;
\node[draw, circle, fill=none, scale = 0.5, very thick] (5vo) [below=of 5center]  {} ;
\node[draw, cross out, very thick, scale = 0.6] (5hx) [right=of 5center] {} ;
\node[draw, cross out, very thick, scale = 0.6] (5vx) [above=of 5center] {} ;

% gridpoint 6 = (-1,0) 
\node[draw = none] at (-1,0) (6center) {} ;
\node[draw, circle, fill=none, scale = 0.5, very thick] (6ho) [left=of 6center]  {} ;
\node[draw, circle, fill=none, scale = 0.5, very thick] (6vo) [below=of 6center]  {} ;
\node[draw, cross out, very thick, scale = 0.6] (6hx) [right=of 6center] {} ;
\node[draw, cross out, very thick, scale = 0.6] (6vx) [above=of 6center] {} ;
    
% gridpoint 7 = (-1,-1) 
\node[draw = none] at (-1,-1) (7center) {} ;
\node[draw, circle, fill=none, scale = 0.5, very thick] (7ho) [left=of 7center]  {} ;
\node[draw, circle, fill=none, scale = 0.5, very thick] (7vo) [below=of 7center]  {} ;
\node[draw, cross out, very thick, scale = 0.6] (7hx) [right=of 7center] {} ;
\node[draw, cross out, very thick, scale = 0.6] (7vx) [above=of 7center] {} ;
    
% gridpoint 8 = (0,-1) 
\node[draw = none] at (0,-1) (8center) {} ;
\node[draw, circle, fill=none, scale = 0.5, very thick] (8ho) [left=of 8center]  {} ;
\node[draw, circle, fill=none, scale = 0.5, very thick] (8vo) [below=of 8center]  {} ;
\node[draw, cross out, very thick, scale = 0.6] (8hx) [right=of 8center] {} ;
\node[draw, cross out, very thick, scale = 0.6] (8vx) [above=of 8center] {} ;    
    
% gridpoint 9 = (1,-1) 
\node[draw = none] at (1,-1) (9center) {} ;
\node[draw, circle, fill=none, scale = 0.5, very thick] (9ho) [left=of 9center]  {} ;
\node[draw, circle, fill=none, scale = 0.5, very thick] (9vo) [below=of 9center]  {} ;
\node[draw, cross out, very thick, scale = 0.6] (9hx) [right=of 9center] {} ;
\node[draw, cross out, very thick, scale = 0.6] (9vx) [above=of 9center] {} ;

%% graph
% connections
\draw[arrow_outer] (7hx) -- (8ho);
\draw[arrow_outer] (8hx) -- (9ho);
\draw[arrow_outer] (9vx) -- (2vo);
\draw[arrow_outer] (2vx) -- (3vo);
\draw[arrow_outer] (4hx) -- (3ho);
\draw[arrow_outer] (5hx) -- (4ho);
\draw[arrow_outer] (6vx) -- (5vo);
\draw[arrow_outer] (7vx) -- (6vo);
% corners
%\draw[->, blue!80, very thick] (1.3, -1.1)  arc[radius=0.2, start angle=0, end angle= -90];
\draw[arrow_outer] (9hx) .. controls (1.4, -1.25) and (1.25, -1.4)    .. (9vo); % corner a) at 9
\draw[arrow_outer] (5vx) .. controls (-1.25, 1.4) and (-1.4, 1.25)    .. (5ho); % corner b) at 5
\draw[arrow_outer] (3vx) .. controls (1.25, 1.4) and (1.4, 1.25)      .. (3hx); % corner c) at 3 
\draw[arrow_outer] (7vo) .. controls (-1.25, -1.4) and (-1.4, -1.25)  .. (7ho); % corner d) at 7
    
%% pairs
% at 2
\draw (1, 0) ellipse[x radius = 0.265, y radius = 0.1, rotate = 90, color = red!100];
% at 3
\draw (1, 1) ellipse[x radius = 0.265, y radius = 0.1, rotate = 90, color = red!100];
\draw (1, 1) ellipse[x radius = 0.265, y radius = 0.1, rotate = 0, color = red!100];
% at 4
\draw (0, 1) ellipse[x radius = 0.265, y radius = 0.1, rotate = 0, color = red!100];
% at 5
\draw (-1, 1) ellipse[x radius = 0.265, y radius = 0.1, rotate = 90, color = red!100];
\draw (-1, 1) ellipse[x radius = 0.265, y radius = 0.1, rotate = 0, color = red!100];
% at 6
\draw (-1, 0) ellipse[x radius = 0.265, y radius = 0.1, rotate = 90, color = red!100];
% at 7
\draw (-1, -1) ellipse[x radius = 0.265, y radius = 0.1, rotate = 90, color = red!100];
\draw (-1, -1) ellipse[x radius = 0.265, y radius = 0.1, rotate = 0, color = red!100];
% at 8
\draw (0, -1) ellipse[x radius = 0.265, y radius = 0.1, rotate = 0, color = red!100];
% at 9 
\draw (1, -1) ellipse[x radius = 0.265, y radius = 0.1, rotate = 90, color = red!100];
\draw (1, -1) ellipse[x radius = 0.265, y radius = 0.1, rotate = 0, color = red!100];

    
    \draw[arrow_start] (-1.5, -0.8) -- (-1.1,-0.8);
    \node[draw = none, scale = 0.8, ] at (-1.6, -0.7) {Start};
\end{tikzpicture}
    \begin{subfigure}[c]{0.3\textwidth}
    \begin{tikzpicture}[node distance=0.1, scale = 2.25]
    %\draw[step=1cm,gray, ultra thin] (-1.5,-1.5) grid (1.5,1.5);
    %\draw[step=1cm,gray, ultra thin] ( 1.6,-1.5) grid (4.4,1.5);
    
    %% 3x3 Grid
% gridpoint 1 = (0,0) 
\node[draw = none] at (0,0) (1center) {} ;
\node[draw, circle, fill=none, scale = 0.5, very thick] (1ho) [left=of 1center]  {} ;
\node[draw, circle, fill=none, scale = 0.5, very thick] (1vo) [below=of 1center]  {} ;
\node[draw, cross out, very thick, scale = 0.6] (1hx) [right=of 1center] {} ;
\node[draw, cross out, very thick, scale = 0.6] (1vx) [above=of 1center] {} ;

% gridpoint 2 = (1,0) 
\node[draw = none] at (1,0) (2center) {} ;
\node[draw, circle, fill=none, scale = 0.5, very thick] (2ho) [left=of 2center]  {} ;
\node[draw, circle, fill=none, scale = 0.5, very thick] (2vo) [below=of 2center]  {} ;
\node[draw, cross out, very thick, scale = 0.6] (2hx) [right=of 2center] {} ;
\node[draw, cross out, very thick, scale = 0.6] (2vx) [above=of 2center] {} ;
    
% gridpoint 3 = (1,1) 
\node[draw = none] at (1,1) (3center) {} ;
\node[draw, circle, fill=none, scale = 0.5, very thick] (3ho) [left=of 3center]  {} ;
\node[draw, circle, fill=none, scale = 0.5, very thick] (3vo) [below=of 3center]  {} ;
\node[draw, cross out, very thick, scale = 0.6] (3hx) [right=of 3center] {} ;
\node[draw, cross out, very thick, scale = 0.6] (3vx) [above=of 3center] {} ;

% gridpoint 4 = (0,1) 
\node[draw = none] at (0,1) (4center) {} ;
\node[draw, circle, fill=none, scale = 0.5, very thick] (4ho) [left=of 4center]  {} ;
\node[draw, circle, fill=none, scale = 0.5, very thick] (4vo) [below=of 4center]  {} ;
\node[draw, cross out, very thick, scale = 0.6] (4hx) [right=of 4center] {} ;
\node[draw, cross out, very thick, scale = 0.6] (4vx) [above=of 4center] {} ;

% gridpoint 5 = (-1,1) 
\node[draw = none] at (-1,1) (5center) {} ;
\node[draw, circle, fill=none, scale = 0.5, very thick] (5ho) [left=of 5center]  {} ;
\node[draw, circle, fill=none, scale = 0.5, very thick] (5vo) [below=of 5center]  {} ;
\node[draw, cross out, very thick, scale = 0.6] (5hx) [right=of 5center] {} ;
\node[draw, cross out, very thick, scale = 0.6] (5vx) [above=of 5center] {} ;

% gridpoint 6 = (-1,0) 
\node[draw = none] at (-1,0) (6center) {} ;
\node[draw, circle, fill=none, scale = 0.5, very thick] (6ho) [left=of 6center]  {} ;
\node[draw, circle, fill=none, scale = 0.5, very thick] (6vo) [below=of 6center]  {} ;
\node[draw, cross out, very thick, scale = 0.6] (6hx) [right=of 6center] {} ;
\node[draw, cross out, very thick, scale = 0.6] (6vx) [above=of 6center] {} ;
    
% gridpoint 7 = (-1,-1) 
\node[draw = none] at (-1,-1) (7center) {} ;
\node[draw, circle, fill=none, scale = 0.5, very thick] (7ho) [left=of 7center]  {} ;
\node[draw, circle, fill=none, scale = 0.5, very thick] (7vo) [below=of 7center]  {} ;
\node[draw, cross out, very thick, scale = 0.6] (7hx) [right=of 7center] {} ;
\node[draw, cross out, very thick, scale = 0.6] (7vx) [above=of 7center] {} ;
    
% gridpoint 8 = (0,-1) 
\node[draw = none] at (0,-1) (8center) {} ;
\node[draw, circle, fill=none, scale = 0.5, very thick] (8ho) [left=of 8center]  {} ;
\node[draw, circle, fill=none, scale = 0.5, very thick] (8vo) [below=of 8center]  {} ;
\node[draw, cross out, very thick, scale = 0.6] (8hx) [right=of 8center] {} ;
\node[draw, cross out, very thick, scale = 0.6] (8vx) [above=of 8center] {} ;    
    
% gridpoint 9 = (1,-1) 
\node[draw = none] at (1,-1) (9center) {} ;
\node[draw, circle, fill=none, scale = 0.5, very thick] (9ho) [left=of 9center]  {} ;
\node[draw, circle, fill=none, scale = 0.5, very thick] (9vo) [below=of 9center]  {} ;
\node[draw, cross out, very thick, scale = 0.6] (9hx) [right=of 9center] {} ;
\node[draw, cross out, very thick, scale = 0.6] (9vx) [above=of 9center] {} ;

%% graph
% connections
\draw[arrow_outer] (7hx) -- (8ho);
\draw[arrow_outer] (8hx) -- (9ho);
\draw[arrow_outer] (9vx) -- (2vo);
\draw[arrow_outer] (2vx) -- (3vo);
\draw[arrow_outer] (4hx) -- (3ho);
\draw[arrow_outer] (5hx) -- (4ho);
\draw[arrow_outer] (6vx) -- (5vo);
\draw[arrow_outer] (7vx) -- (6vo);
% corners
%\draw[->, blue!80, very thick] (1.3, -1.1)  arc[radius=0.2, start angle=0, end angle= -90];
\draw[arrow_outer] (9hx) .. controls (1.4, -1.25) and (1.25, -1.4)    .. (9vo); % corner a) at 9
\draw[arrow_outer] (5vx) .. controls (-1.25, 1.4) and (-1.4, 1.25)    .. (5ho); % corner b) at 5
\draw[arrow_outer] (3vx) .. controls (1.25, 1.4) and (1.4, 1.25)      .. (3hx); % corner c) at 3 
\draw[arrow_outer] (7vo) .. controls (-1.25, -1.4) and (-1.4, -1.25)  .. (7ho); % corner d) at 7
    
%% pairs
% at 2
\draw (1, 0) ellipse[x radius = 0.265, y radius = 0.1, rotate = 90, color = red!100];
% at 3
\draw (1, 1) ellipse[x radius = 0.265, y radius = 0.1, rotate = 90, color = red!100];
\draw (1, 1) ellipse[x radius = 0.265, y radius = 0.1, rotate = 0, color = red!100];
% at 4
\draw (0, 1) ellipse[x radius = 0.265, y radius = 0.1, rotate = 0, color = red!100];
% at 5
\draw (-1, 1) ellipse[x radius = 0.265, y radius = 0.1, rotate = 90, color = red!100];
\draw (-1, 1) ellipse[x radius = 0.265, y radius = 0.1, rotate = 0, color = red!100];
% at 6
\draw (-1, 0) ellipse[x radius = 0.265, y radius = 0.1, rotate = 90, color = red!100];
% at 7
\draw (-1, -1) ellipse[x radius = 0.265, y radius = 0.1, rotate = 90, color = red!100];
\draw (-1, -1) ellipse[x radius = 0.265, y radius = 0.1, rotate = 0, color = red!100];
% at 8
\draw (0, -1) ellipse[x radius = 0.265, y radius = 0.1, rotate = 0, color = red!100];
% at 9 
\draw (1, -1) ellipse[x radius = 0.265, y radius = 0.1, rotate = 90, color = red!100];
\draw (1, -1) ellipse[x radius = 0.265, y radius = 0.1, rotate = 0, color = red!100];

    
    % direction
    %\draw[->, black!60, ultra thick] (0,0.4)  arc[radius=0.4, start angle= 90, end angle= 420];
    
    % draw boxes
    \draw[thick] (0.75,-1.4) -- (1.4,-1.4) -- (1.4, -0.25) -- (0.75,-0.25) -- (0.75,-1.4);
    \draw[thick] (0.75,-0.25) -- (1.4, -0.25) -- (1.4, 0.75) -- (0.75, 0.75) -- (0.75, -0.25);
    \draw[thick] (0.25, 0.75) -- (1.4, 0.75) -- (1.4, 1.4) -- (0.25, 1.4) -- (0.25, 0.75);
    \draw[thick] (-0.75, 0.75) -- (0.25, 0.75) -- (0.25, 1.4) -- (-0.75, 1.4) -- (-0.75, 0.75);
    \draw[thick] (-1.4, 0.25) -- (-0.75, 0.25) -- (-0.75, 1.4) -- (-1.4, 1.4) -- (-1.4, 0.25);
    \draw[thick] (-1.4, -0.75) -- (-0.75, -0.75) -- (-0.75, 0.25) -- (-1.4, 0.25) -- (-1.4, -0.75);
    \draw[thick] (-1.4, -1.4) -- (-0.25, -1.4) -- (-0.25, -0.75) -- (-1.4, -0.75) -- (-1.4, -1.4);
    \draw[thick] (-0.25,-1.4) -- (0.75, -1.4) -- (0.75, -0.75) -- (-0.25, -0.75) -- (-0.25,-1.4);
    
    % arrows
    \draw[->, blue!80, very thick] (1.65, 0) -- (1.65, 0.5);
    \draw[->, blue!80, very thick] (0, 1.65) -- (-0.5, 1.65);  
    \draw[->, blue!80, very thick] (-1.65, 0) -- (-1.65, -0.5);
    \draw[->, blue!80, very thick] (0, -1.65) -- (0.5, -1.65);  
    \draw[->, blue!80, very thick] (1.65, 1)  arc[radius=0.65, start angle=0 , end angle= 90];
    \draw[->, blue!80, very thick] (-1, 1.65)  arc[radius=0.65, start angle=90 , end angle= 180];
    \draw[->, blue!80, very thick] (-1.65, -1)  arc[radius=0.65, start angle=180 , end angle= 270];
    \draw[->, blue!80, very thick] (1, -1.65)  arc[radius=0.65, start angle=270 , end angle= 360];
\end{tikzpicture}
    \subcaption{Segmentierung gegen den Uhrzeigersin}
    \end{subfigure}
    \hspace{0.2\textwidth}
    \begin{subfigure}[c]{0.3\textwidth}
    \input{Diagrams/Evaluation/clockwiseSquareGraph.tex}
    \subcaption{Segmentierung in Uhrzeigersinn}
    \end{subfigure}
    \caption{Segmentierung des Graphen aus Abb. \ref{Abb: VorzeichenBestimmung}}
    \label{Abb: Segmentierung}
\end{figure}

\begin{figure}[h!]
    \centering
    \begin{tikzpicture}[scale = 1.5]

\begin{scope}
\draw[step=1cm, white, ultra thin] (-2.5,0) grid (5,0);
\draw[->, blue!80, very thick] (0,-0.325)  arc[radius=0.65, start angle=270 , end angle= 360];
\draw[->, blue!80, very thick] (4,-0.325)  arc[radius=0.65, start angle=-180 , end angle= -270];
\node[draw = none, text width=2em, scale = 1.5 ] at (-0.02,0) (Num1) {$a$)};
\node[draw = none, text width=11em, scale = 1 ] at (2.5,0) (Num1) {+\;+\;+\;+ = +1};
\node[draw = none, text width=2em, scale = 1.5 ] at (3.98,0) (Num1) {$\bar a$)};
\node[draw = none, text width=11em, scale = 1 ] at (6.5,0) (Num1) {+\;+\;+\;+ = +1};
\end{scope}

\begin{scope}[shift = {(0,-1)}]
\draw[->, blue!80, very thick] (0.65,-0.325)  arc[radius=0.65, start angle=0 , end angle= 90];
\draw[->, blue!80, very thick] (4.65,-0.325)  arc[radius=0.65, start angle=-90 , end angle= -180];
\node[draw = none, text width=2em, scale = 1.5 ] at (-0.02,0) (Num1) {$b$)};
\node[draw = none, text width=11em, scale = 1 ] at (2.5,0) (Num1) {+\;+\;-\;- = +1};
\node[draw = none, text width=2em, scale = 1.5 ] at (3.98,0) (Num1) {$\bar b$)};
\node[draw = none, text width=11em, scale = 1 ] at (6.5,0) (Num1) {-\;-\;+\;+ = +1};
\end{scope}

\begin{scope}[shift = {(0,-2)}]
\draw[->, blue!80, very thick] (0.65, 0.35)  arc[radius=0.65, start angle=90 , end angle= 180];
\draw[->, blue!80, very thick] (4.65, 0.35)  arc[radius=0.65, start angle=0 , end angle= -90];
\node[draw = none, text width=2em, scale = 1.5 ] at (-0.02,0) (Num1) {$c$)};
\node[draw = none, text width=11em, scale = 1 ] at (2.5,0) (Num1) {-\;-\;-\;- = +1};
\node[draw = none, text width=2em, scale = 1.5 ] at (3.98,0) (Num1) {$\bar c$)};
\node[draw = none, text width=11em, scale = 1 ] at (6.5,0) (Num1) {-\;-\;-\;- = +1};
\end{scope}

\begin{scope}[shift = {(0,-3)}]
\draw[->, blue!80, very thick] (0, 0.35)  arc[radius=0.65, start angle=170 , end angle= 270];
\draw[->, blue!80, very thick] (4, 0.35)  arc[radius=0.65, start angle=-270 , end angle= -360];
\node[draw = none, text width=2em, scale = 1.5 ] at (-0.02,0) (Num1) {$d$)};
\node[draw = none, text width=11em, scale = 1 ] at (2.5,0) (Num1) {-\;-\;-\;+ = -1};
\node[draw = none, text width=2em, scale = 1.5 ] at (3.98,0) (Num1) {$\bar d$)};
\node[draw = none, text width=11em, scale = 1 ] at (6.5,0) (Num1) {+\;-\;-\;- = -1};
\end{scope}

\begin{scope}[shift = {(0,-4)}]
\draw[->, blue!80, very thick] (0.25, -0.25) -- (0.25, 0.25); 
\draw[->, blue!80, very thick] (4.25,  0.25) -- (4.25, -0.25);
\node[draw = none, text width=2em, scale = 1.5 ] at (-0.02,0) (Num1) {$g$)};
\node[draw = none, text width=11em, scale = 1 ] at (2.5,0) (Num1) {+\;+ = +1};
\node[draw = none, text width=2em, scale = 1.5 ] at (3.98,0) (Num1) {$\bar g$)};
\node[draw = none, text width=11em, scale = 1 ] at (6.5,0) (Num1) {+\;+ = +1};
\end{scope}

\begin{scope}[shift = {(0,-5)}]
\draw[->, blue!80, very thick] (0, 0) -- (0.5, 0); % right 
\draw[->, blue!80, very thick] (4.5, 0) -- (4, 0); % left
\node[draw = none, text width=2em, scale = 1.5 ] at (-0.02,0) (Num1) {$h$)};
\node[draw = none, text width=11em, scale = 1 ] at (2.5,0) (Num1) {+\;+ = +1};
\node[draw = none, text width=2em, scale = 1.5 ] at (3.98,0) (Num1) {$\bar h$)};
\node[draw = none, text width=11em, scale = 1 ] at (6.5,0) (Num1) {+\;+ = +1};
\end{scope}

\end{tikzpicture}
    \caption{Alle Teilsequenzen eines beliebigen gerichteten Graphen }
    \label{Abb: directedElemets}
\end{figure}


\subsection{Vorzeichen von Überschneidungen}

Das Vorzeichen eines geschlossenen, zusammenhängenden Graphen mit einer Selbstüberschneidung kann mit der selben Methode bestimmt werden, wie für geschlossenen Graphen ohne Selbstüberschneidung. Das Vorzeichen ergibt sich dann als Produkt des Vorzeichens $V_c$ für die Schließung, des Vorzeichens $V_k$ für die Kreuzung und des Vorzeichens $V_s$ aufgrund der Vorkommenden Segmente $a)$ bis $\bar h)$. Die Kreuzung kann durch keines der Segmente $a)$ bis $\bar h)$ dargestellt werden. Der Algorithmus zur Bestimmung des Vorzeichens durch Umordnen legt aber eindeutig fest, wie die Kreuzung durchlaufen wird. Dadurch kann der Graph in zwei Schleifen mit entgegengesetzter Durchlaufrichtung aufgeteilt werden. Ersetzt man die Kreuzung durch zwei Elemente $x$ und $\bar x$, wobei x eines der Segmente  $a)$, $b)$, $c)$ und $d)$ ist, erhält man zwei geschlossenen Graphen mit positiven Vorzeichen, welches sich als Produkt $V_{1,s}\cdot V_{1,c}$ bzw. $V_{2,s}\cdot V_{2,c}$. Damit die Ersetzung der Kreuzung durch der Segmentpaare $x-\bar x$ eine gültige Darstellung als Produkt von Graßman Paaren hat müssen die Getrennten Graphen auf dem Gitter außeinandergeschoben werden und die Terme $P_a \cdot P_b$ oder $P_c \cdot P_d$ hinzugefügt werden. Die translation am gitter ändert das Vorzeichen nicht, aber die zusätzlichen paare führen jedoch zu einem negativen Vorzeichen. Das Auftauschen der Kreuzung trägt also mit einem negativen Vorzeichen bei. Dies erkennt man auch aus der Rechnung . 

\begin{equation}
\begin{aligned}
&V_c \cdot V_k \cdot V_s = V_c \cdot V_k \cdot V_{1,s} \cdot V_{2,s}   \overset{!}{=} V_{1,c} \cdot V_{1,s}  \cdot V_{2,c} \cdot V_{2,s} \\
\\
& V_{c} =  V_{1,c} = V_{2,c} = -1  \;\;\;\;\Rightarrow\;\;\;\;   V_k = -1
\end{aligned}
\end{equation}

\noindent Die Argumentation lässt sich nun leicht auf Graphen mit $N_k$ Selbstüberschneidungen ausweiten, indem man diese induktiv wie oben in Graphen ohne Selbstüberschneidung zerlegt. Dabei liefert jede Kreuzung ein negatives Vorzeichen. 

\subsection{Wahl der Koeffizienten}
In der Arbeit wurden alle Koeffizienten $a_{i,\alpha}$ für $\alpha \in \{a,b,c,d,e,f\}$ zu $-1$ gewählt. Mit dieser Wahl tragen alle Monomer nach wie vor mit $-1$ ein zum Gesamtvorzeichen bei, da diese immer mit einem Produkt ${a_{i,\alpha} \cdot a_{i,\alpha'} = 1}$ auftreten. Der Faktor $-1$ kommt dabei durch die Spurbildung, wie in Abschnitt \ref{sec: vorzeicehnMonomer} beschrieben, zustande. Die Kreuzungen tragen, unabhängig von der Wahl der $a_{i,\alpha}$, einem Faktor $-1$ bei. Jedes Segment $a)$ bis $\bar{h})$ eines Graphen kommt mit einen Koeffizienten $a_{i,\alpha}$ und liefert somit einen Faktor $-1$. Insgesamt trägt dann jeder Punkt auf dem Gitter mit einem Faktor $-1$ bei, sodass sich das gesamte Vorzeichen für jeden Graphen bei der Spurbildung unter Berücksichtigung der Koeffizienten zu $(-1)^N$ ergibt.