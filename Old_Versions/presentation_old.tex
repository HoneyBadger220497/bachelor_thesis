\documentclass[11pt]{beamer}
\usetheme{CambridgeUS}
\usepackage[utf8]{inputenc}
\usepackage[german]{babel}
\usepackage[T1]{fontenc}
\usepackage{amsmath}
\usepackage{amsfonts}
\usepackage{amssymb}
\usepackage{graphicx}
\usepackage{verbatim}
\usepackage{color}
\usepackage{amsmath,amssymb, amsfonts, bm}
% tabulars
\usepackage{arydshln} % dashed hlines
% tikz
\usepackage{tikz}
\usetikzlibrary{positioning}
\usetikzlibrary{shapes}
\usetikzlibrary{arrows}
\usetikzlibrary{decorations.markings}

\tikzset{
  arrow_outer/.style={-latex, shorten >=5, shorten <=5,  very thick, color = blue!80}, 
  arrow_inner/.style={-latex, shorten >=2, shorten <=2, thick, color = blue!80},
  arrow_start/.style={-latex, shorten >=2, shorten <=2, thick, color = black!80},
  arrow_double/.style={<->, shorten >=2, shorten <=2, very thick, color = black!80},
  arrow_down/.style={->, shorten >=2, shorten <=2, very thick, color = black!80},
  arrow_grid_in/.style={-latex, shorten >=2, shorten <=2, ultra thick, color = black!100},
  arrow_grid_out/.style={-latex, shorten >=2, shorten <=2, ultra thick, color = black!50},
  grid_point/.style={circle, draw, color=black!100, fill=black!100, minimum size = 8pt, inner sep = 0},
  small_grid_point/.style={circle, draw, color=black!100, fill=black!100, minimum size = 4pt, inner sep = 0},
  small_grid_point_right/.style={circle, draw, color=red!80, fill=red!80, minimum size = 4pt, inner sep = 0}
  ,
  small_grid_point_left/.style={circle, draw, color=blue!80, fill=blue!80, minimum size = 4pt, inner sep = 0}}
% arrow_corner/.style={thick, color = blue!80, decoration={markings, mark=at position 0.5 with{\arrow{latex}}}, postaction={decorate}}

% Makro-commands
\newcommand{\corr}[1]{\left\langle #1 \right\rangle}
\newcommand{\Sp}[1]{\mathrm{Sp}\left( #1 \right)}
\renewcommand{\exp}[1]{\mathrm{exp}\left( #1 \right)}
%\renewcommand{\cos}[1]{\mathrm{cos}\left( #1 \right)}
%\renewcommand{\sin}[1]{\mathrm{sin}\left( #1 \right)}
%\renewcommand{\sinh}[1]{\mathrm{sinh}\left( #1 \right)}
%\renewcommand{\tanh}[1]{\mathrm{tanh}\left( #1 \right)}
\newcommand{\atanh}[1]{\mathrm{atanh}\left( #1 \right)}
\newcommand{\sign}[1]{\mathrm{sign}\left( #1 \right)}
\renewcommand{\det}[1]{\mathrm{det}\left( #1 \right)}
\newcommand{\pf}[1]{\mathrm{pf}\left( #1 \right)}
\renewcommand{\ln}[1]{\mathrm{ln}\left( #1 \right)}
\renewcommand{\d}[0]{\mathrm{d}}
\newcommand{\tlim}{ \underset{V,N \to \infty}{\text{lim}^{*}}}


\definecolor{darkblue}{rgb}{0,0,0.8}

\setbeamercolor{section in toc}{fg=black,bg=white}
\setbeamercolor{alerted text}{fg=darkblue!80!gray}
\setbeamercolor*{palette primary}{fg=darkblue!60!black,bg=gray!30!white}
\setbeamercolor*{palette secondary}{fg=darkblue!70!black,bg=gray!15!white}
\setbeamercolor*{palette tertiary}{bg=darkblue!80!black,fg=gray!10!white}
\setbeamercolor*{palette quaternary}{fg=darkblue,bg=gray!5!white}

\setbeamercolor*{sidebar}{fg=darkblue,bg=gray!15!white}

\setbeamercolor*{palette sidebar primary}{fg=darkblue!10!black}
\setbeamercolor*{palette sidebar secondary}{fg=white}
\setbeamercolor*{palette sidebar tertiary}{fg=darkblue!50!black}
\setbeamercolor*{palette sidebar quaternary}{fg=gray!10!white}

%\setbeamercolor*{titlelike}{parent=palette primary}
\setbeamercolor{titlelike}{parent=pallette primary,fg=darkblue}
\setbeamercolor{frametitle}{bg=gray!10!white}
\setbeamercolor{frametitle right}{bg=gray!60!white}

\setbeamercolor*{separation line}{}
\setbeamercolor*{fine separation line}{}

\setbeamertemplate{title page}[default][colsep=-4bp,rounded=true]
\setbeamercolor*{title}{fg=darkblue, bg=white}  

% Umgebungs-commands
\usepackage{mdframed}
\newmdenv[backgroundcolor=gray!10!white, skipabove=12pt, linecolor=white, innerbottommargin = 10pt,
frametitlerule=true, frametitlerulecolor=black, frametitlebackgroundcolor=black!10, frametitlerulewidth=0pt]{grayframe}

% Change Footline
\setbeamertemplate{footline}
{%
   \leavevmode%
   \hbox{\begin{beamercolorbox}[wd=.2\paperwidth,ht=2.5ex,dp=1.125ex,leftskip=.3cm plus1fill,rightskip=.3cm]{author in head/foot}%
     \usebeamerfont{author in head/foot}\insertshortauthor
   \end{beamercolorbox}%
   \begin{beamercolorbox}[wd=.8\paperwidth,ht=2.5ex,dp=1.125ex,leftskip=.3cm,rightskip=.3cm plus1fil]{title in head/foot}%
     \usebeamerfont{title in head/foot} Berechnung der spontanen Magnetisierung von Ising Ferromagneten mit Graßmann Zahlen
   \end{beamercolorbox}}%
   \vskip0pt%
}

\author{Joachim Pomper}
\title{Analytische Berechnung der spontanen Magnetisierung von isotropen homogenen Ising Ferromagneten unter der Verwendung von Graßmann Zahlen}
%\setbeamercovered{transparent} 
%\setbeamertemplate{navigation symbols}{} 
%\logo{} 
\institute{Technische Universität Graz} 
\date{14.10.2020} 
\begin{document}


\begin{frame}
\titlepage
\end{frame}

\begin{frame}
\tableofcontents
\end{frame}

\section{Einleitung}
    
    \begin{frame}{Warum das 2d Modell exakt lösen?}
        \begin{itemize}
            \item Einfachstes statistisches Modell welches einen Phasenübergang 2ter Ordnung aufweist.
            \item[]
            \pause
            \item Exakte Lösungen geben neue Einsicht in das System.
            \item[]
            \pause
            \item Kann als Test-Modell für numerische Näherungsverfahren verwendet werden.
            \item[]
        \end{itemize}
    \end{frame}

\section{Grundlagen}
\subsection{Ising Modell}
    \begin{frame}{Periodisches Ising-Gitter}
        \begin{figure}[h]
            \centering
            \begin{tikzpicture}[scale = 1.5]
\begin{scope}
    \node[draw = none] at (0,0)   (p0p0) {$\sigma$};
\node[draw = none] at (0,1)   (p0p1) {$\sigma$};
\node[draw = none] at (1,0)   (p1p0) {$\sigma$};
\node[draw = none] at (0,-1)  (p0m1) {$\sigma$};
\node[draw = none] at (-1,0)  (m1p0) {$\sigma$};
\node[draw = none] at (1,-1)  (p1m1) {$\sigma$};
\node[draw = none] at (-1,1)  (m1p1) {$\sigma$};
\node[draw = none] at (-1,-1) (m1m1) {$\sigma$};
\node[draw = none] at (1,1)   (p1p1) {$\sigma$};

\node[draw = none] at (1.7,0)   (p2p0) {};
\node[draw = none] at (1.7,1)   (p2p1) {};
\node[draw = none] at (1,1.7)   (p1p2) {};
\node[draw = none] at (0,1.7)   (p0p2) {};
\node[draw = none] at (-1,1.7)  (m1p2) {};
\node[draw = none] at (-1.7,1)  (m2p1) {};
\node[draw = none] at (-1.7,0)  (m2p0) {};
\node[draw = none] at (-1.7,-1) (m2m1) {};
\node[draw = none] at (-1,-1.7) (m1m2) {};
\node[draw = none] at (0,-1.7)  (p0m2) {};
\node[draw = none] at (1,-1.7)  (p1m2) {};
\node[draw = none] at (1.7,-1)  (p2m1) {};

%% arrows
\draw[arrow_grid_in] (p0p0) -- (p0p1);
\draw[arrow_grid_in] (p0p0) -- (p1p0);

\draw[arrow_grid_in] (p1p0) -- (p1p1);


\draw[arrow_grid_in] (p0p1) -- (p1p1);
\draw[arrow_grid_out] (p0p1) -- (p0p2);

\draw[arrow_grid_out] (p1p1) -- (p1p2);


\draw[arrow_grid_in] (p1m1) -- (p1p0);


\draw[arrow_grid_in] (p0m1) -- (p0p0);
\draw[arrow_grid_in] (p0m1) -- (p1m1);

\draw[arrow_grid_in] (m1m1) -- (m1p0);
\draw[arrow_grid_in] (m1m1) -- (p0m1);

\draw[arrow_grid_in] (m1p0) -- (m1p1);
\draw[arrow_grid_in] (m1p0) -- (p0p0);

\draw[arrow_grid_in] (m1p1) -- (p0p1);
\draw[arrow_grid_out] (m1p1) -- (m1p2);



\draw[arrow_grid_out] (m1m2) -- (m1m1);
\draw[arrow_grid_out] (p0m2) -- (p0m1);
\draw[arrow_grid_out] (p1m2) -- (p1m1);


    \draw[arrow_grid_out] (m2p1) -- (m1p1);
    \draw[arrow_grid_out] (m2p0) -- (m1p0);
    \draw[arrow_grid_out] (m2m1) -- (m1m1);

\end{scope}
\begin{scope}[shift = {(2,0)}]
    \node[draw = none] at (0,0)   (p0p0) {$\sigma$};
\node[draw = none] at (0,1)   (p0p1) {$\sigma$};
\node[draw = none] at (1,0)   (p1p0) {$\sigma$};
\node[draw = none] at (0,-1)  (p0m1) {$\sigma$};
\node[draw = none] at (-1,0)  (m1p0) {$\sigma$};
\node[draw = none] at (1,-1)  (p1m1) {$\sigma$};
\node[draw = none] at (-1,1)  (m1p1) {$\sigma$};
\node[draw = none] at (-1,-1) (m1m1) {$\sigma$};
\node[draw = none] at (1,1)   (p1p1) {$\sigma$};

\node[draw = none] at (1.7,0)   (p2p0) {};
\node[draw = none] at (1.7,1)   (p2p1) {};
\node[draw = none] at (1,1.7)   (p1p2) {};
\node[draw = none] at (0,1.7)   (p0p2) {};
\node[draw = none] at (-1,1.7)  (m1p2) {};
\node[draw = none] at (-1.7,1)  (m2p1) {};
\node[draw = none] at (-1.7,0)  (m2p0) {};
\node[draw = none] at (-1.7,-1) (m2m1) {};
\node[draw = none] at (-1,-1.7) (m1m2) {};
\node[draw = none] at (0,-1.7)  (p0m2) {};
\node[draw = none] at (1,-1.7)  (p1m2) {};
\node[draw = none] at (1.7,-1)  (p2m1) {};

%% arrows
\draw[arrow_grid_in] (p0p0) -- (p0p1);
\draw[arrow_grid_in] (p0p0) -- (p1p0);

\draw[arrow_grid_in] (p1p0) -- (p1p1);


\draw[arrow_grid_in] (p0p1) -- (p1p1);
\draw[arrow_grid_out] (p0p1) -- (p0p2);

\draw[arrow_grid_out] (p1p1) -- (p1p2);


\draw[arrow_grid_in] (p1m1) -- (p1p0);


\draw[arrow_grid_in] (p0m1) -- (p0p0);
\draw[arrow_grid_in] (p0m1) -- (p1m1);

\draw[arrow_grid_in] (m1m1) -- (m1p0);
\draw[arrow_grid_in] (m1m1) -- (p0m1);

\draw[arrow_grid_in] (m1p0) -- (m1p1);
\draw[arrow_grid_in] (m1p0) -- (p0p0);

\draw[arrow_grid_in] (m1p1) -- (p0p1);
\draw[arrow_grid_out] (m1p1) -- (m1p2);



\draw[arrow_grid_out] (m1m2) -- (m1m1);
\draw[arrow_grid_out] (p0m2) -- (p0m1);
\draw[arrow_grid_out] (p1m2) -- (p1m1);


    \node[draw = none, scale = 1] at (-0.34,0.5)   (J) {$J_{i,j}$};
    \node[draw = none, scale = 1] at ( 0.5,-0.28)   (J) {$J_{i,k}$};
    \node[draw = none, scale = 0.8] at (0.1, -0.1)   (J) {$i$};
    \node[draw = none, scale = 0.8] at (1.1, -0.1)   (J) {$k$};
    \node[draw = none, scale = 0.8] at (0.1,  0.9)   (J) {$j$};
    %\node[draw = none, scale = 1] at (-0.25, -0.2)   (J) {$\sigma_i$};
\end{scope}
\begin{scope}[shift = {(4,0)}]
    \node[draw = none] at (0,0)   (p0p0) {$\sigma$};
\node[draw = none] at (0,1)   (p0p1) {$\sigma$};
\node[draw = none] at (1,0)   (p1p0) {$\sigma$};
\node[draw = none] at (0,-1)  (p0m1) {$\sigma$};
\node[draw = none] at (-1,0)  (m1p0) {$\sigma$};
\node[draw = none] at (1,-1)  (p1m1) {$\sigma$};
\node[draw = none] at (-1,1)  (m1p1) {$\sigma$};
\node[draw = none] at (-1,-1) (m1m1) {$\sigma$};
\node[draw = none] at (1,1)   (p1p1) {$\sigma$};

\node[draw = none] at (1.7,0)   (p2p0) {};
\node[draw = none] at (1.7,1)   (p2p1) {};
\node[draw = none] at (1,1.7)   (p1p2) {};
\node[draw = none] at (0,1.7)   (p0p2) {};
\node[draw = none] at (-1,1.7)  (m1p2) {};
\node[draw = none] at (-1.7,1)  (m2p1) {};
\node[draw = none] at (-1.7,0)  (m2p0) {};
\node[draw = none] at (-1.7,-1) (m2m1) {};
\node[draw = none] at (-1,-1.7) (m1m2) {};
\node[draw = none] at (0,-1.7)  (p0m2) {};
\node[draw = none] at (1,-1.7)  (p1m2) {};
\node[draw = none] at (1.7,-1)  (p2m1) {};

%% arrows
\draw[arrow_grid_in] (p0p0) -- (p0p1);
\draw[arrow_grid_in] (p0p0) -- (p1p0);

\draw[arrow_grid_in] (p1p0) -- (p1p1);


\draw[arrow_grid_in] (p0p1) -- (p1p1);
\draw[arrow_grid_out] (p0p1) -- (p0p2);

\draw[arrow_grid_out] (p1p1) -- (p1p2);


\draw[arrow_grid_in] (p1m1) -- (p1p0);


\draw[arrow_grid_in] (p0m1) -- (p0p0);
\draw[arrow_grid_in] (p0m1) -- (p1m1);

\draw[arrow_grid_in] (m1m1) -- (m1p0);
\draw[arrow_grid_in] (m1m1) -- (p0m1);

\draw[arrow_grid_in] (m1p0) -- (m1p1);
\draw[arrow_grid_in] (m1p0) -- (p0p0);

\draw[arrow_grid_in] (m1p1) -- (p0p1);
\draw[arrow_grid_out] (m1p1) -- (m1p2);



\draw[arrow_grid_out] (m1m2) -- (m1m1);
\draw[arrow_grid_out] (p0m2) -- (p0m1);
\draw[arrow_grid_out] (p1m2) -- (p1m1);


    \draw[arrow_grid_out] (p1p0) -- (p2p0);
    \draw[arrow_grid_out] (p1p1) -- (p2p1);
    \draw[arrow_grid_out] (p1m1) -- (p2m1);
\end{scope}

\end{tikzpicture}
        \end{figure}
    \end{frame}
    
    \begin{frame}{Hamiltonfunktion des 2d Ising-Gitters}
        \begin{grayframe}
            \begin{equation} \nonumber
                H(S) := - J \sum_{(i,j)}  \,\sigma_i(S) \sigma_j(S) 
            \end{equation}
        \end{grayframe}
        \vspace{0.5cm}
        \begin{equation} \nonumber
        S \in \{ \left(\sigma_1, \dots, \sigma_N \right) \;\vert\; \sigma_i \in \{-1,1\} \} 
        \end{equation}\\
        \vspace{0.5cm}
        Annahmen für das behandelte Modell :
        \textcolor{white}{
            \begin{itemize}
                \item Nur nächste Nachbar Wechselwirkung
                \item Isotropie und Homogenität des Modells
                \item Ferromagnetischer Austausch d.h. $J>0$
                \item Kein externes Magnetfeld
            \end{itemize}}
        
    \end{frame}

\subsection{Statistische Physik}
    \begin{frame}{Zentrale Größen der Statistischen Physik}
    
        \begin{grayframe}[frametitle = {Zustandssumme}]
        \begin{equation} \nonumber
            Z(T,N,V) := \sum_{\{S\}} \mathrm{e}^{- \beta H( S ) } 
        \end{equation}
        \end{grayframe}
        
        \vspace{0.5cm}
        \begin{equation} \nonumber
            \beta := \frac{1}{k_B T}
        \end{equation}
        
        \begin{grayframe}[frametitle = {Spin-Spin-Korrelation}]
        \begin{equation} \nonumber
             \corr{\sigma_{p} \sigma_{q}}_{T,N,V}
        \end{equation}   
        \end{grayframe}


    \end{frame}
    
    \begin{frame}{Thermodynamischer Limes} 
        
        Endliche System werden auf ein beliebig großes System extrapoliert.
        \vspace{0.5cm}
    
        \begin{grayframe}
            \begin{equation} \nonumber
                \centering
                \begin{tabular}{lccccc}
                            &  &                 &    &      $N \longrightarrow \infty$\\
                    $\tlim$ &  &  $\iff$           &   &      $V \longrightarrow \infty$ \\
                            &  &                 &   & $n =  N/V = konst. < \infty$
                \end{tabular}
                \end{equation}
        \end{grayframe}
    \end{frame}
    
    \begin{frame}{Spontane Magnetisierung} 
        \begin{grayframe}[frametitle = {Definition}]   
            \begin{equation} \nonumber
                \mathcal{M}_S := n \mu \corr{\sigma}
            \end{equation}
        \end{grayframe}
        \vspace{0.8cm} 
        \only<1>{
            \centering
             \begin{equation} \nonumber
                \text{ Mittleres magnetische Moment pro Volumen ohne externes Magnetfeld}
            \end{equation}}
        \only<2-3>{
            \begin{equation}\nonumber
                \lim_{m \rightarrow \infty} \corr{\sigma_{(0,0)} \sigma_{(m,0)}} = \corr{\sigma}^2
            \end{equation}
         }
         \visible<3>{
            \begin{grayframe}[frametitle = {Verwendeter Ausdruck}]   
                \begin{equation} \nonumber
                    \mathcal{M}_S = n \mu \lim_{m \rightarrow \infty} \sqrt{\corr{\sigma_{(0,0)} \sigma_{(m,0)}}}
                \end{equation}
            \end{grayframe}
        }
    \end{frame}
    
\subsection{Graßmann Zahlen}
    
    \begin{frame}{Graßmann Algebra}
     $(\mathcal A, \cdot, +, \wedge)$ heißt Graßmann Algebra falls gilt: 
     \vspace{0.5cm}
    \begin{itemize}
    \item $(\mathcal A, \cdot, +)$ ist ein Vektorraum
    \item $\wedge: \mathcal A \rightarrow \mathcal A$ ist assoziativ.
    \item Es gibt ein neutrales Elemtent $1$ bzgl. $\wedge: \mathcal A \rightarrow \mathcal A$ 
    \item Es gibt ein Familie $(\eta_1, \eta_2, \dots, \eta_n)$ welche $\mathcal A$ generiert.
    \item Die Generatoren erfüllen die Eigenschaft $\eta_i \wedge \eta_j = - \eta_j \wedge \eta_i$
    \end{itemize}
    \vspace{0.5cm}
    Ein Objekt $\eta \in\mathcal{A}$ heißt dann Graßmann-Zahl.
    \end{frame}
    
    \begin{frame}{Generatoren}
    Generatoren antikommutieren
    \begin{equation} \nonumber
    \eta_i \; \eta_j = - \eta_j \;\eta_i
    \end{equation}
    \pause
    Quadrate von Generatoren verschwinden
    \begin{equation} \nonumber
    \eta_i^2 = 0
    \end{equation}
    \pause
    Paare von Generatoren kommutieren
    \begin{equation} \nonumber
    (\eta_i\; \eta_j) (\eta_l \;\eta_k) = (\eta_l\; \eta_k) (\eta_i\; \eta_j)
    \end{equation}

    \end{frame}
    
    \begin{frame}{Graßmann Zahlen}
    
     Für jede Graßmann Zahl $f$ gibt es eine eindeutige Darstellung\\
     \begin{align} \nonumber 
     f = f_0\;1 + \sum_{i} f_i \;\eta_i + \sum_{i_1<i_2} f_{i_1,i_2} \;\eta_{i_1} \,\eta_{i_2} + \sum_{i_1<i_2<i_3} \dots \;+ f_{1,2,\dots,n} \; \eta_{1}\,\eta_{2}\,\cdots\,\eta_{n}
     \end{align}\\
     \vspace{0.5cm}
     \pause
     \begin{grayframe}[frametitle = {Spur einer Graßmann Zahl}]
     \begin{equation}\nonumber
     \Sp{f} = f_{1,2,\dots,n} \in \mathbb{C}
     \end{equation}
     \end{grayframe}
     \end{frame}
    
    \begin{frame}{Graßmann Funktionen}
    
    Quadratische Wirkung 
    \begin{equation} \nonumber
    \bm{\eta}^T \bm{A}\; \bm{\eta} = (\eta_1, \eta_2) \left(\begin{array}{cc} 1&2\\-2&3\end{array}\right) \left(\begin{array}{c} \eta_1 \\ \eta_2 \end{array}\right) = - 2 \eta_2 \eta_1 + 2 \eta_1 \eta_2 = 4 \eta_1 \eta_2
    \end{equation}
    \visible<2>{Exponentialfunktion 
    \begin{align}
    \exp{\eta_1} & = 1 + \eta_1 + \frac{\eta_1^2}{2} + \dots = 1 + \eta_1 \nonumber
    \end{align}}
    \end{frame}
    
\subsection{Pfaffsche Determinante}

    \begin{frame}{Pfaffsche Determinante}
    \begin{grayframe}[frametitle = {Definition}]
    Für $\bm{M} = -\bm{M}^T \in \mathbb{C}^{2k\times2k}$ \\
    \begin{equation} \nonumber
        \pf{\bm{M}} = \frac{1}{2^n\,n!} \sum_{P \in S_{2n}} \sign{P} \prod_{i=1}^n M_{P(2i-1),P(2i)} 
    \end{equation}
    \end{grayframe} 
    \vspace{0.5cm}
    \pause
    \begin{align}
    \pf{\begin{array}{cccc}  
        0  &  a  &  b & c \\
        -a &  0  &  d & e \\
        -b & -d  &  0 & f \\
        -c & -e  & -f & 0 \\
        \end{array}} &= af - be +dc \nonumber 
    \end{align}\\
    \end{frame}
    
    \begin{frame}{Zentrale Größen im Beweis}
    
    \begin{grayframe}[frametitle = {Graßmann Zustanssumme}]
    \begin{equation} \nonumber 
    \mathcal{Z} = \Sp{\exp{\frac{1}{2}\bm{\eta}^T \bm{A}\; \bm{\eta}}} \visible<2>{= \pf{\bm{A}}}
    \end{equation}
    \end{grayframe}
    \vspace{0.5cm}
    \begin{grayframe}[frametitle = {Graßmann Korreltation}]
    \begin{equation} \nonumber 
    \corr{\eta_{i}\eta_{j}} = \frac{ \Sp{\exp{\frac{1}{2}\,\bm{\eta}^T \bm{A}\; \bm{\eta}}\eta_{i}\eta_{j}}}{ \Sp{\exp{\frac{1}{2}\,\bm{\eta}^T \bm{A}\; \bm{\eta}}}} \visible<2>{= \left(\bm{A}^{-1}\right)_{i,j}}
    \end{equation}
    \end{grayframe}
    
    \end{frame}
    
\section{Der Beweis}
\subsection{Reformulierung des Ising-Modells}
    
    \begin{frame}{Graphische Darstellung der Zustandssumme}
    
    \begin{grayframe}
    \begin{equation} \nonumber
    \Xi [t_{i,j}] =  \sum_{\{G\}} \prod_{(i,j)\in K_{G}} t_{i,j}
    \end{equation} 
    \end{grayframe}
    \vspace{0.3cm}
    Die Graphen $G = (V_G, K_G)$ besitzen die Eigenschaften: 
    \begin{itemize}
    \item[i)] Alle Vertices des Graphen liegen auf dem Gitter
    \item[ii)] $G$ ist geschlossen (auch über periodischen Rand)
    \item[iii)] $G$ kann durchlaufen werden ohne eine Kante zweimal zu nutzen. 
    \end{itemize}

    \begin{figure}[h!]
        \centering
        \begin{tikzpicture}[scale = 0.8]
         \draw[step=1cm,gray, ultra thin] (-3.5,-1.5) grid (4.5,1.5);

\node[grid_point] at (-3,1)   (m3p1) {};
\node[grid_point] at (-3,0)   (m3p0) {};
\node[grid_point] at (-3,-1)  (m3m1) {};

\node[grid_point] at (-2,1)   (m2p1) {};
\node[grid_point] at (-2,0)   (m2p0) {};
\node[grid_point] at (-2,-1)  (m2m1) {};

\node[grid_point] at (-1,1)   (m1p1) {};
\node[grid_point] at (-1,0)   (m1p0) {};
\node[grid_point] at (-1,-1)  (m1m1) {};

\node[grid_point] at (0,1)   (p0p1) {};
\node[grid_point] at (0,0)   (p0p0) {};
\node[grid_point] at (0,-1)  (p0m1) {};

\node[grid_point] at (3,1)   (p3p1) {};
\node[grid_point] at (3,0)   (p3p0) {};
\node[grid_point] at (3,-1)  (p3m1) {};

\node[grid_point] at (2,1)   (p2p1) {};
\node[grid_point] at (2,0)   (p2p0) {};
\node[grid_point] at (2,-1)  (p2m1) {};

\node[grid_point] at (1,1)   (p1p1) {};
\node[grid_point] at (1,0)   (p1p0) {};
\node[grid_point] at (1,-1)  (p1m1) {};

\node[grid_point] at (4,1)   (p4p1) {};
\node[grid_point] at (4,0)   (p4p0) {};
\node[grid_point] at (4,-1)  (p4m1) {};

%% Graph
\draw[-, black!100, very thick] (m3p1) -- (m2p1) ;
\draw[-, black!100, very thick] (m2p1) -- (m1p1) ;
\draw[-, black!100, very thick] (m1p1) -- (m1p0) ;
\draw[-, black!100, very thick] (m1p0) -- (p0p0) ;
\draw[-, black!100, very thick] (p0p0) -- (p0m1) ;
\draw[-, black!100, very thick] (p0m1) -- (m1m1) ;
\draw[-, black!100, very thick] (m1m1) -- (m2m1) ;
\draw[-, black!100, very thick] (m2m1) -- (m3m1) ;
\draw[-, black!100, very thick] (m3m1) -- (m3p0) ;
\draw[-, black!100, very thick] (m3p0) -- (m3p1) ;

\draw[-, black!100, very thick] (p1p1) -- (p2p1) ;
\draw[-, black!100, very thick] (p2p1) -- (p3p1) ;
\draw[-, black!100, very thick] (p3p1) -- (p3p0) ;
\draw[-, black!100, very thick] (p3p0) -- (p4p0) ;
\draw[-, black!100, very thick] (p4p0) -- (p4m1) ;
\draw[-, black!100, very thick] (p4m1) -- (p3m1) ;
\draw[-, black!100, very thick] (p3m1) -- (p3p0) ;
\draw[-, black!100, very thick] (p3p0) -- (p2p0) ;
\draw[-, black!100, very thick] (p2p0) -- (p1p0) ;
\draw[-, black!100, very thick] (p1p0) -- (p1p1) ;

        \end{tikzpicture}
        \label{Abb: erlaubte Graphen}
    \end{figure}
    \end{frame}
    
    \begin{frame}{Hochtemperatur-Darstellung Zustandssumme}
    
    \begin{grayframe}[frametitle = {Zustandssumme}]
    \begin{equation} \nonumber
    Z = 2^N \cosh(\beta J)^{2N} \Xi[t] 
    \end{equation}
    
    \vspace{0.5cm}
    %\centering
    
    Mit Kantengewichtung:  
    \begin{equation} \nonumber
    t = \tanh(\beta J) 
    \end{equation}
    \end{grayframe}
    
    \end{frame}
    
    \begin{frame}{Hochtemperatur-Darstellung Korrelation}
    
    \begin{grayframe}[frametitle = {Spin-Spin-Korrelation}]
    \begin{equation} \nonumber
    \corr{\sigma_{0,0}\,\sigma_{0,m}} = \frac{t^m \Xi[\tilde{t}_{i,j}]}{\Xi[t]} 
    \end{equation}
    
    \vspace{0.5cm}
    Mit modifizierter Kantengewichtung: 
    \begin{equation} \nonumber
    \tilde{t}_{i,j} = \left\{\begin{array}{ll} t^{-1} & \text{Kante auf x-Achse zwischen } 0 \; \text{und } m\\
          t & \text{Sonst} \end{array} \right.
    \end{equation}
    \end{grayframe}
    \end{frame}
    
    \begin{frame}{Gitterdefekt $D_m$ für $\corr{\sigma_{0,0}\,\sigma_{0,m}}$}
    
    \begin{equation} \nonumber 
        D_2 = \{(0,5), (5,10)\}
    \end{equation}
    \begin{figure}[h!]
        \centering
        \begin{tikzpicture}[scale = 1]


\node[small_grid_point] at (0,0) (0){}; 

% right grid point
\node[small_grid_point_right] at (0,1)  {}; 
\node[small_grid_point_right] at (0,2) {}; 
\node[small_grid_point_right] at (1,-2) {}; 
\node[small_grid_point_right] at (1,-1) {}; 
\node[small_grid_point_right] at (1,0) (10){}; 
\node[small_grid_point_right] at (1,1) (1) {}; 
\node[small_grid_point_right] at (1,2) {}; 
\node[small_grid_point_right] at (2,-2) {}; 
\node[small_grid_point_right] at (2,-1) (3){}; 
\node[small_grid_point_right] at (2,0) (20){}; 
\node[small_grid_point_right] at (2,1) {}; 
\node[small_grid_point_right] at (2,2) {}; 

%%% left grid
\node[small_grid_point_left] at (0,-1) {}; 
\node[small_grid_point_left] at (0,-2) {}; 
\node[small_grid_point_left] at (-1,-2) {}; 
\node[small_grid_point_left] at (-1,-1) (2){}; 
\node[small_grid_point_left] at (-1,0) {}; 
\node[small_grid_point_left] at (-1,1) {}; 
\node[small_grid_point_left] at (-1,2) {}; 
\node[small_grid_point_left] at (-2,-2) {}; 
\node[small_grid_point_left] at (-2,-1) {}; 
\node[small_grid_point_left] at (-2,0) {}; 
\node[small_grid_point_left] at (-2,1) (4){}; 
\node[small_grid_point_left] at (-2,2) {}; 

\draw[arrow_grid_out] (10) -- (20);
\draw[arrow_grid_out] (0) -- (10);

%enumeration
\node[draw = none, scale = 1] at (0.3,0.3) {0};
%%% right grid
\node[draw = none, scale = 1] at (0.3,1.3) {1};
\node[draw = none, scale = 1] at (0.3,2.3) {2};
\node[draw = none, scale = 1] at (1.3,-1.7) {3};
\node[draw = none, scale = 1] at (1.3,-0.7) {4};
\node[draw = none, scale = 1] at (1.3,0.3) {5};
\node[draw = none, scale = 1] at (1.3,1.3) {6};
\node[draw = none, scale = 1] at (1.3,2.3) {7};
\node[draw = none, scale = 1] at (2.3,-1.7) {8};
\node[draw = none, scale = 1] at (2.3,-0.7) {9};
\node[draw = none, scale = 1] at (2.3,0.3) {10};
\node[draw = none, scale = 1] at (2.3,1.3) {11};
\node[draw = none, scale = 1] at (2.3,2.3) {12};
%%% left grid
\node[draw = none, scale = 1] at (0.3,-0.7) {13};
\node[draw = none, scale = 1] at (0.3,-1.7) {14};
\node[draw = none, scale = 1] at (-0.7,-1.7) {19};
\node[draw = none, scale = 1] at (-0.7,-0.7) {18};
\node[draw = none, scale = 1] at (-0.7,0.3) {17};
\node[draw = none, scale = 1] at (-0.7,1.3) {16};
\node[draw = none, scale = 1] at (-0.7,2.3) {15};
\node[draw = none, scale = 1] at (-1.7,-1.7) {24};
\node[draw = none, scale = 1] at (-1.7,-0.7) {23};
\node[draw = none, scale = 1] at (-1.7,0.3) {22}; 
\node[draw = none, scale = 1] at (-1.7,1.3) {21};
\node[draw = none, scale = 1] at (-1.7,2.3) {20};
\end{tikzpicture}
    \end{figure}
    
    \end{frame}
    
    \begin{frame}{Grundlegende Idee}
    \begin{itemize}
    \item Graphen als Produkte von Graßmann-Zahlen darstellen.
    \item[]
    \pause
    \item Finde Graßmann Wirkung $A$ sodass
    \item[]
    \end{itemize}
    \begin{equation} \nonumber
    \Sp{e^A} \overset{!}{=} \Xi[t] = \sum_{\{ G\}} t^{N_K(G)}
    \end{equation}
    \end{frame}
    
    \begin{frame}{Graphen mit Graßmann Zahlen}
    \centering
    Ordnen jedem Gitterpunkt 4 Graßmann Generatoren zu. 
    \vspace{.5cm}
    \begin{figure}[h]
        \centering
        \begin{tikzpicture}[scale = 0.6]
            \node[draw = none] at (0,0) (center) {$\sigma_i$} ;
            \node[draw, circle, fill=none, scale = 0.5, very thick] at (-1,0) (center) {} ;
            \node[draw, circle, fill=none, scale = 0.5, very thick] at (0,-1)  {} ;
            \node[draw, cross out, very thick, scale = 0.6] at (1,0)  {} ;
            \node[draw, cross out, very thick, scale = 0.6] at (0,1)  {} ;
            \node[draw=none] at (-2,0)  {$h_{i}^o$} ;
            \node[draw=none] at (0,-2)  {$v_{i}^o$} ;
            \node[draw=none] at (2,0) {$h_{i}^x$} ;
            \node[draw=none] at (0,2) {$v_{i}^x$} ;
        \end{tikzpicture}
    \end{figure}
    \end{frame}

    \begin{frame}{Graphen mit Graßmann Zahlen}

        \begin{figure}[h]
        \centering
        \input{Graphix/GV_Pairs_gh.tex}
        \end{figure}
        
    \end{frame}
    
    \begin{frame}{Graphen mit Graßmann Zahlen}

    Jeder Graph auf dem Gitter lässt sich darstellen als

    \begin{equation} \nonumber
    G = \prod_{( \bm{x}_i,\bm{x}_i + \bm{e}_y) \in K} \visible<3-5>{\textbf<3>{t}}\; P^{(h)}_{\bm{x}_i} \prod_{(\bm{x}_i,\bm{x}_i + \bm{e}_x) \in K} \visible<3-5>{\textbf<3>{t}}\; P^{(g)}_{\bm{x}_i}
    \end{equation}

    \visible<2-5>{
        Spur der Graßmann Graphen
        \only<1-3>{\begin{equation} \nonumber
        \Sp{G} \overset{!}{=} t^{N_k(G)}
        \end{equation}}
        \only<4-5>{\begin{equation} \nonumber
        \Sp{G} = 0
        \end{equation}}
    }
    \centering 
    
    \vspace{0.5cm}
    
    \visible<5>{\large
       \textbf{Es müssen noch mehr Paare angehängt werden !}
    }
    \end{frame}
    
    \begin{frame}{Graphen mit Graßmann Zahlen}

        \begin{figure}[h]
        \centering
        \input{Graphix/GV_Pairs_a_f.tex}
        \end{figure}
        
    \end{frame}

    \begin{frame}{Graphen mit Graßmann Zahlen}

    \begin{itemize}
    \item Jeder Graph auf Gitter als Graßmann Monom darstellbar
    \item[]
    \pause
    \item Darstellung ist nicht eindeutig !
    \item[]
    \pause
    \item Paar-Gewichte müssen so gewählt werden, dass die Spur das Graphen-Gewicht ergibt.  
    \end{itemize}
    \end{frame}

    \begin{frame}{Graphen mit Graßmann Zahlen}
    
    $\Sp{G}$ verschwindet wenn: 
    \begin{itemize}
        \item[]
        \item[i)] Eine Kante im Graphen doppelt durchlaufen wird.
        \item[]
        \item<2-3>[ii)] Der Graph nicht geschlossen ist.
    \end{itemize}
    
    \vspace{0.8cm}
    
    \visible<3>{ \large  \centering
        \textbf{Die Graßmann Graphen haben genau die gesuchten Eigenschaften!}
    }    
    \end{frame}

    \begin{frame}{Exponentialfunktion als Generator aller Graphen}
    \begin{align} \nonumber
    \exp{\sum_{i\in\Lambda} a_i P_i} \visible<2>{= \prod_{i\in\Lambda} \exp{a_i P_i} = \prod_{i\in\Lambda} 1 + a_i P_i }
    \end{align}
    \end{frame}
    
    \begin{frame}{Graßmann-Wirkung}
    Definieren Graßmann Wirkungen:
    
    \begin{equation} \nonumber
    \begin{aligned}
        A_{gh} &= \sum_{\bm{x}_i \in \Lambda} 
            \; t \; h_{\bm{x}_i}^x \,h_{\bm{x}_i+\bm{e}_x}^o
            + t \; v_{\bm{x}_i}^x \,v_{\bm{x}_i+\bm{e}_y}^o \\
        A_{abcd} &= - \sum_{\bm{x}_i \in \Lambda}  
             h_{\bm{x}_i}^x \,v_{\bm{x}_i}^o  
            +  v_{\bm{x}_i}^x\, h_{\bm{x}_i}^o 
            +  v_{\bm{x}_i}^o \,h_{\bm{x}_i}^o 
            +  v_{\bm{x}_i}^x \,h_{\bm{x}_i}^x\\
        A_{ef} &= - \sum_{\bm{x}_i \in \Lambda} 
             h_{\bm{x}_i}^o \,h_{\bm{x}_i}^x 
            +  v_{\bm{x}_i}^o \,v_{\bm{x}_i}^x
    \end{aligned}
    \end{equation}
    

    
   \pause
    \begin{grayframe}[frametitle = {Zustandssumme mit Graßmann Variablen}]
    \begin{equation} \nonumber
    \sum_{\{ G\}} t^{N_K(G)} \approx (-1)^N\; \Sp{ \,\mathrm{e}^{A_{gh} + A_{abcd} + A_{ef}}}
    \end{equation}
    \end{grayframe}
    
    \end{frame}
    
        \begin{frame}{Transformation der Wirkung}
    
    Transformieren
    
    \begin{equation} \nonumber
     A = A_{gh} + A_{abcd} + A_{ef} 
    \end{equation}
    
    mit Fouriertransformation  $U$
        \begin{align} \nonumber
            \bm{x}_i &\overset{U}{\longleftrightarrow} \bm{k}_i \\ \nonumber
            h_{\bm{x}_i}^x, h_{\bm{x}_i}^o, v_{\bm{x}_i}^x, v_{\bm{x}_i}^o &\overset{U}{\longrightarrow} \hat h_{\bm{k}_i}^x, \hat h_{\bm{k}_i}^o, \hat v_{\bm{k}_i}^x, \hat v_{\bm{k}_i}^o \\ \nonumber
           A &\overset{U}{\longrightarrow} \hat A
        \end{align}
    
    Die Spur bleibt erhalten
        \begin{align} \nonumber
            \Sp{e^A} = \Sp{e^{\hat{A}}}
        \end{align}

    \end{frame}
    
    
    
    
    
    
    
    
    \begin{frame}{Darstellende Matrix}
    \begin{equation} \nonumber
    \renewcommand{\arraystretch}{1.3}
    \bm{\hat{A}} = \frac{1}{2}
        \left(\begin{array}{c|ccc|ccc} 
        \bm{\hat{A}}_{\bm{k}_0}  & \bm{0}   & \cdots    & \bm{0} & \bm{0} & \cdots  & \bm{0} \\ \hline
        \bm{0}  & \bm{0}       & \cdots   & \bm{0}    & \bm{\hat{A}}_{\bm{k}_1} & \cdots  & \bm{0} \\
        \vdots  &\vdots        & \ddots   & \vdots    & \vdots                  & \ddots  & \vdots \\
        \bm{0}  & \bm{0}       & \cdots   & \bm{0} & \bm{0} & \cdots  & \bm{\hat{A}}_{\bm{k}_{2M(M+1)}}  \\ \hline
        \bm{0}  & -\bm{\hat{A}}^T_{\bm{k}_1} & \cdots  & \bm{0} & \bm{0} & \cdots  & \bm{0} \\
        \vdots  &\vdots        & \ddots   & \vdots    & \vdots                  & \ddots  & \vdots \\
        \bm{0}  & \bm{0}       & \cdots   &-\bm{\hat{A}}_{\bm{k}_{2M(M+1)}}^T & \bm{0} & \cdots  & \bm{0} 
        \end{array} \right) 
    \end{equation}
    \vspace{0.5cm}
    \centering
    $\hat{\bm{A}}_{\bm{k}}$ sind komplexe $4\times 4$ Matrizen
    %\pause
    %\begin{equation}\nonumber
    %    \bm{\hat{A}}_{\bm{k}} = \left(\begin{array}{cccc} 
    %        0         &-1 - t\,\mathrm{e}^{\,-ik_1}   &  1       & 1        \\
    %        1 + t\,\mathrm{e}^{\,ik_1} &0         &  -1       &  1        \\
    %        -1        &1        &  0       & -1 - t\,\mathrm{e}^{\,-ik_2}   \\
    %        -1         &-1        &1 + t\,\mathrm{e}^{\,ik_2} &  0        \\
    %    \end{array}\right) 
    %\end{equation}

    \end{frame}
        
\subsection{Berechnung der Spin-Spin Korrelation}

    \begin{frame}{Spin-Spin Korrelation}
    \begin{equation} \nonumber
        \corr{\sigma_{0,0} \sigma_{m,0}}  = \frac{t^m \Xi[\tilde{t}_{i,j}]}{\Xi[t]}  \approx \frac{t^m \, \Sp{\mathrm{e}^{A_D}}}{\Sp{\mathrm{e}^A}} 
    \end{equation}
    \vspace{0.5cm}
    \begin{equation} \nonumber
        A_D = A - \sum_{\bm{x}} t\, h_{\bm{x}}^x\, h_{\bm{x} + \bm{e}_x}^o + \sum_{\bm{x}} t^{-1}\, h_{\bm{x}}^x\, h_{\bm{x} + \bm{e}_x}^o
    \end{equation}
    
    \vspace{0.5cm}

%    \begin{equation} \nonumber
%    \corr{\sigma_{x,y} \sigma_{x+m,y}}  =  t^m \, \corr{\prod_{l = 0}^{m-1} 1 + (t^{-1} - t)\, h_{l,0}^x\, h_{l+1,0}^o} 
%    \end{equation}
    
    \end{frame}

    \begin{frame}{Spin-Spin Korrelation als Tölpliz Determinante}
        \begin{equation} \nonumber
            \frac{t^m \, \Sp{\mathrm{e}^{A_D}}}{\Sp{\mathrm{e}^A}} = \cdots = \det{\bm{C}^{(m)}} 
        \end{equation}

        \begin{grayframe}
         \vspace{0.2cm}
        \begin{equation} \nonumber
            C_{l,l'}^{(m)} = t \delta_{l,l'} + (1-t^2)\sum_{\bm{k} \in \bar{\Lambda}}  \mathrm{e}^{-i\,(l-l'-1) \,k_1} \,\corr{\hat{h}_{\bm{k}}^{x}\, \hat{h}_{-\bm{k}}^{o}} \frac{\Delta k_1}{2\pi}\frac{\Delta k_2}{2\pi} % \corr{h_{l-1,0}^x h_{l',0}^o}
        \end{equation}
        \end{grayframe}
        
        \vspace{0.1cm}
        \vspace{0.5cm}
        \centering
        $\bm{C^{(m)}}$ ist eine $m\times m$ Tölplitz Matrix
    \end{frame}

    \begin{frame}{Graßmann-Korrelationen Berechnen}
    
    \begin{equation} \nonumber
    \corr{\hat{h}_{\bm{k}}^{x}\, \hat{h}_{-\bm{k}}^{o}}_{\Lambda_N} = \frac{\Sp{\mathrm{e}^{\hat{A}}\;\;\hat{h}_{\bm{k}}^{x}\, \hat{h}_{-\bm{k}}^{o}}}{\Sp{\mathrm{e}^{\hat{A}}}} = \left(\bm{A}_{\bm{k}}^{-1}\right)_{2,1} 
    \end{equation} 
    \end{frame}

    \begin{frame}{Thermodynamischer Limes}
    \begin{equation} \nonumber
        \sum_{\bm{k} \in \bar{\Lambda}}  \mathrm{e}^{-i\,(l-l') \,k_1} \,\corr{\hat{h}_{\bm{k}}^{x}\, \hat{h}_{-\bm{k}}^{o}}_{\Lambda_N} \frac{\Delta k_1}{2\pi}\frac{\Delta k_2}{2\pi}
    \end{equation}
    
    \begin{equation}\nonumber
   	{\tiny \begin{array}{c} \Delta k_1\\\downarrow\\0 \end{array} }{\Big\downarrow}{ \tiny\begin{array}{c}\Delta k_2\\\downarrow\\0 \end{array}}
    \end{equation}
    
    \begin{equation} \nonumber
     \frac{1}{(2\pi)^2}\int_{-\pi}^{\pi} \int_{-\pi}^{\pi} \d k_1 \d k_2 \,\,\mathrm{e}^{-i\, (l-l') \,k_1} \corr{\hat{h}_{\bm{k}}^{x}\, \hat{h}_{-\bm{k}}^{o}}
    \end{equation} 
    \end{frame}


\subsection{Berechnung der Magnetisierung}

    \begin{frame}{Korrelationsmatrix in Fourierdarstellung}

        \begin{equation} \nonumber
            \left(\frac{\mathcal{M}_S}{ n \mu}\right)^2 = \lim_{m \rightarrow \infty} \corr{\sigma_{(0,0)} \sigma_{(m,0)}} = \lim_{m \rightarrow \infty} \det{\bm{C}^{(m)}}
        \end{equation}
    
        \pause 
        
        \begin{grayframe}
            \vspace{0.2cm}
            \begin{equation} \nonumber
                C^{(m)}_{l,l+r} = \frac{1}{2\pi} \int_{-\pi}^{\pi}\d\omega  \;\mathrm{e}^{-i  r \omega} \mathrm{e}^{i\delta^*}
            \end{equation}
        \end{grayframe}   
        
        \vspace{0.5cm}
        
        \pause
        \begin{equation} \nonumber
            i\delta^*(\omega) = \frac{1}{2} \ln{\frac{(1-tt^*\mathrm{e}^{i\omega})(1-\frac{t^*}{t}\mathrm{e}^{-i\omega})}{(1-tt^*\mathrm{e}^{-i\omega})(1-\frac{t^*}{t}\mathrm{e}^{i\omega})}}
        \end{equation}

        \vspace{0.3cm}
        
        \begin{equation}\nonumber
            \begin{array}{ccccc}
                t = \tanh(\beta J) & & t^* = \frac{1-t}{1+t}    & &             \omega = -k_1
            \end{array}
        \end{equation}
    
    \end{frame}

    \begin{frame}{Starker Grenzwertsatz von Szegö}
       %\cite{StrongSzegoeTheorem_Hirschman} 
        Für 
        \begin{equation} \nonumber
            f(\omega) = \sum_{k = -\infty}^{\infty} \hat{f}(k)\mathrm{e}^{i k \omega}
        \end{equation}
                \pause
        gilt mit
        \begin{equation} \nonumber
            D_m(f) = \det{\bm{M}} \;\;\text{ mit}\,\, M_{i,j} = \hat{f}(i-j) \;\;,\;\; 1<i,j<m
        \end{equation}
        im Limes $m \rightarrow \infty$ 
                \pause
        \begin{grayframe}
        \vspace{0.4cm}
        \begin{equation} \nonumber
            \ln{D_m(f)} = (m+1)\hat{s}(0) + \sum_{k = 1}^{\infty} k\; \hat{s}(k)\;\hat{s}(-k) + { \scriptstyle \mathcal{O}}(1)
        \end{equation}

        \begin{equation} \nonumber
            \hat{s}(k) = \frac{1}{2 \pi }  \int_{-\pi}^{\pi} \ln{f(\omega)} \mathrm{e}^{-i k \omega} d \omega 
        \end{equation}
        \end{grayframe}
    \end{frame}
    
    \begin{frame}{Kritische Temperatur}
    
    \begin{equation} \nonumber
        \begin{array}{ccc}
        f(\omega) = \mathrm{e}^{i\delta^*(\omega)} &\Rightarrow& \ln{f(\omega)} = i\delta^*(\omega)
        \end{array}
    \end{equation}
        
        \visible<2-4>{
        \begin{align} \nonumber
        2i\delta^*(\omega) =& \ln{1-{\color<3-4> {red} tt^*}\mathrm{e}^{i\omega}} + \ln{1-{\color<3-4> {blue} \frac{t^*}{t}}\mathrm{e}^{-i\omega}}\\ - &\ln{1-{\color<3-4> {red} tt^*}\mathrm{e}^{-i\omega}} - \ln{1-{\color<3-4> {blue} \frac{t^*}{t}}\mathrm{e}^{i\omega}} \nonumber
        \end{align}}
        
        \visible<4>{
        \begin{grayframe}
            \begin{equation} \nonumber
                T_C  := \frac{2 J}{ \ln{1+\sqrt{2}} k_B}
            \end{equation}
        \end{grayframe}} 
    \end{frame}
    
    \begin{frame}{Magnetisierung}
    \Large
    
    \begin{equation} \nonumber
    \frac{\mathcal{M}_S^2}{ n \mu} = \left\{ \begin{array}{cr} \left(1-\frac{1}{\sinh(\frac{2J}{k_B\,T})^4}\right)^{\frac{1}{4}} & \text{für } T < T_c \;\text{bzw. }\;\; t^* < t \\ & \\0 &\text{für } T > T_c  \;\text{bzw. }\;\; t^* > t \end{array} \right.
    \end{equation}
    
    \end{frame}
    
        
\section{Zusammenfassung}
    
        \begin{frame}{Spontane Magnetisierung des 2d Ising-Modells}
    
    \begin{grayframe}
    {\Large
    \begin{equation} \nonumber
    \mathcal{M}_S = \left\{ \begin{array}{cr} n \mu \left(1-\frac{1}{\sinh(\frac{2J}{k_B\,T})^4}\right)^{\frac{1}{8}} & \text{für } T < T_c \\ 0 &\text{für } T > T_c   \end{array} \right.
    \end{equation}}
    \begin{equation} \nonumber
    T_C  := \frac{2 J}{ \ln{1+\sqrt{2}} k_B}
    \end{equation}
\end{grayframe}
    
    \end{frame}

    \begin{frame}{Zusammenfassung: Was wurde gemacht}
    \begin{itemize} 
    \item Graphisch kombinatorisches Problem auf algebraisches Problem abgebildet. (Graßmann-Zahlen)
    \item[]
    \item Periodizität erlaubt Vereinfachung. (Fourier-Transformation, Töplitz-Determiante)
    \item[]
    \item Starke Szegö Grenzwertsatz für
    langreichweitigen Limes.
    \item[]
    \end{itemize}
    \end{frame}
    
    \begin{frame}{Zusammenfassung: Qualität der Methode}
    
    \begin{itemize}
    \item Transkription des Modells in handhabbares algebraisches Problem. 
    \item[]
    \item Berechnung der Zustandssumme
    \item[]
    \item Berechnung der Spin-Spin-Korrelationen
    \item[]
    \item Erweiterung auf größere Spin-Korrelationen
    \item[]
    \item Anwendbar auf andere Probleme wie das 
    ``closed packed dimer problem''
    \item[]
    \end{itemize}
    
    
    \end{frame}
    
    \begin{frame}
        \centering 
        \only<2>{\LARGE \textbf{Danke für ihre Aufmerksamkeit !}}
        \only<1>{\LARGE \textbf{Gibt es Fragen ?}}
    \end{frame}

    \begin{frame}{Vorraussetzungen Grenzwertsatz von Szegö}
        \begin{itemize}
            \item Konvergenz der Fourier-Koeffizienten
            \begin{equation} \nonumber
                \sum_{k = -\infty}^{\infty} |\hat{f}(k)| < \infty \;\;\;\text{, }\;\;\;
                \sum_{k = -\infty}^{\infty} |\hat{f}(k)|^2 |k| < \infty 
            \end{equation}
            
            \item Keine Nullstellen im Definitionsbereich
            \begin{equation} \nonumber
            \forall \omega \in [-\pi,\pi]\;\; :\;\; f(\omega) \neq 0
            \end{equation}
            
            \item Verschwindende Windungszahl 
            \begin{equation} \nonumber
            \left[\mathrm{arg}\left( f\right)\right]_{-\pi}^{\pi} = 0 
            \end{equation}
            \end{itemize}
        \end{frame}

\end{document}